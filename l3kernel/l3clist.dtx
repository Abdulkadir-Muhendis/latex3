% \iffalse meta-comment
%
%% File: l3clist.dtx
%
% Copyright (C) 2004-2011 Frank Mittelbach, The LaTeX3 Project
%           (C) 2012-2019 The LaTeX3 Project
%
% It may be distributed and/or modified under the conditions of the
% LaTeX Project Public License (LPPL), either version 1.3c of this
% license or (at your option) any later version.  The latest version
% of this license is in the file
%
%    https://www.latex-project.org/lppl.txt
%
% This file is part of the "l3kernel bundle" (The Work in LPPL)
% and all files in that bundle must be distributed together.
%
% -----------------------------------------------------------------------
%
% The development version of the bundle can be found at
%
%    https://github.com/latex3/latex3
%
% for those people who are interested.
%
%<*driver>
\documentclass[full,kernel]{l3doc}
\begin{document}
  \DocInput{\jobname.dtx}
\end{document}
%</driver>
% \fi
%
% \title{^^A
%   The \pkg{l3clist} package\\ Comma separated lists^^A
% }
%
% \author{^^A
%  The \LaTeX3 Project\thanks
%    {^^A
%      E-mail:
%        \href{mailto:latex-team@latex-project.org}
%          {latex-team@latex-project.org}^^A
%    }^^A
% }
%
% \date{Released 2019-10-02}
%
% \maketitle
%
% \begin{documentation}
%
% Comma lists contain ordered data where items can be added to the left
% or right end of the list.  This data type allows basic list
% manipulations such as adding/removing items, applying a function to
% every item, removing duplicate items, extracting a given item, using
% the comma list with specified separators, and so on.  Sequences
% (defined in \pkg{l3seq}) are safer, faster, and provide more features,
% so they should often be preferred to comma lists.  Comma lists are
% mostly useful when interfacing with \LaTeXe{} or other code that
% expects or provides comma list data.
%
% Several items can be added at once.  To ease input of comma lists from
% data provided by a user outside an \cs{ExplSyntaxOn} \ldots{}
% \cs{ExplSyntaxOff} block, spaces are removed from both sides of each
% comma-delimited argument upon input.  Blank arguments are ignored, to
% allow for trailing commas or repeated commas (which may otherwise
% arise when concatenating comma lists \enquote{by hand}).  In addition,
% a set of braces is removed if the result of space-trimming is braced:
% this allows the storage of any item in a comma list.  For instance,
% \begin{verbatim}
%   \clist_new:N \l_my_clist
%   \clist_put_left:Nn \l_my_clist { ~a~ , ~{b}~ , c~\d }
%   \clist_put_right:Nn \l_my_clist { ~{e~} , , {{f}} , }
% \end{verbatim}
% results in |\l_my_clist| containing |a,b,c~\d,{e~},{{f}}| namely the
% five items |a|, |b|, |c~\d|, |e~| and~|{f}|.  Comma lists normally do
% not contain empty items so the following gives an empty comma list:
% \begin{verbatim}
%   \clist_clear_new:N \l_my_clist
%   \clist_put_right:Nn \l_my_clist { , ~ , , }
%   \clist_if_empty:NTF \l_my_clist { true } { false }
% \end{verbatim}
% and it leaves \texttt{true} in the input stream.  To include an
% \enquote{unsafe} item (empty, or one that contains a comma, or starts
% or ends with a space, or is a single brace group), surround it with
% braces.
%
% Almost all operations on comma lists are
% noticeably slower than those on sequences so converting the data to
% sequences using \cs{seq_set_from_clist:Nn} (see \pkg{l3seq}) may be
% advisable if speed is important.  The exception is that
% \cs{clist_if_in:NnTF} and \cs{clist_remove_duplicates:N} may be faster
% than their sequence analogues for large lists.  However, these
% functions work slowly for \enquote{unsafe} items that must be braced,
% and may produce errors when their argument contains |{|, |}| or |#|
% (assuming the usual \TeX{} category codes apply).  In addition, comma
% lists cannot store quarks \cs{q_mark} or \cs{q_stop}.  The sequence
% data type should thus certainly be preferred to comma lists to store
% such items.
%
% \section{Creating and initialising comma lists}
%
% \begin{function}{\clist_new:N, \clist_new:c}
%   \begin{syntax}
%     \cs{clist_new:N} \meta{comma list}
%   \end{syntax}
%   Creates a new \meta{comma list} or raises an error if the name is
%   already taken. The declaration is global. The \meta{comma list}
%   initially contains no items.
% \end{function}
%
% \begin{function}[added = 2014-07-05]
%   {
%     \clist_const:Nn, \clist_const:Nx,
%     \clist_const:cn, \clist_const:cx
%   }
%   \begin{syntax}
%     \cs{clist_const:Nn} \meta{clist~var} \Arg{comma list}
%   \end{syntax}
%   Creates a new constant \meta{clist~var} or raises an error
%   if the name is already taken. The value of the
%   \meta{clist~var} is set globally to the
%   \meta{comma list}.
% \end{function}
%
% \begin{function}
%   {\clist_clear:N, \clist_clear:c, \clist_gclear:N, \clist_gclear:c}
%   \begin{syntax}
%     \cs{clist_clear:N} \meta{comma list}
%   \end{syntax}
%   Clears all items from the \meta{comma list}.
% \end{function}
%
% \begin{function}
%   {
%     \clist_clear_new:N,  \clist_clear_new:c,
%     \clist_gclear_new:N, \clist_gclear_new:c
%   }
%   \begin{syntax}
%     \cs{clist_clear_new:N} \meta{comma list}
%   \end{syntax}
%   Ensures that the \meta{comma list} exists globally by applying
%   \cs{clist_new:N} if necessary, then applies
%   \cs[index=clist_clear:N]{clist_(g)clear:N} to leave
%   the list empty.
% \end{function}
%
% \begin{function}
%   {
%     \clist_set_eq:NN,  \clist_set_eq:cN,
%     \clist_set_eq:Nc,  \clist_set_eq:cc,
%     \clist_gset_eq:NN, \clist_gset_eq:cN,
%     \clist_gset_eq:Nc, \clist_gset_eq:cc
%   }
%   \begin{syntax}
%     \cs{clist_set_eq:NN} \meta{comma list_1} \meta{comma list_2}
%   \end{syntax}
%   Sets the content of \meta{comma list_1} equal to that of
%   \meta{comma list_2}.
% \end{function}
%
% \begin{function}[added = 2014-07-17]
%   {
%     \clist_set_from_seq:NN,  \clist_set_from_seq:cN,
%     \clist_set_from_seq:Nc,  \clist_set_from_seq:cc,
%     \clist_gset_from_seq:NN, \clist_gset_from_seq:cN,
%     \clist_gset_from_seq:Nc, \clist_gset_from_seq:cc
%   }
%   \begin{syntax}
%     \cs{clist_set_from_seq:NN} \meta{comma list} \meta{sequence}
%   \end{syntax}
%   Converts the data in the \meta{sequence} into a \meta{comma list}:
%   the original \meta{sequence} is unchanged.
%   Items which contain either spaces or commas are surrounded by braces.
% \end{function}
%
% \begin{function}
%   {
%     \clist_concat:NNN,  \clist_concat:ccc,
%     \clist_gconcat:NNN, \clist_gconcat:ccc
%   }
%   \begin{syntax}
%     \cs{clist_concat:NNN} \meta{comma list_1} \meta{comma list_2} \meta{comma list_3}
%   \end{syntax}
%   Concatenates the content of \meta{comma list_2} and \meta{comma list_3}
%   together and saves the result in \meta{comma list_1}. The items in
%   \meta{comma list_2} are placed at the left side of the new comma list.
% \end{function}
%
% \begin{function}[EXP, pTF, added=2012-03-03]
%   {\clist_if_exist:N, \clist_if_exist:c}
%   \begin{syntax}
%     \cs{clist_if_exist_p:N} \meta{comma list}
%     \cs{clist_if_exist:NTF} \meta{comma list} \Arg{true code} \Arg{false code}
%   \end{syntax}
%   Tests whether the \meta{comma list} is currently defined.  This does
%   not check that the \meta{comma list} really is a comma list.
% \end{function}
%
% \section{Adding data to comma lists}
%
% \begin{function}[added = 2011-09-06]
%   {
%     \clist_set:Nn,  \clist_set:NV,
%     \clist_set:No,  \clist_set:Nx,
%     \clist_set:cn,  \clist_set:cV,
%     \clist_set:co,  \clist_set:cx,
%     \clist_gset:Nn, \clist_gset:NV,
%     \clist_gset:No, \clist_gset:Nx,
%     \clist_gset:cn, \clist_gset:cV,
%     \clist_gset:co, \clist_gset:cx
%   }
%   \begin{syntax}
%     \cs{clist_set:Nn} \meta{comma list} |{|\meta{item_1},\ldots{},\meta{item_n}|}|
%   \end{syntax}
%   Sets \meta{comma list} to contain the \meta{items},
%   removing any previous content from the variable.
%   Blank items are omitted, spaces are removed from both sides of each
%   item, then a set of braces is removed if the resulting space-trimmed
%   item is braced.
%   To store some \meta{tokens} as a single \meta{item} even if the
%   \meta{tokens} contain commas or spaces, add a set of braces:
%   \cs{clist_set:Nn} \meta{comma list} |{| \Arg{tokens} |}|.
% \end{function}
%
% \begin{function}[updated = 2011-09-05]
%   {
%     \clist_put_left:Nn,  \clist_put_left:NV,
%     \clist_put_left:No,  \clist_put_left:Nx,
%     \clist_put_left:cn,  \clist_put_left:cV,
%     \clist_put_left:co,  \clist_put_left:cx,
%     \clist_gput_left:Nn, \clist_gput_left:NV,
%     \clist_gput_left:No, \clist_gput_left:Nx,
%     \clist_gput_left:cn, \clist_gput_left:cV,
%     \clist_gput_left:co, \clist_gput_left:cx
%   }
%   \begin{syntax}
%     \cs{clist_put_left:Nn} \meta{comma list} |{|\meta{item_1},\ldots{},\meta{item_n}|}|
%   \end{syntax}
%   Appends the \meta{items} to the left of the \meta{comma list}.
%   Blank items are omitted, spaces are removed from both sides of each
%   item, then a set of braces is removed if the resulting space-trimmed
%   item is braced.
%   To append some \meta{tokens} as a single \meta{item} even if the
%   \meta{tokens} contain commas or spaces, add a set of braces:
%   \cs{clist_put_left:Nn} \meta{comma list} |{| \Arg{tokens} |}|.
% \end{function}
%
% \begin{function}[updated = 2011-09-05]
%   {
%     \clist_put_right:Nn,  \clist_put_right:NV,
%     \clist_put_right:No,  \clist_put_right:Nx,
%     \clist_put_right:cn,  \clist_put_right:cV,
%     \clist_put_right:co,  \clist_put_right:cx,
%     \clist_gput_right:Nn, \clist_gput_right:NV,
%     \clist_gput_right:No, \clist_gput_right:Nx,
%     \clist_gput_right:cn, \clist_gput_right:cV,
%     \clist_gput_right:co, \clist_gput_right:cx
%   }
%   \begin{syntax}
%     \cs{clist_put_right:Nn} \meta{comma list} |{|\meta{item_1},\ldots{},\meta{item_n}|}|
%   \end{syntax}
%   Appends the \meta{items} to the right of the \meta{comma list}.
%   Blank items are omitted, spaces are removed from both sides of each
%   item, then a set of braces is removed if the resulting space-trimmed
%   item is braced.
%   To append some \meta{tokens} as a single \meta{item} even if the
%   \meta{tokens} contain commas or spaces, add a set of braces:
%   \cs{clist_put_right:Nn} \meta{comma list} |{| \Arg{tokens} |}|.
% \end{function}
%
% \section{Modifying comma lists}
%
% While comma lists are normally used as ordered lists, it may be
% necessary to modify the content. The functions here may be used
% to update comma lists, while retaining the order of the unaffected
% entries.
%
% \begin{function}
%   {
%     \clist_remove_duplicates:N,  \clist_remove_duplicates:c,
%     \clist_gremove_duplicates:N, \clist_gremove_duplicates:c
%   }
%   \begin{syntax}
%     \cs{clist_remove_duplicates:N} \meta{comma list}
%   \end{syntax}
%   Removes duplicate items from the \meta{comma list}, leaving the
%   left most copy of each item in the \meta{comma list}.  The \meta{item}
%   comparison takes place on a token basis, as for \cs{tl_if_eq:nn(TF)}.
%   \begin{texnote}
%     This function iterates through every item in the \meta{comma list} and
%     does a comparison with the \meta{items} already checked. It is therefore
%     relatively slow with large comma lists.
%     Furthermore, it may fail if any of the items in the
%     \meta{comma list} contains |{|, |}|, or |#|
%     (assuming the usual \TeX{} category codes apply).
%   \end{texnote}
% \end{function}
%
% \begin{function}[updated = 2011-09-06]
%   {
%     \clist_remove_all:Nn,  \clist_remove_all:cn,
%     \clist_gremove_all:Nn, \clist_gremove_all:cn
%   }
%   \begin{syntax}
%     \cs{clist_remove_all:Nn} \meta{comma list} \Arg{item}
%   \end{syntax}
%   Removes every occurrence of \meta{item} from the \meta{comma list}.
%   The \meta{item} comparison takes place on a token basis, as for
%   \cs{tl_if_eq:nn(TF)}.
%   \begin{texnote}
%     The function may fail if the \meta{item} contains |{|, |}|, or |#|
%     (assuming the usual \TeX{} category codes apply).
%   \end{texnote}
% \end{function}
%
% \begin{function}[added = 2014-07-18]
%   {
%     \clist_reverse:N, \clist_reverse:c,
%     \clist_greverse:N, \clist_greverse:c
%   }
%   \begin{syntax}
%     \cs{clist_reverse:N} \meta{comma list}
%   \end{syntax}
%   Reverses the order of items stored in the \meta{comma list}.
% \end{function}
%
% \begin{function}[added = 2014-07-18]{\clist_reverse:n}
%   \begin{syntax}
%     \cs{clist_reverse:n} \Arg{comma list}
%   \end{syntax}
%   Leaves the items in the \meta{comma list} in the input stream in
%   reverse order.  Contrarily to other what is done for other
%   \texttt{n}-type \meta{comma list} arguments, braces and spaces are
%   preserved by this process.
%   \begin{texnote}
%     The result is returned within \tn{unexpanded}, which means that the
%     comma list does not expand further when appearing in an
%     \texttt{x}-type or \texttt{e}-type argument expansion.
%   \end{texnote}
% \end{function}
%
% \begin{function}[added = 2017-02-06]
%   {\clist_sort:Nn, \clist_sort:cn, \clist_gsort:Nn, \clist_gsort:cn}
%   \begin{syntax}
%     \cs{clist_sort:Nn} \meta{clist var} \Arg{comparison code}
%   \end{syntax}
%   Sorts the items in the \meta{clist var} according to the
%   \meta{comparison code}, and assigns the result to
%   \meta{clist var}. The details of sorting comparison are
%   described in Section~\ref{sec:l3sort:mech}.
% \end{function}
%
% \section{Comma list conditionals}
%
% \begin{function}[EXP,pTF]{\clist_if_empty:N, \clist_if_empty:c}
%   \begin{syntax}
%     \cs{clist_if_empty_p:N} \meta{comma list}
%     \cs{clist_if_empty:NTF} \meta{comma list} \Arg{true code} \Arg{false code}
%   \end{syntax}
%   Tests if the \meta{comma list} is empty (containing no items).
% \end{function}
%
% \begin{function}[EXP, pTF, added = 2014-07-05]{\clist_if_empty:n}
%   \begin{syntax}
%     \cs{clist_if_empty_p:n} \Arg{comma list}
%     \cs{clist_if_empty:nTF} \Arg{comma list} \Arg{true code} \Arg{false code}
%   \end{syntax}
%   Tests if the \meta{comma list} is empty (containing no items).
%   The rules for space trimming are as for other \texttt{n}-type
%   comma-list functions, hence the comma list |{~,~,,~}| (without
%   outer braces) is empty, while |{~,{},}| (without outer braces)
%   contains one element, which happens to be empty: the comma-list
%   is not empty.
% \end{function}
%
% \begin{function}[updated = 2011-09-06, TF]
%   {
%      \clist_if_in:Nn, \clist_if_in:NV, \clist_if_in:No,
%      \clist_if_in:cn, \clist_if_in:cV, \clist_if_in:co,
%      \clist_if_in:nn, \clist_if_in:nV, \clist_if_in:no
%   }
%   \begin{syntax}
%     \cs{clist_if_in:NnTF} \meta{comma list} \Arg{item} \Arg{true code} \Arg{false code}
%   \end{syntax}
%   Tests if the \meta{item} is present in the \meta{comma list}.
%   In the case of an \texttt{n}-type \meta{comma list}, the usual rules
%   of space trimming and brace stripping apply.  Hence,
%   \begin{verbatim}
%     \clist_if_in:nnTF { a , {b}~ , {b} , c } { b } {true} {false}
%   \end{verbatim}
%   yields \texttt{true}.
%   \begin{texnote}
%     The function may fail if the \meta{item} contains |{|, |}|, or |#|
%     (assuming the usual \TeX{} category codes apply).
%   \end{texnote}
% \end{function}
%
% \section{Mapping to comma lists}
%
% The functions described in this section apply a specified function
% to each item of a comma list.
% All mappings are done at the current group level, \emph{i.e.}~any
% local assignments made by the \meta{function} or \meta{code} discussed
% below remain in effect after the loop.
%
% When the comma list is given explicitly, as an \texttt{n}-type argument,
% spaces are trimmed around each item.
% If the result of trimming spaces is empty, the item is ignored.
% Otherwise, if the item is surrounded by braces, one set is removed,
% and the result is passed to the mapped function. Thus, if the
% comma list that is being mapped is \verb*|{a , {{b} }, ,{}, {c},}|
% then the arguments passed to the mapped function are
% `\verb*|a|', `\verb*|{b} |', an empty argument, and `\verb*|c|'.
%
% When the comma list is given as an \texttt{N}-type argument, spaces
% have already been trimmed on input, and items are simply stripped
% of one set of braces if any. This case is more efficient than using
% \texttt{n}-type comma lists.
%
% \begin{function}[rEXP, updated = 2012-06-29]
%   {\clist_map_function:NN, \clist_map_function:cN, \clist_map_function:nN}
%   \begin{syntax}
%     \cs{clist_map_function:NN} \meta{comma list} \meta{function}
%   \end{syntax}
%   Applies \meta{function} to every \meta{item} stored in the
%   \meta{comma list}. The \meta{function} receives one argument for
%   each iteration. The \meta{items} are returned from left to right.
%   The function \cs{clist_map_inline:Nn} is in general more efficient
%   than \cs{clist_map_function:NN}.
% \end{function}
%
% \begin{function}[updated = 2012-06-29]
%   {\clist_map_inline:Nn, \clist_map_inline:cn, \clist_map_inline:nn}
%   \begin{syntax}
%     \cs{clist_map_inline:Nn} \meta{comma list} \Arg{inline function}
%   \end{syntax}
%   Applies \meta{inline function} to every \meta{item} stored
%   within the \meta{comma list}. The \meta{inline function} should
%   consist of code which receives the \meta{item} as |#1|.
%   The \meta{items} are returned from left to right.
% \end{function}
%
% \begin{function}[updated = 2012-06-29]
%   {\clist_map_variable:NNn, \clist_map_variable:cNn, \clist_map_variable:nNn}
%   \begin{syntax}
%     \cs{clist_map_variable:NNn} \meta{comma list} \meta{variable} \Arg{code}
%   \end{syntax}
%   Stores each \meta{item} of the \meta{comma list} in turn in the
%   (token list) \meta{variable} and applies the \meta{code}.  The
%   \meta{code} will usually make use of the \meta{variable}, but this
%   is not enforced.  The assignments to the \meta{variable} are local.
%   Its value after the loop is the last \meta{item} in the \meta{comma
%   list}, or its original value if there were no \meta{item}.  The
%   \meta{items} are returned from left to right.
% \end{function}
%
% \begin{function}[rEXP, updated = 2012-06-29]{\clist_map_break:}
%   \begin{syntax}
%     \cs{clist_map_break:}
%   \end{syntax}
%   Used to terminate a \cs[no-index]{clist_map_\ldots{}} function before all
%   entries in the \meta{comma list} have been processed. This
%   normally takes place within a conditional statement, for example
%   \begin{verbatim}
%     \clist_map_inline:Nn \l_my_clist
%       {
%         \str_if_eq:nnTF { #1 } { bingo }
%           { \clist_map_break: }
%           {
%             % Do something useful
%           }
%       }
%   \end{verbatim}
%   Use outside of a \cs[no-index]{clist_map_\ldots{}} scenario leads to low
%   level \TeX{} errors.
%   \begin{texnote}
%     When the mapping is broken, additional tokens may be inserted
%     before further items are taken
%     from the input stream. This depends on the design of the mapping
%     function.
%   \end{texnote}
% \end{function}
%
% \begin{function}[updated = 2012-06-29, rEXP]{\clist_map_break:n}
%   \begin{syntax}
%     \cs{clist_map_break:n} \Arg{code}
%   \end{syntax}
%   Used to terminate a \cs[no-index]{clist_map_\ldots{}} function before all
%   entries in the \meta{comma list} have been processed, inserting
%   the \meta{code} after the mapping has ended. This
%   normally takes place within a conditional statement, for example
%   \begin{verbatim}
%     \clist_map_inline:Nn \l_my_clist
%       {
%         \str_if_eq:nnTF { #1 } { bingo }
%           { \clist_map_break:n { <code> } }
%           {
%             % Do something useful
%           }
%       }
%   \end{verbatim}
%   Use outside of a \cs[no-index]{clist_map_\ldots{}} scenario leads to low
%   level \TeX{} errors.
%   \begin{texnote}
%     When the mapping is broken, additional tokens may be inserted
%     before the \meta{code} is
%     inserted into the input stream.
%     This depends on the design of the mapping function.
%   \end{texnote}
% \end{function}
%
% \begin{function}[EXP, added = 2012-07-13]
%   {\clist_count:N, \clist_count:c, \clist_count:n}
%   \begin{syntax}
%     \cs{clist_count:N} \meta{comma list}
%   \end{syntax}
%   Leaves the number of items in the \meta{comma list} in the input
%   stream as an \meta{integer denotation}. The total number of items
%   in a \meta{comma list} includes those which are duplicates,
%   \emph{i.e.}~every item in a \meta{comma list} is counted.
% \end{function}
%
% \section{Using the content of comma lists directly}
%
% \begin{function}[EXP, added = 2013-05-26]{\clist_use:Nnnn, \clist_use:cnnn}
%   \begin{syntax}
%     \cs{clist_use:Nnnn} \meta{clist~var} \Arg{separator~between~two} \Arg{separator~between~more~than~two} \Arg{separator~between~final~two}
%   \end{syntax}
%   Places the contents of the \meta{clist~var} in the input stream,
%   with the appropriate \meta{separator} between the items.  Namely, if
%   the comma list has more than two items, the \meta{separator between
%     more than two} is placed between each pair of items except the
%   last, for which the \meta{separator between final two} is used.  If
%   the comma list has exactly two items, then they are placed in the input
%   stream separated by the \meta{separator between two}.  If the comma
%   list has a single item, it is placed in the input stream, and a comma
%   list with no items produces no output.  An error is raised if
%   the variable does not exist or if it is invalid.
%
%   For example,
%   \begin{verbatim}
%     \clist_set:Nn \l_tmpa_clist { a , b , , c , {de} , f }
%     \clist_use:Nnnn \l_tmpa_clist { ~and~ } { ,~ } { ,~and~ }
%   \end{verbatim}
%   inserts \enquote{\texttt{a, b, c, de, and f}} in the input
%   stream.  The first separator argument is not used in this case
%   because the comma list has more than $2$ items.
%   \begin{texnote}
%     The result is returned within the \tn{unexpanded}
%     primitive (\cs{exp_not:n}), which means that the \meta{items}
%     do not expand further when appearing in an \texttt{x}-type
%     argument expansion.
%   \end{texnote}
% \end{function}
%
% \begin{function}[EXP, added = 2013-05-26]{\clist_use:Nn, \clist_use:cn}
%   \begin{syntax}
%     \cs{clist_use:Nn} \meta{clist~var} \Arg{separator}
%   \end{syntax}
%   Places the contents of the \meta{clist~var} in the input stream,
%   with the \meta{separator} between the items. If the comma
%   list has a single item, it is placed in the input stream, and a comma
%   list with no items produces no output.  An error is raised if
%   the variable does not exist or if it is invalid.
%
%   For example,
%   \begin{verbatim}
%     \clist_set:Nn \l_tmpa_clist { a , b , , c , {de} , f }
%     \clist_use:Nn \l_tmpa_clist { ~and~ }
%   \end{verbatim}
%   inserts \enquote{\texttt{a and b and c and de and f}} in the input
%   stream.
%   \begin{texnote}
%     The result is returned within the \tn{unexpanded}
%     primitive (\cs{exp_not:n}), which means that the \meta{items}
%     do not expand further when appearing in an \texttt{x}-type
%     argument expansion.
%   \end{texnote}
% \end{function}
%
% \section{Comma lists as stacks}
%
% Comma lists can be used as stacks, where data is pushed to and popped
% from the top of the comma list. (The left of a comma list is the top, for
% performance reasons.) The stack functions for comma lists are not
% intended to be mixed with the general ordered data functions detailed
% in the previous section: a comma list should either be used as an
% ordered data type or as a stack, but not in both ways.
%
% \begin{function}[noTF, added = 2012-05-14, updated = 2019-02-16]
%   {\clist_get:NN, \clist_get:cN}
%   \begin{syntax}
%     \cs{clist_get:NN} \meta{comma list} \meta{token list variable}
%   \end{syntax}
%   Stores the left-most item from a \meta{comma list} in the
%   \meta{token list variable} without removing it from the
%   \meta{comma list}. The \meta{token list variable} is assigned locally.
%   In the non-branching version, if the \meta{comma list} is empty the
%   \meta{token list variable} is set to the marker value \cs{q_no_value}.
% \end{function}
%
% \begin{function}[updated = 2011-09-06]{\clist_pop:NN, \clist_pop:cN}
%   \begin{syntax}
%     \cs{clist_pop:NN} \meta{comma list} \meta{token list variable}
%   \end{syntax}
%   Pops the left-most item from a \meta{comma list} into the
%   \meta{token list variable}, \emph{i.e.}~removes the item from the
%   comma list and stores it in the \meta{token list variable}.
%   Both of the variables are assigned locally.
% \end{function}
%
% \begin{function}{\clist_gpop:NN, \clist_gpop:cN}
%   \begin{syntax}
%     \cs{clist_gpop:NN} \meta{comma list} \meta{token list variable}
%   \end{syntax}
%   Pops the left-most item from a \meta{comma list} into the
%   \meta{token list variable}, \emph{i.e.}~removes the item from the
%   comma list and stores it in the \meta{token list variable}.
%   The \meta{comma list} is modified globally, while the assignment of
%   the \meta{token list variable} is local.
% \end{function}
%
% \begin{function}[TF, added = 2012-05-14]{\clist_pop:NN, \clist_pop:cN}
%   \begin{syntax}
%     \cs{clist_pop:NNTF} \meta{comma list} \meta{token list variable} \Arg{true code} \Arg{false code}
%   \end{syntax}
%   If the \meta{comma list} is empty, leaves the \meta{false code} in the
%   input stream.  The value of the \meta{token list variable} is
%   not defined in this case and should not be relied upon.  If the
%   \meta{comma list} is non-empty, pops the top item from the
%   \meta{comma list} in the \meta{token list variable}, \emph{i.e.}~removes
%   the item from the \meta{comma list}. Both the \meta{comma list} and the
%   \meta{token list variable} are assigned locally.
% \end{function}
%
% \begin{function}[TF, added = 2012-05-14]{\clist_gpop:NN, \clist_gpop:cN}
%   \begin{syntax}
%     \cs{clist_gpop:NNTF} \meta{comma list} \meta{token list variable} \Arg{true code} \Arg{false code}
%   \end{syntax}
%   If the \meta{comma list} is empty, leaves the \meta{false code} in the
%   input stream.  The value of the \meta{token list variable} is
%   not defined in this case and should not be relied upon.  If the
%   \meta{comma list} is non-empty, pops the top item from the
%   \meta{comma list} in the \meta{token list variable}, \emph{i.e.}~removes
%   the item from the \meta{comma list}. The \meta{comma list} is modified
%   globally, while the \meta{token list variable} is assigned locally.
% \end{function}
%
% \begin{function}
%   {
%     \clist_push:Nn,  \clist_push:NV,  \clist_push:No,  \clist_push:Nx,
%     \clist_push:cn,  \clist_push:cV,  \clist_push:co,  \clist_push:cx,
%     \clist_gpush:Nn, \clist_gpush:NV, \clist_gpush:No, \clist_gpush:Nx,
%     \clist_gpush:cn, \clist_gpush:cV, \clist_gpush:co, \clist_gpush:cx
%   }
%   \begin{syntax}
%     \cs{clist_push:Nn} \meta{comma list} \Arg{items}
%   \end{syntax}
%   Adds the \Arg{items} to the top of the \meta{comma list}.
%   Spaces are removed from both sides of each item as for any
%   \texttt{n}-type comma list.
% \end{function}
%
% \section{Using a single item}
%
% \begin{function}[added = 2014-07-17, EXP]
%   {\clist_item:Nn, \clist_item:cn, \clist_item:nn}
%   \begin{syntax}
%     \cs{clist_item:Nn} \meta{comma list} \Arg{integer expression}
%   \end{syntax}
%   Indexing items in the \meta{comma list} from~$1$ at the top (left), this
%   function evaluates the \meta{integer expression} and leaves the
%   appropriate item from the comma list in the input stream. If the
%   \meta{integer expression} is negative, indexing occurs from the
%   bottom (right) of the comma list. When the \meta{integer expression}
%   is larger than the number of items in the \meta{comma list} (as
%   calculated by \cs{clist_count:N}) then the function expands to
%   nothing.
%   \begin{texnote}
%     The result is returned within the \tn{unexpanded}
%     primitive (\cs{exp_not:n}), which means that the \meta{item}
%     does not expand further when appearing in an \texttt{x}-type
%     argument expansion.
%   \end{texnote}
% \end{function}
%
% \begin{function}[EXP, added = 2016-12-06]
%   {\clist_rand_item:N, \clist_rand_item:n, \clist_rand_item:c}
%   \begin{syntax}
%     \cs{clist_rand_item:N} \meta{clist~var}
%     \cs{clist_rand_item:n} \Arg{comma list}
%   \end{syntax}
%   Selects a pseudo-random item of the \meta{comma list}.  If the
%   \meta{comma list} has no item, the result is empty.
%   \begin{texnote}
%     The result is returned within the \tn{unexpanded}
%     primitive (\cs{exp_not:n}), which means that the \meta{item}
%     does not expand further when appearing in an \texttt{x}-type
%     argument expansion.
%   \end{texnote}
% \end{function}
%
% \section{Viewing comma lists}
%
% \begin{function}[updated = 2015-08-03]{\clist_show:N, \clist_show:c}
%   \begin{syntax}
%     \cs{clist_show:N} \meta{comma list}
%   \end{syntax}
%   Displays the entries in the \meta{comma list} in the terminal.
% \end{function}
%
% \begin{function}[updated = 2013-08-03]{\clist_show:n}
%   \begin{syntax}
%     \cs{clist_show:n} \Arg{tokens}
%   \end{syntax}
%   Displays the entries in the comma list in the terminal.
% \end{function}
%
% \begin{function}[added = 2014-08-22, updated = 2015-08-03]{\clist_log:N, \clist_log:c}
%   \begin{syntax}
%     \cs{clist_log:N} \meta{comma list}
%   \end{syntax}
%   Writes the entries in the \meta{comma list} in the log file.  See
%   also \cs{clist_show:N} which displays the result in the terminal.
% \end{function}
%
% \begin{function}[added = 2014-08-22]{\clist_log:n}
%   \begin{syntax}
%     \cs{clist_log:n} \Arg{tokens}
%   \end{syntax}
%   Writes the entries in the comma list in the log file.  See also
%   \cs{clist_show:n} which displays the result in the terminal.
% \end{function}
%
% \section{Constant and scratch comma lists}
%
% \begin{variable}[added = 2012-07-02]{\c_empty_clist}
%   Constant that is always empty.
% \end{variable}
%
% \begin{variable}[added = 2011-09-06]{\l_tmpa_clist, \l_tmpb_clist}
%   Scratch comma lists for local assignment. These are never used by
%   the kernel code, and so are safe for use with any \LaTeX3-defined
%   function. However, they may be overwritten by other non-kernel
%   code and so should only be used for short-term storage.
% \end{variable}
%
% \begin{variable}[added = 2011-09-06]{\g_tmpa_clist, \g_tmpb_clist}
%   Scratch comma lists for global assignment. These are never used by
%   the kernel code, and so are safe for use with any \LaTeX3-defined
%   function. However, they may be overwritten by other non-kernel
%   code and so should only be used for short-term storage.
% \end{variable}
%
% \end{documentation}
%
% \begin{implementation}
%
% \section{\pkg{l3clist} implementation}
%
% \TestFiles{m3clist002}
%
%    \begin{macrocode}
%<*initex|package>
%    \end{macrocode}
%
%    \begin{macrocode}
%<@@=clist>
%    \end{macrocode}
%
% \begin{variable}{\c_empty_clist}
%   An empty comma list is simply an empty token list.
%    \begin{macrocode}
\cs_new_eq:NN \c_empty_clist \c_empty_tl
%    \end{macrocode}
% \end{variable}
%
% \begin{variable}{\l_@@_internal_clist}
%   Scratch space for various internal uses. This comma list variable
%   cannot be declared as such because it comes before \cs{clist_new:N}
%    \begin{macrocode}
\tl_new:N \l_@@_internal_clist
%    \end{macrocode}
% \end{variable}
%
% \begin{macro}{\@@_tmp:w}
%   A temporary function for various purposes.
%    \begin{macrocode}
\cs_new_protected:Npn \@@_tmp:w { }
%    \end{macrocode}
% \end{macro}
%
% \subsection{Removing spaces around items}
%
% \begin{macro}{\@@_trim_next:w}
%   Called as \cs{exp:w} \cs{@@_trim_next:w} \cs{prg_do_nothing:}
%   \meta{comma list} \ldots{} it expands to \Arg{trimmed item} where
%   the \meta{trimmed item} is the first non-empty result from removing
%   spaces from both ends of comma-delimited items in the \meta{comma
%   list}.  The \cs{prg_do_nothing:} marker avoids losing braces.  The
%   test for blank items is a somewhat optimized \cs{tl_if_empty:oTF}
%   construction; if blank, another item is sought, otherwise trim
%   spaces.
%    \begin{macrocode}
\cs_new:Npn \@@_trim_next:w #1 ,
  {
    \tl_if_empty:oTF { \use_none:nn #1 ? }
      { \@@_trim_next:w \prg_do_nothing: }
      { \tl_trim_spaces_apply:oN {#1} \exp_end: }
  }
%    \end{macrocode}
% \end{macro}
%
% \begin{macro}[rEXP]{\@@_sanitize:n}
% \begin{macro}{\@@_sanitize:Nn}
%   The auxiliary \cs{@@_sanitize:Nn} receives a delimiter
%   (\cs{c_empty_tl} the first time, afterwards a comma) and that item
%   as arguments.  Unless we are done with the loop it calls
%   \cs{@@_wrap_item:w} to unbrace the item (using a comma delimiter is
%   safe since |#2| came from removing spaces from an argument delimited
%   by a comma) and possibly re-brace it if needed.
%    \begin{macrocode}
\cs_new:Npn \@@_sanitize:n #1
  {
    \exp_after:wN \@@_sanitize:Nn \exp_after:wN \c_empty_tl
    \exp:w \@@_trim_next:w \prg_do_nothing:
    #1 , \q_recursion_tail , \q_recursion_stop
  }
\cs_new:Npn \@@_sanitize:Nn #1#2
  {
    \quark_if_recursion_tail_stop:n {#2}
    #1 \@@_wrap_item:w #2 ,
    \exp_after:wN \@@_sanitize:Nn \exp_after:wN ,
    \exp:w \@@_trim_next:w \prg_do_nothing:
  }
%    \end{macrocode}
% \end{macro}
% \end{macro}
%
% \begin{macro}[TF]{\@@_if_wrap:n}
% \begin{macro}{\@@_if_wrap:w}
%   True if the argument must be wrapped to avoid getting altered by some
%   clist operations.  That is the case whenever the argument
%   \begin{itemize}
%   \item starts or end with a space or contains a comma,
%   \item is empty, or
%   \item consists of a single braced group.
%   \end{itemize}
%   All \pkg{l3clist} functions go through the same test when they need
%   to determine whether to brace an item, so it is not a problem that
%   this test has false positives such as \enquote{\cs{q_mark}
%   \texttt{?}}.  If the argument starts or end with a space or contains
%   a comma then one of the three arguments of \cs{@@_if_wrap:w} will
%   have its end delimiter (partly) in one of the three copies of |#1|
%   in \cs{@@_if_wrap:nTF}; this has a knock-on effect meaning that the
%   result of the expansion is not empty; in that case, wrap.
%   Otherwise, the argument is safe unless it starts with a brace group
%   (or is empty) and it is empty or consists of a single
%   \texttt{n}-type argument.
%    \begin{macrocode}
\prg_new_conditional:Npnn \@@_if_wrap:n #1 { TF }
  {
    \tl_if_empty:oTF
      {
        \@@_if_wrap:w
          \q_mark ? #1 ~ \q_mark ? ~ #1 \q_mark , ~ \q_mark #1 ,
      }
      {
        \tl_if_head_is_group:nTF { #1 { } }
          {
            \tl_if_empty:nTF {#1}
              { \prg_return_true: }
              {
                \tl_if_empty:oTF { \use_none:n #1}
                  { \prg_return_true: }
                  { \prg_return_false: }
              }
          }
          { \prg_return_false: }
      }
      { \prg_return_true: }
  }
\cs_new:Npn \@@_if_wrap:w #1 \q_mark ? ~ #2 ~ \q_mark #3 , { }
%    \end{macrocode}
% \end{macro}
% \end{macro}
%
% \begin{macro}{\@@_wrap_item:w}
%   Safe items are put in \cs{exp_not:n}, otherwise we put an extra set
%   of braces.
%    \begin{macrocode}
\cs_new:Npn \@@_wrap_item:w #1 ,
  { \@@_if_wrap:nTF {#1} { \exp_not:n { {#1} } } { \exp_not:n {#1} } }
%    \end{macrocode}
% \end{macro}
%
% \subsection{Allocation and initialisation}
%
% \begin{macro}{\clist_new:N, \clist_new:c}
% \UnitTested
%   Internally, comma lists are just token lists.
%    \begin{macrocode}
\cs_new_eq:NN \clist_new:N \tl_new:N
\cs_new_eq:NN \clist_new:c \tl_new:c
%    \end{macrocode}
% \end{macro}
%
% \begin{macro}
%   {
%     \clist_const:Nn, \clist_const:cn,
%     \clist_const:Nx, \clist_const:cx
%   }
%   Creating and initializing a constant comma list is done by
%   sanitizing all items (stripping spaces and braces).
%    \begin{macrocode}
\cs_new_protected:Npn \clist_const:Nn #1#2
  { \tl_const:Nx #1 { \@@_sanitize:n {#2} } }
\cs_generate_variant:Nn \clist_const:Nn { c , Nx , cx }
%    \end{macrocode}
% \end{macro}
%
% \begin{macro}{\clist_clear:N, \clist_clear:c}
% \UnitTested
% \begin{macro}{\clist_gclear:N, \clist_gclear:c}
% \UnitTested
%   Clearing comma lists is just the same as clearing token lists.
%    \begin{macrocode}
\cs_new_eq:NN \clist_clear:N  \tl_clear:N
\cs_new_eq:NN \clist_clear:c  \tl_clear:c
\cs_new_eq:NN \clist_gclear:N \tl_gclear:N
\cs_new_eq:NN \clist_gclear:c \tl_gclear:c
%    \end{macrocode}
% \end{macro}
% \end{macro}
%
% \begin{macro}{\clist_clear_new:N, \clist_clear_new:c}
% \UnitTested
% \begin{macro}{\clist_gclear_new:N, \clist_gclear_new:c}
% \UnitTested
%   Once again a copy from the token list functions.
%    \begin{macrocode}
\cs_new_eq:NN \clist_clear_new:N  \tl_clear_new:N
\cs_new_eq:NN \clist_clear_new:c  \tl_clear_new:c
\cs_new_eq:NN \clist_gclear_new:N \tl_gclear_new:N
\cs_new_eq:NN \clist_gclear_new:c \tl_gclear_new:c
%    \end{macrocode}
% \end{macro}
% \end{macro}
%
% \begin{macro}
%   {\clist_set_eq:NN, \clist_set_eq:cN, \clist_set_eq:Nc, \clist_set_eq:cc}
% \UnitTested
% \begin{macro}
%   {
%     \clist_gset_eq:NN, \clist_gset_eq:cN,
%     \clist_gset_eq:Nc, \clist_gset_eq:cc
%   }
% \UnitTested
%   Once again, these are simple copies from the token list functions.
%    \begin{macrocode}
\cs_new_eq:NN \clist_set_eq:NN  \tl_set_eq:NN
\cs_new_eq:NN \clist_set_eq:Nc  \tl_set_eq:Nc
\cs_new_eq:NN \clist_set_eq:cN  \tl_set_eq:cN
\cs_new_eq:NN \clist_set_eq:cc  \tl_set_eq:cc
\cs_new_eq:NN \clist_gset_eq:NN \tl_gset_eq:NN
\cs_new_eq:NN \clist_gset_eq:Nc \tl_gset_eq:Nc
\cs_new_eq:NN \clist_gset_eq:cN \tl_gset_eq:cN
\cs_new_eq:NN \clist_gset_eq:cc \tl_gset_eq:cc
%    \end{macrocode}
% \end{macro}
% \end{macro}
%
% \begin{macro}
%   {
%     \clist_set_from_seq:NN, \clist_set_from_seq:cN,
%     \clist_set_from_seq:Nc, \clist_set_from_seq:cc
%   }
% \UnitTested
% \begin{macro}
%   {
%     \clist_gset_from_seq:NN, \clist_gset_from_seq:cN,
%     \clist_gset_from_seq:Nc, \clist_gset_from_seq:cc
%   }
% \UnitTested
% \begin{macro}{\@@_set_from_seq:NNNN, \@@_set_from_seq:n}
%   Setting a comma list from a comma-separated list is done using a
%   simple mapping.  Safe items are put in \cs{exp_not:n}, otherwise we
%   put an extra set of braces.  The first comma must be removed, except
%   in the case of an empty comma-list.
%    \begin{macrocode}
\cs_new_protected:Npn \clist_set_from_seq:NN
  { \@@_set_from_seq:NNNN \clist_clear:N  \tl_set:Nx  }
\cs_new_protected:Npn \clist_gset_from_seq:NN
  { \@@_set_from_seq:NNNN \clist_gclear:N \tl_gset:Nx }
\cs_new_protected:Npn \@@_set_from_seq:NNNN #1#2#3#4
  {
    \seq_if_empty:NTF #4
      { #1 #3 }
      {
        #2 #3
          {
            \exp_after:wN \use_none:n \exp:w \exp_end_continue_f:w
            \seq_map_function:NN #4 \@@_set_from_seq:n
          }
      }
  }
\cs_new:Npn \@@_set_from_seq:n #1
  {
    ,
    \@@_if_wrap:nTF {#1}
      { \exp_not:n { {#1} } }
      { \exp_not:n {#1} }
  }
\cs_generate_variant:Nn \clist_set_from_seq:NN  {     Nc }
\cs_generate_variant:Nn \clist_set_from_seq:NN  { c , cc }
\cs_generate_variant:Nn \clist_gset_from_seq:NN {     Nc }
\cs_generate_variant:Nn \clist_gset_from_seq:NN { c , cc }
%    \end{macrocode}
% \end{macro}
% \end{macro}
% \end{macro}
%
% \begin{macro}{\clist_concat:NNN, \clist_concat:ccc}
% \UnitTested
% \begin{macro}{\clist_gconcat:NNN, \clist_gconcat:ccc}
% \UnitTested
% \begin{macro}{\@@_concat:NNNN}
%   Concatenating comma lists is not quite as easy as it seems, as
%   there needs to be the correct addition of a comma to the output. So
%   a little work to do.
%    \begin{macrocode}
\cs_new_protected:Npn \clist_concat:NNN
  { \@@_concat:NNNN \tl_set:Nx }
\cs_new_protected:Npn \clist_gconcat:NNN
  { \@@_concat:NNNN \tl_gset:Nx }
\cs_new_protected:Npn \@@_concat:NNNN #1#2#3#4
  {
    #1 #2
      {
        \exp_not:o #3
        \clist_if_empty:NF #3 { \clist_if_empty:NF #4 { , } }
        \exp_not:o #4
      }
  }
\cs_generate_variant:Nn \clist_concat:NNN  { ccc }
\cs_generate_variant:Nn \clist_gconcat:NNN { ccc }
%    \end{macrocode}
% \end{macro}
% \end{macro}
% \end{macro}
%
% \begin{macro}[pTF]{\clist_if_exist:N, \clist_if_exist:c}
%   Copies of the \texttt{cs} functions defined in \pkg{l3basics}.
%    \begin{macrocode}
\prg_new_eq_conditional:NNn \clist_if_exist:N \cs_if_exist:N
  { TF , T , F , p }
\prg_new_eq_conditional:NNn \clist_if_exist:c \cs_if_exist:c
  { TF , T , F , p }
%    \end{macrocode}
% \end{macro}
%
% \subsection{Adding data to comma lists}
%
% \begin{macro}
%   {
%     \clist_set:Nn, \clist_set:NV,
%     \clist_set:No, \clist_set:Nx,
%     \clist_set:cn, \clist_set:cV,
%     \clist_set:co, \clist_set:cx
%   }
% \begin{macro}
%   {
%     \clist_gset:Nn, \clist_gset:NV,
%     \clist_gset:No, \clist_gset:Nx,
%     \clist_gset:cn, \clist_gset:cV,
%     \clist_gset:co, \clist_gset:cx
%   }
%    \begin{macrocode}
\cs_new_protected:Npn \clist_set:Nn #1#2
  { \tl_set:Nx #1 { \@@_sanitize:n {#2} } }
\cs_new_protected:Npn \clist_gset:Nn #1#2
  { \tl_gset:Nx #1 { \@@_sanitize:n {#2} } }
\cs_generate_variant:Nn \clist_set:Nn  { NV , No , Nx , c , cV , co , cx }
\cs_generate_variant:Nn \clist_gset:Nn { NV , No , Nx , c , cV , co , cx }
%    \end{macrocode}
% \end{macro}
% \end{macro}
%
% \begin{macro}
%   {
%     \clist_put_left:Nn, \clist_put_left:NV,
%     \clist_put_left:No, \clist_put_left:Nx,
%     \clist_put_left:cn, \clist_put_left:cV,
%     \clist_put_left:co, \clist_put_left:cx
%   }
% \UnitTested
% \begin{macro}
%   {
%     \clist_gput_left:Nn, \clist_gput_left:NV,
%     \clist_gput_left:No, \clist_gput_left:Nx,
%     \clist_gput_left:cn, \clist_gput_left:cV,
%     \clist_gput_left:co, \clist_gput_left:cx
%   }
% \UnitTested
% \begin{macro}{\@@_put_left:NNNn}
%   Everything is based on concatenation after storing in
%   \cs{l_@@_internal_clist}.  This avoids having to worry here about
%   space-trimming and so on.
%    \begin{macrocode}
\cs_new_protected:Npn \clist_put_left:Nn
  { \@@_put_left:NNNn \clist_concat:NNN \clist_set:Nn }
\cs_new_protected:Npn \clist_gput_left:Nn
  { \@@_put_left:NNNn \clist_gconcat:NNN \clist_set:Nn }
\cs_new_protected:Npn \@@_put_left:NNNn #1#2#3#4
  {
    #2 \l_@@_internal_clist {#4}
    #1 #3 \l_@@_internal_clist #3
  }
\cs_generate_variant:Nn \clist_put_left:Nn  {     NV , No , Nx }
\cs_generate_variant:Nn \clist_put_left:Nn  { c , cV , co , cx }
\cs_generate_variant:Nn \clist_gput_left:Nn {     NV , No , Nx }
\cs_generate_variant:Nn \clist_gput_left:Nn { c , cV , co , cx }
%    \end{macrocode}
% \end{macro}
% \end{macro}
% \end{macro}
%
% \begin{macro}
%   {
%     \clist_put_right:Nn, \clist_put_right:NV,
%     \clist_put_right:No, \clist_put_right:Nx,
%     \clist_put_right:cn, \clist_put_right:cV,
%     \clist_put_right:co, \clist_put_right:cx
%   }
% \UnitTested
% \begin{macro}
%   {
%     \clist_gput_right:Nn, \clist_gput_right:NV,
%     \clist_gput_right:No, \clist_gput_right:Nx,
%     \clist_gput_right:cn, \clist_gput_right:cV,
%     \clist_gput_right:co, \clist_gput_right:cx
%   }
% \UnitTested
% \begin{macro}{\@@_put_right:NNNn}
%    \begin{macrocode}
\cs_new_protected:Npn \clist_put_right:Nn
  { \@@_put_right:NNNn \clist_concat:NNN \clist_set:Nn }
\cs_new_protected:Npn \clist_gput_right:Nn
  { \@@_put_right:NNNn \clist_gconcat:NNN \clist_set:Nn }
\cs_new_protected:Npn \@@_put_right:NNNn #1#2#3#4
  {
    #2 \l_@@_internal_clist {#4}
    #1 #3 #3 \l_@@_internal_clist
  }
\cs_generate_variant:Nn \clist_put_right:Nn  {     NV , No , Nx }
\cs_generate_variant:Nn \clist_put_right:Nn  { c , cV , co , cx }
\cs_generate_variant:Nn \clist_gput_right:Nn {     NV , No , Nx }
\cs_generate_variant:Nn \clist_gput_right:Nn { c , cV , co , cx }
%    \end{macrocode}
% \end{macro}
% \end{macro}
% \end{macro}
%
% \subsection{Comma lists as stacks}
%
% \begin{macro}{\clist_get:NN, \clist_get:cN}
% \UnitTested
% \begin{macro}{\@@_get:wN}
%   Getting an item from the left of a comma list is pretty easy: just
%   trim off the first item using the comma.  No need to trim spaces as
%   comma-list \emph{variables} are assumed to have \enquote{cleaned-up}
%   items.  (Note that grabbing a comma-delimited item removes an outer
%   pair of braces if present, exactly as needed to uncover the
%   underlying item.)
%    \begin{macrocode}
\cs_new_protected:Npn \clist_get:NN #1#2
  {
    \if_meaning:w #1 \c_empty_clist
      \tl_set:Nn #2 { \q_no_value }
    \else:
      \exp_after:wN \@@_get:wN #1 , \q_stop #2
    \fi:
  }
\cs_new_protected:Npn \@@_get:wN #1 , #2 \q_stop #3
  { \tl_set:Nn #3 {#1} }
\cs_generate_variant:Nn \clist_get:NN { c }
%    \end{macrocode}
% \end{macro}
% \end{macro}
%
% \begin{macro}{\clist_pop:NN, \clist_pop:cN}
% \UnitTested
% \begin{macro}{\clist_gpop:NN, \clist_gpop:cN}
% \UnitTested
% \begin{macro}
%   {\@@_pop:NNN, \@@_pop:wwNNN, \@@_pop:wN}
%   An empty clist leads to \cs{q_no_value}, otherwise grab until the
%   first comma and assign to the variable.  The second argument of
%   \cs{@@_pop:wwNNN} is a comma list ending in a comma and
%   \cs{q_mark}, unless the original clist contained exactly one item:
%   then the argument is just \cs{q_mark}.  The next auxiliary picks
%   either \cs{exp_not:n} or \cs{use_none:n} as |#2|, ensuring that the
%   result can safely be an empty comma list.
%    \begin{macrocode}
\cs_new_protected:Npn \clist_pop:NN
  { \@@_pop:NNN \tl_set:Nx }
\cs_new_protected:Npn \clist_gpop:NN
  { \@@_pop:NNN \tl_gset:Nx }
\cs_new_protected:Npn \@@_pop:NNN #1#2#3
  {
    \if_meaning:w #2 \c_empty_clist
      \tl_set:Nn #3 { \q_no_value }
    \else:
      \exp_after:wN \@@_pop:wwNNN #2 , \q_mark \q_stop #1#2#3
    \fi:
  }
\cs_new_protected:Npn \@@_pop:wwNNN #1 , #2 \q_stop #3#4#5
  {
    \tl_set:Nn #5 {#1}
    #3 #4
      {
        \@@_pop:wN \prg_do_nothing:
          #2 \exp_not:o
          , \q_mark \use_none:n
        \q_stop
      }
  }
\cs_new:Npn \@@_pop:wN #1 , \q_mark #2 #3 \q_stop { #2 {#1} }
\cs_generate_variant:Nn \clist_pop:NN  { c }
\cs_generate_variant:Nn \clist_gpop:NN { c }
%    \end{macrocode}
% \end{macro}
% \end{macro}
% \end{macro}
%
% \begin{macro}[TF]{\clist_get:NN, \clist_get:cN}
% \begin{macro}[TF]{\clist_pop:NN, \clist_pop:cN}
% \begin{macro}[TF]{\clist_gpop:NN, \clist_gpop:cN}
% \begin{macro}{\@@_pop_TF:NNN}
%   The same, as branching code: very similar to the above.
%    \begin{macrocode}
\prg_new_protected_conditional:Npnn \clist_get:NN #1#2 { T , F , TF }
  {
    \if_meaning:w #1 \c_empty_clist
      \prg_return_false:
    \else:
      \exp_after:wN \@@_get:wN #1 , \q_stop #2
      \prg_return_true:
    \fi:
  }
\prg_generate_conditional_variant:Nnn \clist_get:NN { c } { T , F , TF }
\prg_new_protected_conditional:Npnn \clist_pop:NN #1#2 { T , F , TF }
  { \@@_pop_TF:NNN \tl_set:Nx #1 #2 }
\prg_new_protected_conditional:Npnn \clist_gpop:NN #1#2 { T , F , TF }
  { \@@_pop_TF:NNN \tl_gset:Nx #1 #2 }
\cs_new_protected:Npn \@@_pop_TF:NNN #1#2#3
  {
    \if_meaning:w #2 \c_empty_clist
      \prg_return_false:
    \else:
      \exp_after:wN \@@_pop:wwNNN #2 , \q_mark \q_stop #1#2#3
      \prg_return_true:
    \fi:
  }
\prg_generate_conditional_variant:Nnn \clist_pop:NN { c } { T , F , TF }
\prg_generate_conditional_variant:Nnn \clist_gpop:NN { c } { T , F , TF }
%    \end{macrocode}
% \end{macro}
% \end{macro}
% \end{macro}
% \end{macro}
%
% \begin{macro}{
%   \clist_push:Nn, \clist_push:NV, \clist_push:No, \clist_push:Nx,
%   \clist_push:cn, \clist_push:cV, \clist_push:co, \clist_push:cx
% }
% \UnitTested
% \begin{macro}{
%   \clist_gpush:Nn, \clist_gpush:NV, \clist_gpush:No, \clist_gpush:Nx,
%   \clist_gpush:cn, \clist_gpush:cV, \clist_gpush:co, \clist_gpush:cx
% }
% \UnitTested
%   Pushing to a comma list is the same as adding on the left.
%    \begin{macrocode}
\cs_new_eq:NN \clist_push:Nn  \clist_put_left:Nn
\cs_new_eq:NN \clist_push:NV  \clist_put_left:NV
\cs_new_eq:NN \clist_push:No  \clist_put_left:No
\cs_new_eq:NN \clist_push:Nx  \clist_put_left:Nx
\cs_new_eq:NN \clist_push:cn  \clist_put_left:cn
\cs_new_eq:NN \clist_push:cV  \clist_put_left:cV
\cs_new_eq:NN \clist_push:co  \clist_put_left:co
\cs_new_eq:NN \clist_push:cx  \clist_put_left:cx
\cs_new_eq:NN \clist_gpush:Nn \clist_gput_left:Nn
\cs_new_eq:NN \clist_gpush:NV \clist_gput_left:NV
\cs_new_eq:NN \clist_gpush:No \clist_gput_left:No
\cs_new_eq:NN \clist_gpush:Nx \clist_gput_left:Nx
\cs_new_eq:NN \clist_gpush:cn \clist_gput_left:cn
\cs_new_eq:NN \clist_gpush:cV \clist_gput_left:cV
\cs_new_eq:NN \clist_gpush:co \clist_gput_left:co
\cs_new_eq:NN \clist_gpush:cx \clist_gput_left:cx
%    \end{macrocode}
% \end{macro}
% \end{macro}
%
% \subsection{Modifying comma lists}
%
% \begin{variable}{\l_@@_internal_remove_clist, \l_@@_internal_remove_seq}
%   An internal comma list and a sequence for the removal routines.
%    \begin{macrocode}
\clist_new:N \l_@@_internal_remove_clist
\seq_new:N \l_@@_internal_remove_seq
%    \end{macrocode}
% \end{variable}
%
% \begin{macro}{\clist_remove_duplicates:N, \clist_remove_duplicates:c}
% \UnitTested
% \begin{macro}{\clist_gremove_duplicates:N, \clist_gremove_duplicates:c}
% \UnitTested
% \begin{macro}{\@@_remove_duplicates:NN}
%   Removing duplicates means making a new list then copying it.
%    \begin{macrocode}
\cs_new_protected:Npn \clist_remove_duplicates:N
  { \@@_remove_duplicates:NN \clist_set_eq:NN }
\cs_new_protected:Npn \clist_gremove_duplicates:N
  { \@@_remove_duplicates:NN \clist_gset_eq:NN }
\cs_new_protected:Npn \@@_remove_duplicates:NN #1#2
  {
    \clist_clear:N \l_@@_internal_remove_clist
    \clist_map_inline:Nn #2
      {
        \clist_if_in:NnF \l_@@_internal_remove_clist {##1}
          { \clist_put_right:Nn \l_@@_internal_remove_clist {##1} }
      }
    #1 #2 \l_@@_internal_remove_clist
  }
\cs_generate_variant:Nn \clist_remove_duplicates:N  { c }
\cs_generate_variant:Nn \clist_gremove_duplicates:N { c }
%    \end{macrocode}
% \end{macro}
% \end{macro}
% \end{macro}
%
% \begin{macro}{\clist_remove_all:Nn, \clist_remove_all:cn}
% \UnitTested
% \begin{macro}{\clist_gremove_all:Nn, \clist_gremove_all:cn}
% \UnitTested
% \begin{macro}{\@@_remove_all:NNNn}
% \begin{macro}{\@@_remove_all:w}
% \begin{macro}{\@@_remove_all:}
%   The method used here for safe items is very similar to
%   \cs{tl_replace_all:Nnn}.  However, if the item contains commas or
%   leading/trailing spaces, or is empty, or consists of a single brace
%   group, we know that it can only appear within braces so the code
%   would fail; instead just convert to a sequence and do the removal
%   with \pkg{l3seq} code (it involves somewhat elaborate code to do
%   most of the work expandably but the final token list comparisons
%   non-expandably).
%
%   For \enquote{safe} items, build a function delimited by the
%   \meta{item} that should be removed,
%   surrounded with commas, and call that function followed by
%   the expanded comma list, and another copy of the \meta{item}.
%   The loop is controlled by the argument grabbed by
%   \cs{@@_remove_all:w}: when the item was found,
%   the \cs{q_mark} delimiter used is the one inserted by
%   \cs{@@_tmp:w}, and \cs{use_none_delimit_by_q_stop:w}
%   is deleted. At the end, the final \meta{item} is
%   grabbed, and the argument of \cs{@@_tmp:w} contains
%   \cs{q_mark}: in that case, \cs{@@_remove_all:w}
%   removes the second \cs{q_mark} (inserted by \cs{@@_tmp:w}),
%   and lets \cs{use_none_delimit_by_q_stop:w} act.
%
%   No brace is lost because items are always grabbed with a leading comma.
%   The result of the first assignment has an extra leading comma,
%   which we remove in a second assignment.
%   Two exceptions: if the clist lost all of its elements, the result
%   is empty, and we shouldn't remove anything; if the clist started up
%   empty, the first step happens to turn it into a single comma, and
%   the second step removes it.
%    \begin{macrocode}
\cs_new_protected:Npn \clist_remove_all:Nn
  { \@@_remove_all:NNNn \clist_set_from_seq:NN \tl_set:Nx }
\cs_new_protected:Npn \clist_gremove_all:Nn
  { \@@_remove_all:NNNn \clist_gset_from_seq:NN \tl_gset:Nx }
\cs_new_protected:Npn \@@_remove_all:NNNn #1#2#3#4
  {
    \@@_if_wrap:nTF {#4}
      {
        \seq_set_from_clist:NN \l_@@_internal_remove_seq #3
        \seq_remove_all:Nn \l_@@_internal_remove_seq {#4}
        #1 #3 \l_@@_internal_remove_seq
      }
      {
        \cs_set:Npn \@@_tmp:w ##1 , #4 ,
          {
            ##1
            , \q_mark , \use_none_delimit_by_q_stop:w ,
            \@@_remove_all:
          }
        #2 #3
          {
            \exp_after:wN \@@_remove_all:
            #3 , \q_mark , #4 , \q_stop
          }
        \clist_if_empty:NF #3
          {
            #2 #3
              {
                \exp_args:No \exp_not:o
                  { \exp_after:wN \use_none:n #3 }
              }
          }
      }
  }
\cs_new:Npn \@@_remove_all:
  { \exp_after:wN \@@_remove_all:w \@@_tmp:w , }
\cs_new:Npn \@@_remove_all:w #1 , \q_mark , #2 , { \exp_not:n {#1} }
\cs_generate_variant:Nn \clist_remove_all:Nn  { c }
\cs_generate_variant:Nn \clist_gremove_all:Nn { c }
%    \end{macrocode}
% \end{macro}
% \end{macro}
% \end{macro}
% \end{macro}
% \end{macro}
%
% \begin{macro}
%   {
%     \clist_reverse:N, \clist_reverse:c,
%     \clist_greverse:N, \clist_greverse:c
%   }
%   Use \cs{clist_reverse:n} in an \texttt{x}-expanding assignment.  The
%   extra work that \cs{clist_reverse:n} does to preserve braces and
%   spaces would not be needed for the well-controlled case of
%   \texttt{N}-type comma lists, but the slow-down is not too bad.
%    \begin{macrocode}
\cs_new_protected:Npn \clist_reverse:N #1
  { \tl_set:Nx #1 { \exp_args:No \clist_reverse:n {#1} } }
\cs_new_protected:Npn \clist_greverse:N #1
  { \tl_gset:Nx #1 { \exp_args:No \clist_reverse:n {#1} } }
\cs_generate_variant:Nn \clist_reverse:N { c }
\cs_generate_variant:Nn \clist_greverse:N { c }
%    \end{macrocode}
% \end{macro}
%
% \begin{macro}[EXP]{\clist_reverse:n}
% \begin{macro}[EXP]{\@@_reverse:wwNww, \@@_reverse_end:ww}
%   The reversed token list is built one item at a time, and stored
%   between \cs{q_stop} and \cs{q_mark}, in the form of |?| followed by
%   zero or more instances of \enquote{\meta{item}\texttt{,}}.  We start from a comma
%   list \enquote{\meta{item_1}\texttt{,\ldots,}\meta{item_n}}.  During the loop,
%   the auxiliary \cs{@@_reverse:wwNww} receives \enquote{\texttt{?}\meta{item_i}} as
%   |#1|, \enquote{\meta{item_{i+1}}\texttt{,\ldots,}\meta{item_n}} as |#2|,
%   \cs{@@_reverse:wwNww} as |#3|, what remains until \cs{q_stop} as
%   |#4|, and \enquote{\meta{item_{i-1}}\texttt{,\ldots,}\meta{item_1}\texttt{,}} as |#5|.
%   The auxiliary moves |#1| just before |#5|, with a comma, and calls
%   itself (|#3|).  After the last item is moved, \cs{@@_reverse:wwNww}
%   receives \enquote{\cs{q_mark} \cs{@@_reverse:wwNww} \texttt{!}} as its argument
%   |#1|, thus \cs{@@_reverse_end:ww} as its argument |#3|.  This second
%   auxiliary cleans up until the marker~|!|, removes the trailing comma
%   (introduced when the first item was moved after \cs{q_stop}), and
%   leaves its argument~|#1| within \cs{exp_not:n}.  There is also a
%   need to remove a leading comma, hence \cs{exp_not:o} and
%   \cs{use_none:n}.
%    \begin{macrocode}
\cs_new:Npn \clist_reverse:n #1
  {
    \@@_reverse:wwNww ? #1 ,
      \q_mark \@@_reverse:wwNww ! ,
      \q_mark \@@_reverse_end:ww
      \q_stop ? \q_mark
  }
\cs_new:Npn \@@_reverse:wwNww
    #1 , #2 \q_mark #3 #4 \q_stop ? #5 \q_mark
  { #3 ? #2 \q_mark #3 #4 \q_stop #1 , #5 \q_mark }
\cs_new:Npn \@@_reverse_end:ww #1 ! #2 , \q_mark
  { \exp_not:o { \use_none:n #2 } }
%    \end{macrocode}
% \end{macro}
% \end{macro}
%
% \begin{macro}
%   {\clist_sort:Nn, \clist_sort:cn, \clist_gsort:Nn, \clist_gsort:cn}
%   Implemented in \pkg{l3sort}.
% \end{macro}
%
% \subsection{Comma list conditionals}
%
% \begin{macro}[pTF]{\clist_if_empty:N, \clist_if_empty:c}
% \UnitTested
%   Simple copies from the token list variable material.
%    \begin{macrocode}
\prg_new_eq_conditional:NNn \clist_if_empty:N \tl_if_empty:N
  { p , T , F , TF }
\prg_new_eq_conditional:NNn \clist_if_empty:c \tl_if_empty:c
  { p , T , F , TF }
%    \end{macrocode}
% \end{macro}
%
% \begin{macro}[EXP, pTF]{\clist_if_empty:n}
% \begin{macro}[EXP]{\@@_if_empty_n:w}
% \begin{macro}[EXP]{\@@_if_empty_n:wNw}
%   As usual, we insert a token (here |?|) before grabbing
%   any argument: this avoids losing braces. The argument
%   of \cs{tl_if_empty:oTF} is empty if |#1| is |?| followed
%   by blank spaces (besides, this particular variant of
%   the emptiness test is optimized). If the item of the
%   comma list is blank, grab the next one. As soon as one
%   item is non-blank, exit: the second auxiliary grabs
%   \cs{prg_return_false:} as |#2|, unless every item in
%   the comma list was blank and the loop actually got broken
%   by the trailing |\q_mark \prg_return_false:| item.
%    \begin{macrocode}
\prg_new_conditional:Npnn \clist_if_empty:n #1 { p , T , F , TF }
  {
    \@@_if_empty_n:w ? #1
    , \q_mark \prg_return_false:
    , \q_mark \prg_return_true:
    \q_stop
  }
\cs_new:Npn \@@_if_empty_n:w #1 ,
  {
    \tl_if_empty:oTF { \use_none:nn #1 ? }
      { \@@_if_empty_n:w ? }
      { \@@_if_empty_n:wNw }
  }
\cs_new:Npn \@@_if_empty_n:wNw #1 \q_mark #2#3 \q_stop {#2}
%    \end{macrocode}
% \end{macro}
% \end{macro}
% \end{macro}
%
% \begin{macro}[TF]
%   {
%     \clist_if_in:Nn, \clist_if_in:NV, \clist_if_in:No,
%     \clist_if_in:cn, \clist_if_in:cV, \clist_if_in:co,
%     \clist_if_in:nn, \clist_if_in:nV, \clist_if_in:no
%   }
% \begin{macro}{\@@_if_in_return:nnN}
% \UnitTested
%   For \enquote{safe} items, we simply surround the comma list, and the
%   item, with commas, then use the same code as for \cs{tl_if_in:Nn}.
%   For \enquote{unsafe} items we follow the same route as
%   \cs{seq_if_in:Nn}, mapping through the list a comparison function.
%   If found, return \texttt{true} and remove \cs{prg_return_false:}.
%    \begin{macrocode}
\prg_new_protected_conditional:Npnn \clist_if_in:Nn #1#2 { T  , F , TF }
  {
    \exp_args:No \@@_if_in_return:nnN #1 {#2} #1
  }
\prg_new_protected_conditional:Npnn \clist_if_in:nn #1#2 { T  , F , TF }
  {
    \clist_set:Nn \l_@@_internal_clist {#1}
    \exp_args:No \@@_if_in_return:nnN \l_@@_internal_clist {#2}
      \l_@@_internal_clist
  }
\cs_new_protected:Npn \@@_if_in_return:nnN #1#2#3
  {
    \@@_if_wrap:nTF {#2}
      {
        \cs_set:Npx \@@_tmp:w ##1
          {
            \exp_not:N \tl_if_eq:nnT {##1}
            \exp_not:n
              {
                {#2}
                { \clist_map_break:n { \prg_return_true: \use_none:n } }
              }
          }
        \clist_map_function:NN #3 \@@_tmp:w
        \prg_return_false:
      }
      {
        \cs_set:Npn \@@_tmp:w ##1 ,#2, { }
        \tl_if_empty:oTF
          { \@@_tmp:w ,#1, {} {} ,#2, }
          { \prg_return_false: } { \prg_return_true: }
      }
  }
\prg_generate_conditional_variant:Nnn \clist_if_in:Nn
  { NV , No , c , cV , co } { T , F , TF }
\prg_generate_conditional_variant:Nnn \clist_if_in:nn
  { nV , no } { T , F , TF }
%    \end{macrocode}
% \end{macro}
% \end{macro}
%
% \subsection{Mapping to comma lists}
%
% \begin{macro}{\clist_map_function:NN, \clist_map_function:cN}
% \UnitTested
% \begin{macro}{\@@_map_function:Nw}
%   If the variable is empty, the mapping is skipped (otherwise,
%   that comma-list would be seen as consisting of one empty item).
%   Then loop over the comma-list, grabbing one comma-delimited
%   item at a time. The end is marked by \cs{q_recursion_tail}.
%   The auxiliary function \cs{@@_map_function:Nw} is also used
%   in \cs{clist_map_inline:Nn}.
%    \begin{macrocode}
\cs_new:Npn \clist_map_function:NN #1#2
  {
    \clist_if_empty:NF #1
      {
        \exp_last_unbraced:NNo \@@_map_function:Nw #2 #1
          , \q_recursion_tail ,
        \prg_break_point:Nn \clist_map_break: { }
      }
  }
\cs_new:Npn \@@_map_function:Nw #1#2 ,
  {
    \quark_if_recursion_tail_break:nN {#2} \clist_map_break:
    #1 {#2}
    \@@_map_function:Nw #1
  }
\cs_generate_variant:Nn \clist_map_function:NN { c }
%    \end{macrocode}
% \end{macro}
% \end{macro}
%
% \begin{macro}{\clist_map_function:nN}
% \UnitTested
% \begin{macro}{\@@_map_function_n:Nn}
% \begin{macro}{\@@_map_unbrace:Nw}
%   The \texttt{n}-type mapping function is a bit more awkward,
%   since spaces must be trimmed from each item.
%   Space trimming is again based on \cs{@@_trim_next:w}.
%   The auxiliary \cs{@@_map_function_n:Nn} receives as arguments the
%   function, and the next non-empty item (after space trimming but
%   before brace removal).  One level of braces is removed by
%   \cs{@@_map_unbrace:Nw}.
%    \begin{macrocode}
\cs_new:Npn \clist_map_function:nN #1#2
  {
    \exp_after:wN \@@_map_function_n:Nn \exp_after:wN #2
    \exp:w \@@_trim_next:w \prg_do_nothing: #1 , \q_recursion_tail ,
    \prg_break_point:Nn \clist_map_break: { }
  }
\cs_new:Npn \@@_map_function_n:Nn #1 #2
  {
    \quark_if_recursion_tail_break:nN {#2} \clist_map_break:
    \@@_map_unbrace:Nw #1 #2,
    \exp_after:wN \@@_map_function_n:Nn \exp_after:wN #1
    \exp:w \@@_trim_next:w \prg_do_nothing:
  }
\cs_new:Npn \@@_map_unbrace:Nw #1 #2, { #1 {#2} }
%    \end{macrocode}
% \end{macro}
% \end{macro}
% \end{macro}
%
% \begin{macro}{\clist_map_inline:Nn, \clist_map_inline:cn}
% \UnitTested
% \begin{macro}{\clist_map_inline:nn}
% \UnitTested
%   Inline mapping is done by creating a suitable function
%   \enquote{on the fly}: this is done globally to avoid
%   any issues with \TeX{}'s groups.  We use a different
%   function for each level of nesting.
%
%   Since the mapping is non-expandable,  we can perform
%   the space-trimming  needed by the \texttt{n} version
%   simply  by storing the comma-list in a variable.  We
%   don't need  a different comma-list  for each nesting
%   level: the comma-list is expanded before the mapping
%   starts.
%    \begin{macrocode}
\cs_new_protected:Npn \clist_map_inline:Nn #1#2
  {
    \clist_if_empty:NF #1
      {
        \int_gincr:N \g__kernel_prg_map_int
        \cs_gset_protected:cpn
          { @@_map_ \int_use:N \g__kernel_prg_map_int :w } ##1 {#2}
        \exp_last_unbraced:Nco \@@_map_function:Nw
          { @@_map_ \int_use:N \g__kernel_prg_map_int :w }
          #1 , \q_recursion_tail ,
        \prg_break_point:Nn \clist_map_break:
          { \int_gdecr:N \g__kernel_prg_map_int }
      }
  }
\cs_new_protected:Npn \clist_map_inline:nn #1
  {
    \clist_set:Nn \l_@@_internal_clist {#1}
    \clist_map_inline:Nn \l_@@_internal_clist
  }
\cs_generate_variant:Nn \clist_map_inline:Nn { c }
%    \end{macrocode}
% \end{macro}
% \end{macro}
%
% \begin{macro}{\clist_map_variable:NNn, \clist_map_variable:cNn}
% \UnitTested
% \begin{macro}{\clist_map_variable:nNn}
% \begin{macro}{\@@_map_variable:Nnw}
%   As for other comma-list mappings, filter out the case of
%   an empty list. Same approach as \cs{clist_map_function:Nn},
%   additionally we store each item in the given variable.
%   As for inline mappings, space trimming for the \texttt{n}
%   variant is done by storing the comma list in a variable.
%   The quark test is done before assigning the item to the variable:
%   this avoids storing a quark which the user wouldn't expect.
%   The strange \cs{use:n} avoids unlikely problems when |#2| would
%   contain \cs{q_recursion_stop}.
%    \begin{macrocode}
\cs_new_protected:Npn \clist_map_variable:NNn #1#2#3
  {
    \clist_if_empty:NF #1
      {
        \exp_args:Nno \use:nn
          { \@@_map_variable:Nnw #2 {#3} }
          #1
          , \q_recursion_tail , \q_recursion_stop
        \prg_break_point:Nn \clist_map_break: { }
      }
  }
\cs_new_protected:Npn \clist_map_variable:nNn #1
  {
    \clist_set:Nn \l_@@_internal_clist {#1}
    \clist_map_variable:NNn \l_@@_internal_clist
  }
\cs_new_protected:Npn \@@_map_variable:Nnw #1#2#3,
  {
    \quark_if_recursion_tail_stop:n {#3}
    \tl_set:Nn #1 {#3}
    \use:n {#2}
    \@@_map_variable:Nnw #1 {#2}
  }
\cs_generate_variant:Nn \clist_map_variable:NNn { c }
%    \end{macrocode}
% \end{macro}
% \end{macro}
% \end{macro}
%
% \begin{macro}{\clist_map_break:, \clist_map_break:n}
%   The break statements use the general \cs{prg_map_break:Nn} mechanism.
%    \begin{macrocode}
\cs_new:Npn \clist_map_break:
  { \prg_map_break:Nn \clist_map_break: { } }
\cs_new:Npn \clist_map_break:n
  { \prg_map_break:Nn \clist_map_break: }
%    \end{macrocode}
% \end{macro}
%
% \begin{macro}{\clist_count:N, \clist_count:c, \clist_count:n}
% \begin{macro}{\@@_count:n}
% \begin{macro}{\@@_count:w}
%   Counting the items in a comma list is done using the same approach as for
%   other token count functions: turn each entry into a \texttt{+1} then use
%   integer evaluation to actually do the mathematics.
%   In the case of an \texttt{n}-type comma-list, we could of course use
%   \cs{clist_map_function:nN}, but that is very slow, because it carefully
%   removes spaces. Instead, we loop manually, and skip blank items
%   (but not |{}|, hence the extra spaces).
%    \begin{macrocode}
\cs_new:Npn \clist_count:N #1
  {
    \int_eval:n
      {
        0
        \clist_map_function:NN #1 \@@_count:n
      }
  }
\cs_generate_variant:Nn \clist_count:N { c }
\cs_new:Npx \clist_count:n #1
  {
    \exp_not:N \int_eval:n
      {
        0
        \exp_not:N \@@_count:w \c_space_tl
        #1 \exp_not:n { , \q_recursion_tail , \q_recursion_stop }
      }
  }
\cs_new:Npn \@@_count:n #1 { + 1 }
\cs_new:Npx \@@_count:w #1 ,
  {
    \exp_not:n { \exp_args:Nf \quark_if_recursion_tail_stop:n } {#1}
    \exp_not:N \tl_if_blank:nF {#1} { + 1 }
    \exp_not:N \@@_count:w \c_space_tl
  }
%    \end{macrocode}
% \end{macro}
% \end{macro}
% \end{macro}
%
% \subsection{Using comma lists}
%
% \begin{macro}[EXP]{\clist_use:Nnnn, \clist_use:cnnn}
% \begin{macro}[EXP]
%   {\@@_use:wwn, \@@_use:nwwwwnwn, \@@_use:nwwn}
% \begin{macro}[EXP]{\clist_use:Nn, \clist_use:cn}
%   First check that the variable exists.  Then count the items in the
%   comma list.  If it has none, output nothing.  If it has one item,
%   output that item, brace stripped (note that space-trimming has
%   already been done when the comma list was assigned).  If it has two,
%   place the \meta{separator~between~two} in the middle.
%
%   Otherwise, \cs{@@_use:nwwwwnwn} takes the following arguments; 1:
%   a \meta{separator}, 2, 3, 4: three items from the comma list (or
%   quarks), 5: the rest of the comma list, 6: a \meta{continuation}
%   function (\texttt{use_ii} or \texttt{use_iii} with its
%   \meta{separator} argument), 7: junk, and 8: the temporary result,
%   which is built in a brace group following \cs{q_stop}.  The
%   \meta{separator} and the first of the three items are placed in the
%   result, then we use the \meta{continuation}, placing the remaining
%   two items after it.  When we begin this loop, the three items really
%   belong to the comma list, the first \cs{q_mark} is taken as a
%   delimiter to the \texttt{use_ii} function, and the continuation is
%   \texttt{use_ii} itself.  When we reach the last two items of the
%   original token list, \cs{q_mark} is taken as a third item, and now
%   the second \cs{q_mark} serves as a delimiter to \texttt{use_ii},
%   switching to the other \meta{continuation}, \texttt{use_iii}, which
%   uses the \meta{separator between final two}.
%    \begin{macrocode}
\cs_new:Npn \clist_use:Nnnn #1#2#3#4
  {
    \clist_if_exist:NTF #1
      {
        \int_case:nnF { \clist_count:N #1 }
          {
            { 0 } { }
            { 1 } { \exp_after:wN \@@_use:wwn #1 , , { } }
            { 2 } { \exp_after:wN \@@_use:wwn #1 , {#2} }
          }
          {
            \exp_after:wN \@@_use:nwwwwnwn
            \exp_after:wN { \exp_after:wN } #1 ,
            \q_mark , { \@@_use:nwwwwnwn {#3} }
            \q_mark , { \@@_use:nwwn {#4} }
            \q_stop { }
          }
      }
      {
        \__kernel_msg_expandable_error:nnn
          { kernel } { bad-variable } {#1}
      }
  }
\cs_generate_variant:Nn \clist_use:Nnnn { c }
\cs_new:Npn \@@_use:wwn #1 , #2 , #3 { \exp_not:n { #1 #3 #2 } }
\cs_new:Npn \@@_use:nwwwwnwn
    #1#2 , #3 , #4 , #5 \q_mark , #6#7 \q_stop #8
  { #6 {#3} , {#4} , #5 \q_mark , {#6} #7 \q_stop { #8 #1 #2 } }
\cs_new:Npn \@@_use:nwwn #1#2 , #3 \q_stop #4
  { \exp_not:n { #4 #1 #2 } }
\cs_new:Npn \clist_use:Nn #1#2
  { \clist_use:Nnnn #1 {#2} {#2} {#2} }
\cs_generate_variant:Nn \clist_use:Nn { c }
%    \end{macrocode}
% \end{macro}
% \end{macro}
% \end{macro}
%
% \subsection{Using a single item}
%
% \begin{macro}{\clist_item:Nn, \clist_item:cn}
% \begin{macro}{\@@_item:nnnN, \@@_item:ffoN, \@@_item:ffnN}
% \begin{macro}{\@@_item_N_loop:nw}
%   To avoid needing to test the end of the list at each step,
%   we first compute the \meta{length} of the list. If the item number
%   is~$0$, less than $-\meta{length}$, or more than $\meta{length}$,
%   the result is empty. If it is negative, but not less than $-\meta{length}$,
%   add $\meta{length}+1$ to the item number before performing the loop.
%   The loop itself is very simple, return the item if the counter
%   reached~$1$, otherwise, decrease the counter and repeat.
%    \begin{macrocode}
\cs_new:Npn \clist_item:Nn #1#2
  {
    \@@_item:ffoN
      { \clist_count:N #1 }
      { \int_eval:n {#2} }
      #1
      \@@_item_N_loop:nw
  }
\cs_new:Npn \@@_item:nnnN #1#2#3#4
  {
    \int_compare:nNnTF {#2} < 0
      {
        \int_compare:nNnTF {#2} < { - #1 }
          { \use_none_delimit_by_q_stop:w }
          { \exp_args:Nf #4 { \int_eval:n { #2 + 1 + #1 } } }
      }
      {
        \int_compare:nNnTF {#2} > {#1}
          { \use_none_delimit_by_q_stop:w }
          { #4 {#2} }
      }
    { } , #3 , \q_stop
  }
\cs_generate_variant:Nn \@@_item:nnnN { ffo, ff }
\cs_new:Npn \@@_item_N_loop:nw #1 #2,
  {
    \int_compare:nNnTF {#1} = 0
      { \use_i_delimit_by_q_stop:nw { \exp_not:n {#2} } }
      { \exp_args:Nf \@@_item_N_loop:nw { \int_eval:n { #1 - 1 } } }
  }
\cs_generate_variant:Nn \clist_item:Nn { c }
%    \end{macrocode}
% \end{macro}
% \end{macro}
% \end{macro}
%
% \begin{macro}{\clist_item:nn}
% \begin{macro}{
%     \@@_item_n:nw,
%     \@@_item_n_loop:nw,
%     \@@_item_n_end:n,
%     \@@_item_n_strip:n,
%     \@@_item_n_strip:w}
%   This starts in the same way as \cs{clist_item:Nn} by counting the items
%   of the comma list. The final item should be space-trimmed before being
%   brace-stripped, hence we insert a couple of odd-looking
%   \cs{prg_do_nothing:} to avoid losing braces. Blank items are ignored.
%    \begin{macrocode}
\cs_new:Npn \clist_item:nn #1#2
  {
    \@@_item:ffnN
      { \clist_count:n {#1} }
      { \int_eval:n {#2} }
      {#1}
      \@@_item_n:nw
  }
\cs_new:Npn \@@_item_n:nw #1
  { \@@_item_n_loop:nw {#1} \prg_do_nothing: }
\cs_new:Npn \@@_item_n_loop:nw #1 #2,
  {
    \exp_args:No \tl_if_blank:nTF {#2}
      { \@@_item_n_loop:nw {#1} \prg_do_nothing: }
      {
        \int_compare:nNnTF {#1} = 0
          { \exp_args:No \@@_item_n_end:n {#2} }
          {
            \exp_args:Nf \@@_item_n_loop:nw
              { \int_eval:n { #1 - 1 } }
              \prg_do_nothing:
          }
      }
  }
\cs_new:Npn \@@_item_n_end:n #1 #2 \q_stop
  { \tl_trim_spaces_apply:nN {#1} \@@_item_n_strip:n }
\cs_new:Npn \@@_item_n_strip:n #1 { \@@_item_n_strip:w #1 , }
\cs_new:Npn \@@_item_n_strip:w #1 , { \exp_not:n {#1} }
%    \end{macrocode}
% \end{macro}
% \end{macro}
%
% \begin{macro}{\clist_rand_item:n, \clist_rand_item:N, \clist_rand_item:c}
% \begin{macro}{\@@_rand_item:nn}
%   The |N|-type function is not implemented through the |n|-type
%   function for efficiency: for instance comma-list variables do not
%   require space-trimming of their items.  Even testing for emptyness
%   of an |n|-type comma-list is slow, so we count items first and use
%   that both for the emptyness test and the pseudo-random integer.
%   Importantly, \cs{clist_item:Nn} and \cs{clist_item:nn} only evaluate
%   their argument once.
%    \begin{macrocode}
\cs_new:Npn \clist_rand_item:n #1
  { \exp_args:Nf \@@_rand_item:nn { \clist_count:n {#1} } {#1} }
\cs_new:Npn \@@_rand_item:nn #1#2
  {
    \int_compare:nNnF {#1} = 0
      { \clist_item:nn {#2} { \int_rand:nn { 1 } {#1} } }
  }
\cs_new:Npn \clist_rand_item:N #1
  {
    \clist_if_empty:NF #1
      { \clist_item:Nn #1 { \int_rand:nn { 1 } { \clist_count:N #1 } } }
  }
\cs_generate_variant:Nn \clist_rand_item:N { c }
%    \end{macrocode}
% \end{macro}
% \end{macro}
%
% \subsection{Viewing comma lists}
%
% \begin{macro}{\clist_show:N, \clist_show:c, \clist_log:N, \clist_log:c, \@@_show:NN}
%   Apply the general \cs{__kernel_chk_defined:NT} and
%   \cs{msg_show:nnnnnn}.
%    \begin{macrocode}
\cs_new_protected:Npn \clist_show:N { \@@_show:NN \msg_show:nnxxxx }
\cs_generate_variant:Nn \clist_show:N { c }
\cs_new_protected:Npn \clist_log:N { \@@_show:NN \msg_log:nnxxxx }
\cs_generate_variant:Nn \clist_log:N { c }
\cs_new_protected:Npn \@@_show:NN #1#2
  {
    \__kernel_chk_defined:NT #2
      {
        #1 { LaTeX/kernel } { show-clist }
          { \token_to_str:N #2 }
          { \clist_map_function:NN #2 \msg_show_item:n }
          { } { }
      }
  }
%    \end{macrocode}
% \end{macro}
%
% \begin{macro}{\clist_show:n, \clist_log:n, \@@_show:Nn}
%   A variant of the above: no existence check, empty first argument for
%   the message.
%    \begin{macrocode}
\cs_new_protected:Npn \clist_show:n { \@@_show:Nn \msg_show:nnxxxx }
\cs_new_protected:Npn \clist_log:n { \@@_show:Nn \msg_log:nnxxxx }
\cs_new_protected:Npn \@@_show:Nn #1#2
  {
    #1 { LaTeX/kernel } { show-clist }
      { } { \clist_map_function:nN {#2} \msg_show_item:n } { } { }
  }
%    \end{macrocode}
% \end{macro}
%
% \subsection{Scratch comma lists}
%
% \begin{variable}{\l_tmpa_clist, \l_tmpb_clist}
% \begin{variable}{\g_tmpa_clist, \g_tmpb_clist}
%   Temporary comma list variables.
%    \begin{macrocode}
\clist_new:N \l_tmpa_clist
\clist_new:N \l_tmpb_clist
\clist_new:N \g_tmpa_clist
\clist_new:N \g_tmpb_clist
%    \end{macrocode}
% \end{variable}
% \end{variable}
%
%    \begin{macrocode}
%</initex|package>
%    \end{macrocode}
%
% \end{implementation}
%
% \PrintIndex
