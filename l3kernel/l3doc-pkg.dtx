% \iffalse meta-comment
%
%% File: l3doc.dtx
%
% Copyright (C) 1990-2019 The LaTeX3 Project
%
% It may be distributed and/or modified under the conditions of the
% LaTeX Project Public License (LPPL), either version 1.3c of this
% license or (at your option) any later version.  The latest version
% of this license is in the file
%
%    https://www.latex-project.org/lppl.txt
%
% This file is part of the "l3kernel bundle" (The Work in LPPL)
% and all files in that bundle must be distributed together.
%
% -----------------------------------------------------------------------
%
% The development version of the bundle can be found at
%
%    https://github.com/latex3/latex3
%
% for those people who are interested.
%
%<*driver|package>
\RequirePackage{expl3,xparse,calc}
%</driver|package>
%<*driver>
\documentclass[full]{l3doc}
\usepackage{framed}
\begin{document}
  \DocInput{\jobname.dtx}
\end{document}
%</driver>
%
% This isn't included in the typeset documentation because it's a bit
% ugly:
% ^^A Why? --Phelype
%<*package>
\ProvidesExplPackage{l3doc}{2019-08-27}{}
  {L3 Experimental documentation package}
%</package>
% \fi
%
% \title{^^A
%   The \pkg{l3doc} package\\ Experimental documentation features^^A
% }
%
% \author{^^A
%  The \LaTeX3 Project\thanks
%    {^^A
%      E-mail:
%        \href{mailto:latex-team@latex-project.org}
%          {latex-team@latex-project.org}^^A
%    }^^A
% }
%
% \date{Released 2019-08-25}
%
% \maketitle
%
%
% \begin{documentation}
%
% \section{Introduction}
%
% This is an ad-hoc class for documenting the \pkg{expl3} bundle, a
% collection of modules or packages that make up \LaTeX3's programming
% environment.  Eventually it will replace the \cls{ltxdoc} class for
% \LaTeX3, but not before the good ideas in \pkg{hypdoc}, \cls{xdoc2},
% \pkg{docmfp}, and \cls{gmdoc} are incorporated.
%
% \textbf{It is much less stable than the main \pkg{expl3} packages.
%   Use at own risk!}
%
% It is written as a \enquote{self-contained} docstrip file: executing
% |latex l3doc.dtx| generates the \file{l3doc.cls} file and typesets
% this documentation; execute |tex l3doc.dtx| to only generate
% \file{l3doc.cls}.
%
% \section{Features of other packages}
%
% This class builds on the \pkg{ltxdoc} class and the \pkg{doc} package,
% but in the time since they were originally written some improvements
% and replacements have appeared that we would like to use as
% inspiration.
%
% These packages or classes are \pkg{hypdoc}, \pkg{docmfp}, \pkg{gmdoc},
% and \pkg{xdoc}.  I have summarised them below in order to work out
% what sort of features we should aim at a minimum for \pkg{l3doc}.
%
% \subsection{The \pkg{hypdoc} package}
%
% This package provides hyperlink support for the \pkg{doc} package.  I
% have included it in this list to remind me that cross-referencing
% between documentation and implementation of methods is not very
% good. (\emph{E.g.}, it would be nice to be able to automatically
% hyperlink the documentation for a function from its implementation and
% vice-versa.)
%
% \subsection{The \pkg{docmfp} package}
%
% \begin{itemize}
%   \item Provides \cs{DescribeRoutine} and the \env{routine}
%     environment (\emph{etc.}) for MetaFont and MetaPost code.
%   \item Provides \cs{DescribeVariable} and the \env{variable}
%     environment (\emph{etc.})  for more general code.
%   \item Provides \cs{Describe} and the \env{Code} environment
%     (\emph{etc.})  as a generalisation of the above two
%     instantiations.
%   \item Small tweaks to the DocStrip system to aid non-\LaTeX{} use.
% \end{itemize}
%
% \subsection{The \pkg{xdoc2} package}
%
% \begin{itemize}
%   \item Two-sided printing.
%   \item \cs{NewMacroEnvironment}, \cs{NewDescribeEnvironment}; similar
%     idea to \pkg{docmfp} but more comprehensive.
%   \item Tons of small improvements.
% \end{itemize}
%
% \subsection{The \pkg{gmdoc} package}
%
% Radical re-implementation of \pkg{doc} as a package or class.
% \begin{itemize}
%   \item Requires no |\begin{macrocode}| blocks!
%   \item Automatically inserts |\begin{macro}| blocks!
%   \item And a whole bunch of other little things.
% \end{itemize}
%
% \section{Problems \& Todo}
%
% Problems at the moment:
% (1)~not flexible in the types of things that can be documented;
% (2)~no obvious link between the |\begin{function}| environment for
%     documenting things to the |\begin{macro}| function that's used
%     analogously in the implementation.
%
% The \env{macro} should probably be renamed to \env{function} when it
% is used within an implementation section.  But they should have the
% same syntax before that happens!
%
% Furthermore, we need another \enquote{layer} of documentation commands
% to account for \enquote{user-macro} as opposed to
% \enquote{code-functions}; the \pkg{expl3} functions should be
% documented differently, probably, to the \pkg{xparse} user macros (at
% least in terms of indexing).
%
% In no particular order, a list of things to do:
% \begin{itemize}
%   \item Rename \env{function}/\env{macro} environments to better
%     describe their use.
%   \item Generalise \env{function}/\env{macro} for documenting
%     \enquote{other things}, such as environment names, package
%     options, even keyval options.
%   \item New function like \tn{part} but for files (remove awkward
%     \enquote{File} as \tn{partname}).
%   \item Something better to replace \cs{StopEventually}; I'm thinking
%     two environments \env{documentation} and \env{implementation} that
%     can conditionally typeset/ignore their material.  (This has been
%     implemented but needs further consideration.)
%   \item Hyperlink documentation and implementation of macros (see the
%     \textsc{dtx} file of \pkg{svn-multi} v2 as an example).  This is
%     partially done, now, but should be improved.
% \end{itemize}
%
% \section{Documentation}
%
% \subsection{Configuration}
%
% Before class options are processed, \pkg{l3doc} loads a configuration
% file \file{l3doc.cfg} if it exists, allowing you to customise the
% behaviour of the class without having to change the documentation
% source files.
%
% For example, to produce documentation on letter-sized paper instead of
% the default A4 size, create \file{l3doc.cfg} and include the line
% \begin{verbatim}
% \PassOptionsToClass{letterpaper}{l3doc}
% \end{verbatim}
%
% By default, \pkg{l3doc} selects the |T1| font encoding and loads the
% Latin Modern fonts.  To prevent this, use the class option
% |cm-default|.
%
% \subsection{Partitioning documentation and implementation}
%
% \pkg{doc} uses the \cs{OnlyDocumentation}/\cs{AlsoImplementation}
% macros to guide the use of \cs{StopEventually}|{}|, which is intended
% to be placed to partition the documentation and implementation within
% a single \file{.dtx} file.
%
% This isn't very flexible, since it assumes that we \emph{always} want
% to print the documentation.  For the \pkg{expl3} sources, I wanted to
% be be able to input \file{.dtx} files in two modes: only displaying
% the documentation, and only displaying the implementation.  For
% example:
% \begin{verbatim}
% \DisableImplementation
% \DocInput{l3basics,l3prg,...}
% \EnableImplementation
% \DisableDocumentation
% \DocInputAgain
% \end{verbatim}
%
% The idea being that the entire \pkg{expl3} bundle can be documented,
% with the implementation included at the back.  Now, this isn't
% perfect, but it's a start.
%
% Use |\begin{documentation}...\end{documentation}| around the
% documentation, and |\begin{implementation}...\end{implementation}|
% around the implementation.  The
% \cs{EnableDocumentation}/\cs{EnableImplementation} causes them to
% be typeset when the \file{.dtx} file is \cs{DocInput}; use
% \cs{DisableDocumentation}/\cs{DisableImplementation} to omit the
% contents of those environments.
%
% Note that \cs{DocInput} now takes comma-separated arguments, and
% \cs{DocInputAgain} can be used to re-input all \file{.dtx} files
% previously input in this way.
%
% \subsection{General text markup}
%
% Many of the commands in this section come from \pkg{ltxdoc} with some
% improvements.
%
% \begin{function}{\cmd, \cs}
%   \begin{syntax}
%     \cmd{\cmd} \oarg{options} \meta{control sequence}\\
%     \cs{cs} \oarg{options} \marg{csname}
%   \end{syntax}
%   These commands are provided to typeset control sequences.
%   |\cmd\foo| produces \enquote{\cmd\foo} and |\cs{foo}| produces the
%   same.  In general, \cs{cs} is more robust since
%   it doesn't rely on catcodes being \enquote{correct} and is therefore
%   recommended.
%
%   These commands are aware of the |@@| \pkg{l3docstrip} syntax and
%   replace such instances correctly in the typeset documentation.
%   This only happens after a |%<@@=|\meta{module}|>| declaration.
%
%   Additionally, commands can be used in the argument of \cs{cs}.  For
%   instance, |\cs{\meta{name}:\meta{signature}}| produces
%   \cs[no-index]{\meta{name}:\meta{signature}}.
%
%   The \meta{options} are a key--value list which can contain the
%   following keys:
%   \begin{itemize}
%     \item |index=|\meta{name}: the \meta{csname} is indexed as if
%       one had written \cs{cs}\Arg{name}.
%     \item |no-index|: the \meta{csname} is not indexed.
%     \item |module=|\meta{module}: the \meta{csname} is indexed in
%       the list of commands from the \meta{module}; the \meta{module}
%       can in particular be |TeX| for \enquote{\TeX{} and \LaTeXe{}}
%       commands, or empty for commands which should be placed in the
%       main index.  By default, the \meta{module} is deduced
%       automatically from the command name.
%     \item |replace| is a boolean key (\texttt{true} by default) which
%       indicates whether to replace |@@| as \pkg{l3docstrip} does.
%   \end{itemize}
%   These commands allow hyphenation of control sequences after (most) underscores.
%   By default, a hyphen is used to mark the hyphenation, but this can be changed with
%   the \texttt{cs-break-nohyphen} class option.
%   To disable hyphenation of control sequencies entirely, use \texttt{cs-break-off}.
% \end{function}
%
%
% \begin{function}{\tn}
%   \begin{syntax}
%     \cs{tn} \oarg{options} \marg{csname}
%   \end{syntax}
%   Analoguous to \cs{cs} but intended for \enquote{traditional} \TeX{}
%   or \LaTeXe{} commands; they are indexed accordingly.  This is in
%   fact equivalent to \cs{cs} |[module=TeX, replace=false,|
%   \meta{options}|]| \Arg{csname}.
% \end{function}
%
% \begin{function}{\meta}
%   \begin{syntax}
%     \cs{meta} \Arg{name}
%   \end{syntax}
%   \cs{meta} typesets the \meta{name} italicised in \meta{angle
%     brackets}.  Within a \env{function} environment or similar, angle
%   brackets |<...>| are set up to be a shorthand for |\meta{...}|.
%
%   This function has additional functionality over its \pkg{ltxdoc}
%   versions; underscores can be used to subscript material as in math
%   mode.  For example, |\meta{arg_{xy}}| produces
%   \enquote{\meta{arg_{xy}}}.
% \end{function}
%
% \begin{function}{\Arg, \marg, \oarg, \parg}
%   \begin{syntax}
%     |\Arg| \Arg{name}
%   \end{syntax}
%   Typesets the \meta{name} as for \cs{meta} and wraps it in braces.
%
%   The \cs{marg}/\cs{oarg}/\cs{parg} versions follow from \pkg{ltxdoc}
%   in being used for \enquote{mandatory} or \enquote{optional} or
%   \enquote{picture} brackets as per \LaTeXe{} syntax.
% \end{function}
%
% \begin{function}{\file, \env, \pkg, \cls}
%   \begin{syntax}
%     \cs{pkg} \Arg{name}
%   \end{syntax}
%   These all take one argument and are intended to be used as semantic
%   commands for representing files, environments, package names, and
%   class names, respectively.
% \end{function}
%
% \begin{function}{\NB, \NOTE}
%   \begin{syntax}
%     \cs{NB} \marg{tag} \marg{comments}
%     \verb|\begin{NOTE}| \marg{tag}
%     \qquad\meta{comments}
%     \verb|\end{NOTE}|
%   \end{syntax}
%   Make notes in the source that are not typeset by default. When the \verb|show-notes|
%   class option is active, the comments are typeset in a detokenized and verbatim mode, respectively.
% \end{function}
%
% \subsection{Describing functions in the documentation}
%
% \DescribeEnv{function}
% \DescribeEnv{syntax}
% Two heavily-used environments are defined to describe the syntax of
% \pkg{expl3} functions and variables.
% \begin{framed}
%   \vspace{-\baselineskip}
% \begin{verbatim}
% \begin{function}{\function_one:, \function_two:}
%   \begin{syntax}
%     |\foo_bar:| \Arg{meta} \meta{test_1}
%   \end{syntax}
% \meta{description}
% \end{function}
% \end{verbatim}
%   \hrulefill
%   \par
%   \hspace*{0.25\textwidth}
%   \begin{minipage}{0.5\textwidth}
%     \begin{function}{\function_one:, \function_two:}
%       \begin{syntax}
%         |\foo_bar:| \Arg{meta} \meta{test_1}
%       \end{syntax}
%       \meta{description}
%     \end{function}
%   \end{minipage}
% \end{framed}
%
% Function environments take an optional argument to indicate whether
% the function(s) it describes are expandable or restricted-expandable
% or defined in conditional forms. Use |EXP|, |rEXP|, |TF|, |pTF|, or |noTF| for
% this; note that |pTF| implies |EXP| since predicates must always be
% expandable, and that |noTF| means that the function without |TF|
% should be documented in addition to |TF|.  As an example:
% \begin{framed}
%   \vspace{-\baselineskip}
% \begin{verbatim}
% \begin{function}[pTF]{\cs_if_exist:N}
%   \begin{syntax}
%     \cs{cs_if_exist_p:N} \meta{cs}
%   \end{syntax}
% \meta{description}
% \end{function}
% \end{verbatim}
%   \hrulefill
%   \par
%   \hspace*{0.25\textwidth}
%   \begin{minipage}{0.5\textwidth}
%     \begin{function}[pTF]{\cs_if_exist:N}
%       \begin{syntax}
%         \cs{cs_if_exist_p:N} \meta{cs}
%       \end{syntax}
%       \meta{description}
%     \end{function}
%   \end{minipage}
% \end{framed}
%
% \DescribeEnv{variable}
% If you are documenting a variable instead of a function, use the
% \env{variable} environment instead; it behaves identically to the
% \env{function} environment above.
%
% \DescribeEnv{texnote}
% This environment is used to call out sections within \env{function}
% and similar that are only of interest to seasoned \TeX{} developers.
%
% \subsection{Describing functions in the implementation}
%
% \DescribeEnv{macro}
% The well-used environment from \LaTeXe{} for marking up the
% implementation of macros/functions remains the \env{macro}
% environment.  Some changes in \pkg{l3doc}: it now accepts
% comma-separated lists of functions, to avoid a very large number of
% consecutive |\end{macro}| statements.
% Spaces and new lines are ignored (the option |[verb]| prevents this).
% \begin{verbatim}
% % \begin{macro}{\foo:N, \foo:c}
% %   \begin{macrocode}
% ... code for \foo:N and \foo:c ...
% %   \end{macrocode}
% % \end{macro}
% \end{verbatim}
% If you are documenting an auxiliary macro, it's generally not
% necessary to highlight it as much and you also don't need to check it
% for, say, having a test function and having a documentation chunk
% earlier in a \env{function} environment.  \pkg{l3doc} will pick up these
% cases from the presence of |__| in the name, or you may force marking
% as internal by using |\begin{macro}[int]| to mark it as such. The margin
% call-out is then printed in grey for such cases.
%
% For documenting \pkg{expl3}-type conditionals, you may also pass this
% environment a |TF| option (and omit it from the function name) to
% denote that the function is provided with |T|, |F|, and |TF| suffixes.
% A similar |pTF| option prints both |TF| and |_p| predicate forms.
% An option |noTF| prints both the |TF| forms and a form with neither
% |T| nor |F|, to document functions such as \cs[no-index]{prop_get:NN}
% which also have conditional forms (\cs[no-index]{prop_get:NNTF}).
%
%
% \DescribeMacro{\TestFiles}
% \cs{TestFiles}\marg{list of files} is used to indicate which test
% files are used for the current code; they are printed in the
% documentation.
%
% \DescribeMacro{\UnitTested}
% Within a \env{macro} environment, it is a good idea to mark whether a
% unit test has been created for the commands it defines.  This is
% indicated by writing \cs{UnitTested} anywhere within |\begin{macro}|
%   \dots |\end{macro}|.
%
% If the class option |checktest| is enabled, then it is an \emph{error}
% to have a \env{macro} environment without a call to
% \file{Testfiles}.  This is intended for large packages such as
% \pkg{expl3} that should have absolutely comprehensive tests suites and
% whose authors may not always be as sharp at adding new tests with new
% code as they should be.
%
% \DescribeMacro{\TestMissing}
% If a function is missing a test, this may be flagged by writing (as
% many times as needed) \cs{TestMissing} \marg{explanation of test
%   required}.  These missing tests are summarised in the listing
% printed at the end of the compilation run.
%
% \DescribeEnv{variable}
% When documenting variable definitions, use the \env{variable}
% environment instead.  Here it behaves identically to the
% \env{macro} environment, except that if the class option |checktest|
% is enabled, variables are not required to have a test file.
%
% \DescribeEnv{arguments}
% Within a \env{macro} environment, you may use the \env{arguments}
% environment to describe the arguments taken by the function(s).  It
% behaves like a modified enumerate environment.
% \begin{verbatim}
% % \begin{macro}{\foo:nn, \foo:VV}
% % \begin{arguments}
% %   \item Name of froozle to be frazzled
% %   \item Name of muble to be jubled
% % \end{arguments}
% %   \begin{macrocode}
% ... code for \foo:nn and \foo:VV ...
% %   \end{macrocode}
% % \end{macro}
% \end{verbatim}
%
%
% \subsection{Keeping things consistent}
%
% Whenever a function is either documented or defined with
% \env{function} and \env{macro} respectively, its name is stored in a
% sequence for later processing.
%
% At the end of the document (\emph{i.e.}, after the \file{.dtx} file
% has finished processing), the list of names is analysed to check
% whether all defined functions have been documented and vice versa. The
% results are printed in the console output.
%
% If you need to do more serious work with these lists of names, take a
% look at the implementation for the data structures and methods used to
% store and access them directly.
%
% \subsection{Documenting templates}
%
% The following macros are provided for documenting templates; might end
% up being something completely different but who knows.
% \begin{quote}\parskip=0pt\obeylines
%   |\begin{TemplateInterfaceDescription}| \Arg{template type name}
%   |  \TemplateArgument{none}{---}|
%   \textsc{or one or more of these:}
%   |  \TemplateArgument| \Arg{arg no} \Arg{meaning}
%   \textsc{and}
%   |\TemplateSemantics|
%   |  | \meta{text describing the template type semantics}
%   |\end{TemplateInterfaceDescription}|
% \end{quote}
%
% \begin{quote}\parskip=0pt\obeylines
%   |\begin{TemplateDescription}| \Arg{template type name} \Arg{name}
%   \textsc{one or more of these:}
%   |  \TemplateKey| \marg{key name} \marg{type of key}
%   |    |\marg{textual description of meaning}
%   |    |\marg{default value if any}
%   \textsc{and}
%   |\TemplateSemantics|
%   |  | \meta{text describing special additional semantics of the template}
%   |\end{TemplateDescription}|
% \end{quote}
%
% \begin{quote}\parskip=0pt\obeylines
%   |\begin{InstanceDescription}| \oarg{text to specify key column width (optional)}
%   \hfill\marg{template type name}\marg{instance name}\marg{template name}
%   \textsc{one or more of these:}
%   |  \InstanceKey| \marg{key name} \marg{value}
%   \textsc{and}
%   |\InstanceSemantics|
%   |  | \meta{text describing the result of this instance}
%   |\end{InstanceDescription}|
% \end{quote}
%
% \end{documentation}
%
% \begin{implementation}
%
% \section{\pkg{l3doc} implementation}
%
%    \begin{macrocode}
%<@@=codedoc>
%    \end{macrocode}
%
%    \begin{macrocode}
%<*package>
%    \end{macrocode}
%
% \subsection{Variables}
%
% \begin{variable}{\g_docinput_clist}
%   The list of files which have been input through \cs{DocInput}.
%    \begin{macrocode}
\clist_new:N \g_docinput_clist
%    \end{macrocode}
% \end{variable}
%
% \begin{variable}{\g_doc_functions_seq, \g_doc_macros_seq}
%   All functions documented through \env{function}, and all macros
%   introduced through \env{macro}.  They can be compared to see what
%   documentation or code is missing.
%    \begin{macrocode}
\seq_new:N \g_doc_functions_seq
\seq_new:N \g_doc_macros_seq
%    \end{macrocode}
% \end{variable}
%
% \begin{variable}{\l_@@_detect_internals_bool, \l_@@_detect_internals_tl}
%   If \texttt{true}, \pkg{l3doc} will check for use of internal
%   commands \cs[no-index]{__\meta{pkg}_\ldots{}} from other packages in
%   the argument of the \texttt{macro} environment, and in the code typeset in
%   \texttt{macrocode} environments, but not in~\cs{cs}.  Also a token list
%   to store temporary data for this purpose.
%    \begin{macrocode}
\bool_new:N \l_@@_detect_internals_bool
\bool_set_true:N \l_@@_detect_internals_bool
\tl_new:N \l_@@_detect_internals_tl
\tl_new:N \l_@@_detect_internals_cs_tl
%    \end{macrocode}
% \end{variable}
%
% \begin{variable}{\l_@@_output_coffin}
%   The \env{function} environment is typeset by combining coffins
%   containing various pieces (function names, description, \emph{etc.})
%   into this coffin.
%    \begin{macrocode}
\coffin_new:N \l_@@_output_coffin
%    \end{macrocode}
% \end{variable}
%
% \begin{variable}
%   {\l_@@_functions_coffin, \l_@@_descr_coffin, \l_@@_syntax_coffin}
%   These coffins contain respectively the list of function names
%   (argument of the \env{function} environment), the text between
%   |\begin{function}| and |\end{function}|, and the syntax given in the
%   \env{syntax} environment.
%    \begin{macrocode}
\coffin_new:N \l_@@_functions_coffin
\coffin_new:N \l_@@_descr_coffin
\coffin_new:N \l_@@_syntax_coffin
%    \end{macrocode}
% \end{variable}
%
% \begin{variable}{\g_@@_syntax_box}
%   The contents of the \env{syntax} environment are typeset in this box
%   before being transferred to \cs{l_@@_syntax_coffin}.
%    \begin{macrocode}
\box_new:N \g_@@_syntax_box
%    \end{macrocode}
% \end{variable}
%
% \begin{variable}{\l_@@_in_function_bool}
%   True when inside a \texttt{function} or \texttt{variable}
%   environment.  Used by the \texttt{syntax} environment to determine
%   its behaviour.
%    \begin{macrocode}
\bool_new:N \l_@@_in_function_bool
%    \end{macrocode}
% \end{variable}
%
% \begin{variable}{\l_@@_long_name_bool, \l_@@_trial_width_dim}
%   The boolean \cs{l_@@_long_name_bool} is \texttt{true} if the width
%   \cs{l_@@_trial_width_dim} of the coffin \cs{l_@@_functions_coffin}
%   (containing the current function names) is bigger than the space
%   available in the margin.
%    \begin{macrocode}
\bool_new:N \l_@@_long_name_bool
\dim_new:N \l_@@_trial_width_dim
%    \end{macrocode}
% \end{variable}
%
% \begin{variable}{\l_@@_nested_macro_int}
%   The nesting of \env{macro} environments (this is now~$0$ outside a
%   \env{macro} environment).
%    \begin{macrocode}
\int_new:N \l_@@_nested_macro_int
%    \end{macrocode}
% \end{variable}
%
% \begin{variable}
%   {
%     \l_@@_macro_tested_bool,
%     \g_@@_missing_tests_prop,
%     \g_@@_not_tested_seq,
%     \g_@@_testfiles_seq,
%   }
%   A boolean describing whether the current macro has tests, and some
%   global structures which contain information about test files and
%   which tests are missing.
%    \begin{macrocode}
\bool_new:N \l_@@_macro_tested_bool
\prop_new:N \g_@@_missing_tests_prop
\seq_new:N \g_@@_not_tested_seq
\seq_new:N \g_@@_testfiles_seq
%    \end{macrocode}
% \end{variable}
%
% \begin{variable}
%   {
%     \l_@@_macro_internal_set_bool,
%     \l_@@_macro_internal_bool,
%     \l_@@_macro_TF_bool,
%     \l_@@_macro_pTF_bool,
%     \l_@@_macro_noTF_bool,
%     \l_@@_macro_EXP_bool,
%     \l_@@_macro_rEXP_bool,
%     \l_@@_macro_var_bool,
%     \l_@@_override_module_tl,
%     \l_@@_macro_documented_tl,
%   }
%   Contain information about some options of function/macro
%   environments.  We initialize \cs{l_@@_override_module_tl} to avoid
%   overriding module names by an empty name (meaning no module).
%    \begin{macrocode}
\bool_new:N \l_@@_macro_internal_set_bool
\bool_new:N \l_@@_macro_internal_bool
\bool_new:N \l_@@_macro_TF_bool
\bool_new:N \l_@@_macro_pTF_bool
\bool_new:N \l_@@_macro_noTF_bool
\bool_new:N \l_@@_macro_EXP_bool
\bool_new:N \l_@@_macro_rEXP_bool
\bool_new:N \l_@@_macro_var_bool
\tl_new:N \l_@@_override_module_tl
\tl_set:Nn \l_@@_override_module_tl { \q_no_value }
\tl_new:N \l_@@_macro_documented_tl
%    \end{macrocode}
% \end{variable}
%
% \begin{variable}
%   {
%     \g_@@_checkfunc_bool,
%     \g_@@_checktest_bool,
%     \g_@@_cs_break_bool,
%     \g_@@_show_notes_bool,
%     \g_@@_kernel_bool
%   }
%   Information about package options.
%    \begin{macrocode}
\bool_new:N \g_@@_checkfunc_bool
\bool_new:N \g_@@_checktest_bool
\bool_new:N \g_@@_kernel_bool
\bool_new:N \g_@@_cs_break_bool
\bool_new:N \g_@@_show_notes_bool
\bool_gset_true:N \g_@@_cs_break_bool
%    \end{macrocode}
% \end{variable}
%
% \begin{variable}{\l_@@_tmpa_tl, \l_@@_tmpb_tl, \l_@@_tmpa_int, \l_@@_tmpa_seq}
%   Some temporary variables.
%    \begin{macrocode}
\tl_new:N \l_@@_tmpa_tl
\tl_new:N \l_@@_tmpb_tl
\int_new:N \l_@@_tmpa_int
\int_new:N \l_@@_tmpa_seq
%    \end{macrocode}
% \end{variable}
%
% \begin{variable}{\l_@@_names_block_tl}
%   List of local sequence variables (produced through
%   \cs{@@_lseq_name:n}), one for each set of variants in a
%   \env{function} or \env{macro} environment.  More precisely these
%   sequences are named after the base forms, such as \cs{clist_count:n}
%   or \cs{clist_count:N} (which are not variants).  Each of these
%   sequences have the base name (without any signature) as their first
%   item, followed by the list of variant's signatures, or
%   \cs{scan_stop:} to denote the absence of signature (no colon).
%    \begin{macrocode}
\tl_new:N \l_@@_names_block_tl
%    \end{macrocode}
% \end{variable}
%
% \begin{variable}{\g_@@_variants_seq}
%   Stores rather temporarily the list of variants (signatures only) of
%   a function/macro that is being documented.  It is global because we
%   need it to keep its value throughout cells of an alignment.
%    \begin{macrocode}
\seq_new:N \g_@@_variants_seq
%    \end{macrocode}
% \end{variable}
%
% \begin{variable}{\l_@@_names_verb_bool}
%   Set to |true| if the main argument of a macro/function environment
%   should be used as is, without removing any comma or space.
%    \begin{macrocode}
\bool_new:N \l_@@_names_verb_bool
%    \end{macrocode}
% \end{variable}
%
% \begin{variable}{\l_@@_names_seq}
%   List of functions/environments/\ldots{} appearing as arguments of a
%   given \env{function} or \env{macro} environment.  These are the
%   names after conversion of |_@@| and |@@| to |__|\meta{module name}
%   and other sanitizing.
%    \begin{macrocode}
\seq_new:N \l_@@_names_seq
%    \end{macrocode}
% \end{variable}
%
% \begin{variable}{\g_@@_nested_names_seq}
%   Collects all macros in nested \env{macro} environments, to use them
%   in the \enquote{End definition} text.
%    \begin{macrocode}
\seq_new:N \g_@@_nested_names_seq
%    \end{macrocode}
% \end{variable}
%
% \begin{variable}
%   {
%     \l_@@_index_macro_tl, \l_@@_index_key_tl,
%     \l_@@_index_module_tl, \l_@@_index_internal_bool,
%     \l_@@_macro_do_not_index_tl
%   }
%   When analyzing a control sequence found within a \env{macrocode}
%   environment, \cs{l_@@_index_macro_tl} holds the control sequence
%   (partially a string), \cs{l_@@_index_key_tl} holds the future
%   sort key in the index, and \cs{l_@@_index_module_tl} is the
%   subindex in which the control sequence should be listed.
%   \cs{l_@@_index_internal_bool} indicates when the control sequence is
%   internal and should be indexed in a slightly different subindex.
%   Finally, \cs{l_@@_macro_do_not_index_tl} indicates control sequences
%   which should not be indexed in a specifiv \env{macro} envronment.
%    \begin{macrocode}
\tl_new:N \l_@@_index_macro_tl
\tl_new:N \l_@@_index_key_tl
\tl_new:N \l_@@_index_module_tl
\tl_new:N \l_@@_macro_do_not_index_tl
\bool_new:N \l_@@_index_internal_bool
%    \end{macrocode}
% \end{variable}
%
% \begin{variable}{\g_@@_module_name_tl}
%   The module name, set when reading a line |<@@=|\meta{module}|>|.
%    \begin{macrocode}
\tl_new:N \g_@@_module_name_tl
%    \end{macrocode}
% \end{variable}
%
% \begin{variable}{\c_@@_iow_rule_tl, \c_@@_iow_midrule_tl}
%   $40$~equal signs.
%    \begin{macrocode}
\tl_const:Nn \c_@@_iow_rule_tl
  { ======================================== }
\tl_const:Nn \c_@@_iow_mid_rule_tl
  { -------------------------------------- }
%    \end{macrocode}
% \end{variable}
%
% \begin{variable}
%   {\l_@@_macro_box, \l_@@_macro_index_box, \l_@@_macro_int}
%   A vertical box in which the names given to the macro environment are
%   typeset, a horizontal box in which we store the targets created by
%   indexing commands, and the number of macros so far (including those
%   from surrounding \env{macro} environments).
%    \begin{macrocode}
\box_new:N \l_@@_macro_box
\box_new:N \l_@@_macro_index_box
\int_new:N \l_@@_macro_int
%    \end{macrocode}
% \end{variable}
%
% \begin{variable}
%   {
%     \l_@@_cmd_tl,
%     \l_@@_cmd_index_tl,
%     \l_@@_cmd_module_tl,
%     \l_@@_cmd_noindex_bool,
%     \l_@@_cmd_replace_bool,
%   }
%   Variables used to control the behaviour of \cs{cmd}, \cs{cs} and
%   \cs{tn}.
%    \begin{macrocode}
\tl_new:N \l_@@_cmd_tl
\tl_new:N \l_@@_cmd_index_tl
\tl_new:N \l_@@_cmd_module_tl
\bool_new:N \l_@@_cmd_noindex_bool
\bool_new:N \l_@@_cmd_replace_bool
%    \end{macrocode}
% \end{variable}
%
% \begin{variable}{\l_@@_in_implementation_bool}
%   This boolean is \texttt{true} within the \env{implementation}
%   environment, and \texttt{false} anywhere else.
%    \begin{macrocode}
\bool_new:N \l_@@_in_implementation_bool
%    \end{macrocode}
% \end{variable}
%
% \begin{variable}
%   {
%     \g_@@_typeset_documentation_bool,
%     \g_@@_typeset_implementation_bool
%   }
%   These booleans control whether the documentation/implementation
%   should be typeset.  By default both should be.
%    \begin{macrocode}
\bool_new:N \g_@@_typeset_documentation_bool
\bool_new:N \g_@@_typeset_implementation_bool
\bool_set_true:N \g_@@_typeset_documentation_bool
\bool_set_true:N \g_@@_typeset_implementation_bool
%    \end{macrocode}
% \end{variable}
%
% \begin{variable}{\g_@@_base_name_tl, \l_@@_variants_prop}
%   The name of the macro which is being documented (without its
%   signature), and a property list mapping base forms of variants to
%   all variants which have the same base form.
%    \begin{macrocode}
\tl_new:N \g_@@_base_name_tl
\prop_new:N \l_@@_variants_prop
%    \end{macrocode}
% \end{variable}
%
% \begin{variable}{\l_@@_function_label_clist, \l_@@_no_label_bool}
%   Option of a \env{function} environment which replaces the label that
%   would normally be inserted by labels for the given list of control
%   sequences.  This is only useful to avoid duplicate labels when a
%   function's documentation appears multiple times.
%    \begin{macrocode}
\clist_new:N \l_@@_function_label_clist
\bool_new:N \l_@@_no_label_bool
%    \end{macrocode}
% \end{variable}
%
% \begin{variable}{\l_@@_date_added_tl, \l_@@_date_updated_tl}
%   Values of some options of the \env{function} environment.
%    \begin{macrocode}
\tl_new:N \l_@@_date_added_tl
\tl_new:N \l_@@_date_updated_tl
%    \end{macrocode}
% \end{variable}
%
% \begin{variable}{\l_@@_macro_argument_tl}
%   Save the argument of a \env{macro} or \env{function} environment for
%   use in error messages.
%    \begin{macrocode}
\tl_new:N \l_@@_macro_argument_tl
%    \end{macrocode}
% \end{variable}
%
% ^^A Bruno: what does the next line do?
%    \begin{macrocode}
% \int_new:N \c@CodelineNo
%    \end{macrocode}
%
% \subsection{Variants and helpers}
%
% \begin{macro}{\@@_tmpa:w, \@@_tmpb:w}
%   Auxiliary macros for temporary use.
%    \begin{macrocode}
\cs_new_eq:NN \@@_tmpa:w ?
\cs_new_eq:NN \@@_tmpb:w ?
%    \end{macrocode}
% \end{macro}
%
% \begin{macro}
%   {
%     \seq_set_split:NoV,
%     \seq_gput_right:Nf,
%     \str_case:fn,
%     \tl_count:f,
%     \tl_greplace_all:Nno,
%     \tl_if_head_eq_charcode:oNTF,
%     \tl_if_head_eq_charcode:oNT,
%     \tl_if_head_eq_charcode:oNF,
%     \tl_if_in:ooTF,
%     \tl_if_in:NoTF,
%     \tl_if_in:NoT,
%     \tl_if_in:NoF,
%     \tl_remove_all:Nx,
%     \tl_replace_all:Nxn,
%     \tl_replace_all:Non,
%     \tl_replace_all:Nno,
%     \tl_to_str:f,
%   }
%   A few missing variants.
%    \begin{macrocode}
\cs_generate_variant:Nn \seq_set_split:Nnn { NoV }
\cs_generate_variant:Nn \seq_gput_right:Nn { Nf }
\cs_generate_variant:Nn \str_case:nn { fn }
\cs_generate_variant:Nn \tl_count:n { f }
\cs_generate_variant:Nn \tl_greplace_all:Nnn { Nno }
\cs_generate_variant:Nn \tl_if_empty:nTF { f }
\prg_generate_conditional_variant:Nnn
  \tl_if_head_eq_charcode:nN { o } { T , F , TF }
\prg_generate_conditional_variant:Nnn \tl_if_in:nn { oo } { TF }
\prg_generate_conditional_variant:Nnn \tl_if_in:Nn { No } { T , F , TF }
\cs_generate_variant:Nn \tl_remove_all:Nn   { Nx }
\cs_generate_variant:Nn \tl_replace_all:Nnn { Nx , No , Nno }
\cs_generate_variant:Nn \tl_set_rescan:Nnn { NnV }
\cs_generate_variant:Nn \tl_to_str:n { f }
%    \end{macrocode}
% \end{macro}
%
% \begin{macro}[TF]{\@@_if_almost_str:n}
%   Used to test if the argument of |\cmd| or other macros to be indexed
%   is almost a string or not: for instance this is \texttt{false} if |#1|
%   contains |\meta{...}|.  The surprising |f|-expansion are there to
%   cope with the case of |#1| starting with \cs{c_backslash_str}
%   which should be expanded and considered to be \enquote{normal}.
%    \begin{macrocode}
\prg_new_protected_conditional:Npnn \@@_if_almost_str:n #1 { T , F , TF }
  {
    \int_compare:nNnTF
      { \tl_count:n {#1} }
      < { \tl_count:f { \tl_to_str:f {#1} } }
      { \prg_return_false: }
      { \prg_return_true: }
  }
\cs_generate_variant:Nn \@@_if_almost_str:nT { V }
%    \end{macrocode}
% \end{macro}
%
% \begin{macro}{\@@_trim_right:Nn, \@@_trim_right:No}
%   Removes all material after |#2| in the token list variable~|#1|.
%   Perhaps combine with \cs{@@_key_trim_module:n}?
%    \begin{macrocode}
\cs_new_protected:Npn \@@_trim_right:Nn #1#2
  {
    \cs_set:Npn \@@_tmp:w ##1 #2 ##2 \q_stop { \exp_not:n {##1} }
    \tl_set:Nx #1 { \exp_after:wN \@@_tmp:w #1 #2 \q_stop }
  }
\cs_generate_variant:Nn \@@_trim_right:Nn { No }
%    \end{macrocode}
% \end{macro}
%
% \begin{macro}[TF]{\@@_str_if_begin:nn, \@@_str_if_begin:oo}
%   True if the first string starts with the second.
%    \begin{macrocode}
\prg_new_protected_conditional:Npnn \@@_str_if_begin:nn #1#2 { T , F , TF }
  {
    \tl_if_in:ooTF
      { \exp_after:wN \scan_stop: \tl_to_str:n {#1} }
      { \exp_after:wN \scan_stop: \tl_to_str:n {#2} }
      { \prg_return_true: }
      { \prg_return_false: }
  }
\prg_generate_conditional_variant:Nnn \@@_str_if_begin:nn
  { oo } { T , F , TF }
%    \end{macrocode}
% \end{macro}
%
% \begin{macro}{\@@_replace_at_at:N}
% \begin{macro}{\@@_replace_at_at_aux:Nn}
%   The goal is to replace |@@| by the current module name.  We take
%   advantage of this function to also detect internal macros.  If there is
%   no \meta{module~name}, do nothing.  Otherwise, sanitize the catcodes
%   of |@| and~|_|, temporarily change |@@@@| to |aa| with different catcodes and later to |@@|, and replace |__@@| and |_@@| and |@@| by
%   |__|\meta{module~name}.  The result contains |_| with category
%   code letter because this is what the |macrocode| environment
%   expects.  Other use cases can apply \cs{tl_to_str:n} if needed.
%   Note that we include spaces between the
%   |@| in the code below, since it is also processed through the same
%   replacement rules.
%    \begin{macrocode}
\cs_new_protected:Npn \@@_replace_at_at:N #1
  {
    \tl_if_empty:NF \g_@@_module_name_tl
      {
        \exp_args:NNo \@@_replace_at_at_aux:Nn
          #1 \g_@@_module_name_tl
      }
  }
\cs_new_protected:Npx \@@_replace_at_at_aux:Nn #1#2
  {
    \tl_replace_all:Nnn #1 { \token_to_str:N @ } { @ }
    \tl_replace_all:Nnn #1 { \token_to_str:N _ } { _ }
    \tl_replace_all:Nnn #1 { @ @ @ @ } { \token_to_str:N a a }
    \tl_replace_all:Nnn #1 { _ _ @ @ } { _ _ #2 }
    \tl_replace_all:Nnn #1 {   _ @ @ } { _ _ #2 }
    \tl_replace_all:Nnn #1 {     @ @ } { _ _ #2 }
    \tl_replace_all:Nnn #1 { \token_to_str:N a a } { @ @ }
  }
%    \end{macrocode}
% \end{macro}
% \end{macro}
%
% \begin{macro}
%   {
%     \@@_detect_internals:N,
%     \@@_detect_internals_aux:N,
%     \@@_if_detect_internals_ok:NF
%   }
%   After splitting at each |__| and removing the leading item from the
%   sequence (since it does not follow |__|), remove everything after
%   any space or end-of-line to get a good approximation of the control
%   sequence (for the warning message).  Then check if that starts with
%   something allowed: |@@| module name and |:| or |_|, or if the
%   relevant boolean is set |kernel_| (it seems safe to assume we will
%   not define a |\__kernel:...| command).  For the message itself
%   remove anything after any |_| or |:| (with either catcode) to get a
%   guess of the module name.
%    \begin{macrocode}
\cs_new_protected:Npn \@@_detect_internals:N #1
  {
    \bool_if:NT \l_@@_detect_internals_bool
      { \@@_detect_internals_aux:N #1 }
  }
\group_begin:
  \char_set_catcode_active:N \^^M
  \cs_new_protected:Npn \@@_detect_internals_aux:N #1
    {
      \tl_set_eq:NN \l_@@_detect_internals_tl #1
      \tl_replace_all:Non \l_@@_detect_internals_tl { \token_to_str:N _ } { _ }
      \seq_set_split:NnV \l_@@_tmpa_seq { _ _ } \l_@@_detect_internals_tl
      \seq_pop_left:NN \l_@@_tmpa_seq \l_@@_detect_internals_tl
      \seq_map_variable:NNn \l_@@_tmpa_seq \l_@@_detect_internals_tl
        {
          \@@_trim_right:No \l_@@_detect_internals_tl
            \c_catcode_active_space_tl
          \@@_trim_right:Nn \l_@@_detect_internals_tl ^^M
          \@@_if_detect_internals_ok:NF \l_@@_detect_internals_tl
            {
              \tl_set_eq:NN \l_@@_detect_internals_cs_tl \l_@@_detect_internals_tl
              \@@_trim_right:Nn \l_@@_detect_internals_tl _
              \@@_trim_right:Nn \l_@@_detect_internals_tl :
              \@@_trim_right:No \l_@@_detect_internals_tl { \token_to_str:N : }
              \msg_warning:nnxxx { l3doc } { foreign-internal }
                { \tl_to_str:N \l_@@_detect_internals_cs_tl }
                { \tl_to_str:N \l_@@_detect_internals_tl }
                { \tl_to_str:N \g_@@_module_name_tl }
            }
        }
    }
\group_end:
\prg_new_protected_conditional:Npnn \@@_if_detect_internals_ok:N #1 { F }
  {
    \@@_str_if_begin:ooTF {#1} { \g_@@_module_name_tl _ }
      { \prg_return_true: }
      {
        \@@_str_if_begin:ooTF {#1} { \g_@@_module_name_tl : }
          { \prg_return_true: }
          {
            \bool_if:NTF \g_@@_kernel_bool
              {
                \@@_str_if_begin:ooTF {#1} { kernel _ }
                  { \prg_return_true: }
                  { \prg_return_false: }
              }
              { \prg_return_false: }
          }
      }
  }
%    \end{macrocode}
% \end{macro}
%
% \begin{macro}[rEXP]{\@@_signature_base_form:n}
% \begin{macro}
%   {\@@_signature_base_form_aux:n, \@@_signature_base_form_aux:w}
%   Expands to the \enquote{base form} of the signature.  For instance,
%   given |noxcfvV| it would obtain |nnnNnnn|, or given |ow| it would
%   obtain |nw|.  The loop stops at the first token that is not
%   recognized; the rest is enclosed in \cs{exp_not:n}.
%    \begin{macrocode}
\cs_new:Npn \@@_signature_base_form:n #1
  { \@@_signature_base_form_aux:n #1 \q_stop }
\cs_new:Npn \@@_signature_base_form_aux:n #1
  {
    \str_case:nnTF {#1}
      {
        { N } { N }
        { c } { N }
        { n } { n }
        { o } { n }
        { f } { n }
        { e } { n }
        { x } { n }
        { V } { n }
        { v } { n }
      }
      { \@@_signature_base_form_aux:n }
      { \@@_signature_base_form_aux:w #1 }
  }
\cs_new:Npn \@@_signature_base_form_aux:w #1 \q_stop
  { \exp_not:n {#1} }
%    \end{macrocode}
% \end{macro}
% \end{macro}
%
% \begin{macro}{\@@_predicate_from_base:n}
%   Get predicate from a function's base name.  The code is not broken
%   by functions with no signature.  The |n|-type version can be used
%   for keys and other non-control sequences.  The output after
%   |x|-expansion is a string.
%    \begin{macrocode}
\cs_new:Npn \@@_predicate_from_base:n #1
  {
    \@@_get_function_name:n {#1}
    \tl_to_str:n { _p: }
    \@@_get_function_signature:n {#1}
  }
%    \end{macrocode}
% \end{macro}
%
% \begin{macro}{\@@_split_function_do:nn, \@@_split_function_do:on}
% \begin{macro}{\@@_get_function_name:n, \@@_get_function_signature:n}
% \begin{macro}{\@@_split_function_auxi:w, \@@_split_function_auxii:w}
%   Similar to internal functions defined in \pkg{l3basics}, but here we
%   operate on strings directly rather than control sequences.
%    \begin{macrocode}
\cs_new:Npn \@@_get_function_name:n #1
  { \@@_split_function_do:nn {#1} { \use_i:nnn } }
\cs_new:Npn \@@_get_function_signature:n #1
  { \@@_split_function_do:nn {#1} { \use_ii:nnn } }
\cs_set_protected:Npn \@@_tmpa:w #1
  {
    \cs_new:Npn \@@_split_function_do:nn ##1
      {
        \exp_after:wN \@@_split_function_auxi:w
        \tl_to_str:n {##1} \q_mark \c_true_bool
        #1 \q_mark \c_false_bool
        \q_stop
      }
    \cs_new:Npn \@@_split_function_auxi:w
      ##1 #1 ##2 \q_mark ##3##4 \q_stop ##5
      { \@@_split_function_auxii:w {##5} ##1 \q_mark \q_stop {##2} ##3 }
    \cs_new:Npn \@@_split_function_auxii:w
      ##1##2 \q_mark ##3 \q_stop
      { ##1 {##2} }
  }
\exp_args:No \@@_tmpa:w { \token_to_str:N : }
\cs_generate_variant:Nn \@@_split_function_do:nn { o }
%    \end{macrocode}
% \end{macro}
% \end{macro}
% \end{macro}
%
% \begin{macro}[rEXP]{\@@_key_get_base:nN}
%   Get the base form of a function and store it.  As part of getting
%   the base form, change trailing |T| or |F| to |TF|, skipping that
%   change if the function contains no colon to avoid changing for
%   instance some names ending in \texttt{PDF} or similar.  The various
%   letters |z| serve as end-delimiters different from any outcome of
%   \cs{tl_to_str:n}.
%    \begin{macrocode}
\cs_new_protected:Npn \@@_key_get_base:nN #1#2
  {
    \@@_if_almost_str:nTF {#1}
      {
        \@@_key_get_base_TF:nN {#1} \l_@@_tmpa_tl
        \tl_set:Nx #2
          { \@@_split_function_do:on \l_@@_tmpa_tl { \@@_base_form_aux:nnN } }
      }
      { \tl_set:Nn #2 {#1} }
  }
\cs_new:Npx \@@_key_get_base_TF:nN #1#2
  {
    \tl_set:Nx #2 { \exp_not:N \tl_to_str:n {#1} }
    \tl_if_in:NoF #2 { \tl_to_str:n {:} }
      { \exp_not:N \prg_break: }
    \tl_if_in:onT { #2 z } { \tl_to_str:n {TF} z }
      { \exp_not:N \prg_break: }
    \tl_if_in:onT { #2 z } { \tl_to_str:n {T} z }
      {
        \tl_put_right:Nn #2 { \tl_to_str:n {F} }
        \exp_not:N \prg_break:
      }
    \tl_if_in:onT { #2 z } { \tl_to_str:n {F} z }
      {
        \tl_put_right:Nn #2 { z }
        \tl_replace_once:Nnn #2 { \tl_to_str:n {F} z } { \tl_to_str:n {TF} }
        \exp_not:N \prg_break:
      }
    \exp_not:N \prg_break_point:
  }
\cs_new:Npn \@@_base_form_aux:nnN #1#2#3
  {
    \exp_not:n {#1}
    \bool_if:NT #3
      {
        \token_to_str:N :
        \bool_lazy_or:nnTF
            { \str_if_eq_p:nn { #1 ~ } { \exp_args } }
            { \str_if_eq_p:nn { #1 ~ } { \exp_last_unbraced } }
          { \exp_not:n {#2} }
          { \@@_signature_base_form:n {#2} }
      }
  }
%    \end{macrocode}
% \end{macro}
%
% \begin{macro}{\@@_base_form_signature_do:nnn}
%   Do |#2{#1}| if there is no signature, or if |#1| contains two colons
%   in a row (this covers the weird function |\::N| and so on).
%   Otherwise apply |#3| with the following two arguments: the base form
%   of |#1|, and the original signature with an extra pair of braces.
%    \begin{macrocode}
\cs_new_protected:Npn \@@_base_form_signature_do:nnn #1#2#3
  { % Unused?
    \@@_split_function_do:nn {#1}
      { \@@_base_form_aux:nnnnnN {#1} {#2} {#3} }
  }
\cs_new_protected:Npn \@@_base_form_aux:nnnnnN #1#2#3#4#5#6
  { % Unused?
    \bool_if:NTF #6
      {
        \tl_if_head_eq_charcode:nNTF {#4} :
          { #2 {#1} }
          {
            \use:x
              {
                \exp_not:n {#3}
                { \@@_base_form_aux:nnN {#4} {#5} #6 }
              }
                {#4} {#5}
          }
      }
      { #2 {#1} }
  }
%    \end{macrocode}
% \end{macro}
%
% \begin{macro}[pTF]{\@@_date_compare:nNn}
% \begin{macro}{\@@_date_compare_aux:nnnNnnn, \@@_date_compare_aux:w}
%   Expects |#1| and |#3| to be dates in the format YYYY-MM-DD (but
%   accepts YYYY or YYYY-MM too).  Compares them using |#2| (one of |<|,
%   |=|, |>|), filling in zeros for missing data.
%    \begin{macrocode}
\prg_new_conditional:Npnn \@@_date_compare:nNn #1#2#3 { p , T , F , TF }
  { \@@_date_compare_aux:w #1--- \q_mark #2 #3--- \q_stop }
\cs_new:Npn \@@_date_compare_aux:w
    #1 - #2 - #3 - #4 \q_mark #5 #6 - #7 - #8 - #9 \q_stop
  {
    \@@_date_compare_aux:nnnNnnn
      { \tl_if_empty:nTF {#1} { 0 } {#1} }
      { \tl_if_empty:nTF {#2} { 0 } {#2} }
      { \tl_if_empty:nTF {#3} { 0 } {#3} }
      #5
      { \tl_if_empty:nTF {#6} { 0 } {#6} }
      { \tl_if_empty:nTF {#7} { 0 } {#7} }
      { \tl_if_empty:nTF {#8} { 0 } {#8} }
  }
\cs_new:Npn \@@_date_compare_aux:nnnNnnn #1#2#3#4#5#6#7
  {
    \int_compare:nNnTF {#1} = {#5}
      {
        \int_compare:nNnTF {#2} = {#6}
          {
            \int_compare:nNnTF {#3} #4 {#7}
              { \prg_return_true: } { \prg_return_false: }
          }
          {
            \int_compare:nNnTF {#2} #4 {#6}
              { \prg_return_true: } { \prg_return_false: }
          }
      }
      {
        \int_compare:nNnTF {#1} #4 {#5}
          { \prg_return_true: } { \prg_return_false: }
      }
    \use_none:n
    \q_stop
  }
%    \end{macrocode}
% \end{macro}
% \end{macro}
%
% \begin{macro}{\@@_gprop_name:n, \@@_lseq_name:n}
%   We need to keep track of some information about control sequences
%   (and other strings) that are being (or have been) documented.  Some
%   is stored into global props and some into local seqs, whose name
%   does not follow conventions: it is \cs[no-index]{g_@@} or
%   \cs[no-index]{l_@@} followed by a space and by the string, which can
%   be arbitrary.  We cannot reasonably use a single big |prop| for
%   speed reasons.
%    \begin{macrocode}
\cs_new:Npn \@@_gprop_name:n #1 { g_@@ ~ \tl_to_str:n {#1} }
\cs_new:Npn \@@_lseq_name:n #1 { l_@@ ~ \tl_to_str:n {#1} }
%    \end{macrocode}
% \end{macro}
%
% \subsection{Messages}
%
%    \begin{macrocode}
\msg_new:nnnn { l3doc } { no-signature-TF }
  { Function/macro~'#1'~cannot~be~turned~into~a~conditional. }
  {
    A~function~or~macro~environment~with~option~pTF,~TF~or~noTF~
    received~the~argument~'#1'.~This~function's~name~has~no~
    ':'~hence~it~is~not~clear~where~to~add~'_p'~or~'TF'.~
    Please~follow~expl3~naming~conventions.
  }
\msg_new:nnn { l3doc } { deprecated-function }
  { The~deprecated~function(s)~'#1'~should~have~been~removed~on~#2. }
\msg_new:nnn { l3doc } { date-format }
  { The~date~'#1'~should~be~given~in~YYYY-MM-DD~format. }
\msg_new:nnn { l3doc } { future-date }
  { The~added/updated~date~'#2'~of~'#1'~is~in~the~future. }
\msg_new:nnn { l3doc } { syntax-nested-function }
  {
    The~'syntax'~environment~should~be~used~in~the~
    innermost~'function'~environment.
  }
\msg_new:nnn { l3doc } { multiple-syntax }
  {
    The~'syntax'~environment~should~only~be~used~once~in~
    a~'function'~environment.
  }
\msg_new:nnn { l3doc } { deprecated-option }
  { The~option~'#1'~has~been~deprecated~for~'#2'. }
\msg_new:nnn { l3doc } { foreign-internal }
  {
    A~control~sequence~of~the~form~'...__#1'~was~used.~
    It~should~only~be~used~in~the~module~'#2'
    \tl_if_empty:nF {#3} { ,~not~in~'#3' } .
  }
%    \end{macrocode}
%
% \subsection{Class options and configuration}
%
%    \begin{macrocode}
\DeclareOption { full }
  {
    \bool_gset_true:N \g_@@_typeset_documentation_bool
    \bool_gset_true:N \g_@@_typeset_implementation_bool
  }
\DeclareOption { onlydoc }
  {
    \bool_gset_true:N \g_@@_typeset_documentation_bool
    \bool_gset_false:N \g_@@_typeset_implementation_bool
  }
%    \end{macrocode}
%
%    \begin{macrocode}
\DeclareOption { check }
  { \bool_gset_true:N \g_@@_checkfunc_bool }
\DeclareOption { nocheck }
  { \bool_gset_false:N \g_@@_checkfunc_bool }
\DeclareOption { checktest }
  { \bool_gset_true:N \g_@@_checktest_bool }
\DeclareOption { nochecktest }
  { \bool_gset_false:N \g_@@_checktest_bool }
%    \end{macrocode}
%
%    \begin{macrocode}
\DeclareOption { kernel }
  { \bool_gset_true:N \g_@@_kernel_bool }
\DeclareOption { stdmodule }
  { \bool_gset_false:N \g_@@_kernel_bool }
\DeclareOption { cs-break-off }
  { \bool_gset_false:N \g_@@_cs_break_bool }
\DeclareOption { cs-break-nohyphen }
  { \PassOptionsToPackage { nohyphen } { underscore } }
%    \end{macrocode}
%
%    \begin{macrocode}
\DeclareOption { show-notes }
  { \bool_gset_true:N  \g_@@_show_notes_bool }
\DeclareOption { hide-notes }
  { \bool_gset_false:N \g_@@_show_notes_bool }
%    \end{macrocode}
%
%    \begin{macrocode}
\ExecuteOptions { kernel , nocheck , nochecktest }
\ProcessOptions
%    \end{macrocode}
%
% \subsection{Class and package loading}
%
% The \pkg{doc} package redefines \cs{maketitle} so that it can be used
% more than once to produce a single document with multiple titles. This
% redefinition relies on the existence of \cs{@maketitle}, which often
% is the case, but in a few circumstances the class defines a
% self-contained \cs{maketitle}, and then \pkg{doc}'s redefinition
% breaks. Here we nullify \pkg{doc}'s redefinition to maintain the
% class' definition of \cs{maketitle}. However this doesn't respect
% \pkg{doc}'s changes to \cs{maketitle}\dots{} This probably will
% require a case-by-case patching of \cs{maketitle}.
%    \begin{macrocode}
\cs_new_eq:NN \@@_saved_maketitle: \maketitle
\RequirePackage{doc}
\cs_gset_eq:NN \maketitle \@@_saved_maketitle:
%    \end{macrocode}
%
%    \begin{macrocode}
\RequirePackage
  {
    array,% Okay
    alphalph,% Used in \DocInclude to redefine \thepart
    amsmath,% Okay
    amssymb,% Okay
    booktabs,% Used \...rule, used in syntax, function, and variable environments
    color, % Okay
    colortbl, % \arrayrulecolor, used in syntax environment
    hologo, % Used for logos :-)
    enumitem, % Used for the arguments environment
    pifont, % hollow star for rEXP
    textcomp, % Okay
    trace, % Okay?
    csquotes, % Okay
    fancyvrb, % Used for typesetting verbatim
    underscore, % Breakable underscores (expl3 has some :-)
    verbatim, % Okay
  }
%    \end{macrocode}
%
% Must be last, as usual.
%    \begin{macrocode}
\RequirePackage{hypdoc}
%    \end{macrocode}
%
% \subsection{Configuration and tweaks}
%
% \begin{macro}{\MakePrivateLetters}
%   A few more letters are \enquote{private} in a \LaTeX3 programming
%   environment.
%    \begin{macrocode}
\cs_gset:Npn \MakePrivateLetters
  {
    \char_set_catcode_letter:N \@
    \char_set_catcode_letter:N \_
    \char_set_catcode_letter:N \:
  }
%    \end{macrocode}
% \end{macro}
%
% \begin{macro}{CodelineNo}
%   Some configurations which have to do with line numbering.
%    \begin{macrocode}
\setcounter{StandardModuleDepth}{1}
\@addtoreset{CodelineNo}{part}
\tl_replace_once:Nnn \theCodelineNo
  { \HDorg@theCodelineNo }
  { \textcolor[gray]{0.5} { \sffamily\tiny\arabic{CodelineNo} } }
%    \end{macrocode}
% \end{macro}
%
% \begin{macro}{\verbatim, \endverbatim}
%   In \file{.dtx} documents, the \env{verbatim} environment adds extra
%   space because it only removes the first \enquote{\%} sign, and not
%   the indentation (typically a space).  Fix it with \pkg{fancyvrb}:
%    \begin{macrocode}
\fvset{gobble=2}
\cs_gset_eq:NN \verbatim \Verbatim
\cs_gset_eq:NN \endverbatim \endVerbatim
%    \end{macrocode}
% \end{macro}
%
% \begin{macro}{\ifnot@excluded}
%   This function tests whether a macro name stored in
%   \tn{macro@namepart} was excluded from indexing by \tn{DoNotIndex}.
%   Rather than trying to fix catcodes that come into here, turn
%   everything to string catcodes.  This is somewhat inefficient as we
%   could have ensured that \tn{index@excludelist} has string catcodes
%   in the first place.
%    \begin{macrocode}
\cs_set_protected:Npn \ifnot@excluded
  {
    \exp_args:Nxx \expanded@notin
      { \c_backslash_str \tl_to_str:N \macro@namepart , }
      { \exp_args:NV \tl_to_str:n \index@excludelist }
  }
%    \end{macrocode}
% \end{macro}
%
% \begin{macro}{\pdfstringnewline}
% \begin{macro}{\@@_pdfstring_newline:w}
%   We avoid some hyperref warnings by making |\\| (almost) trivial in
%   bookmarks: more precisely it might be used with a star and an
%   optional argument, which we thus remove using an \pkg{xparse}
%   expandable command.  Since there cannot be trailing optional
%   arguments, pick up an extra mandatory one and put it back.
%    \begin{macrocode}
\cs_new:Npn \pdfstringnewline { : ~ }
\DeclareExpandableDocumentCommand
  { \@@_pdfstring_newline:w } { s o m } { \pdfstringnewline #3 }
\pdfstringdefDisableCommands
  { \cs_set_eq:NN \\ \@@_pdfstring_newline:w }
%    \end{macrocode}
% \end{macro}
% \end{macro}
%
% \subsection{Text markup}
%
% Make "|" and |"| be \enquote{short verb} characters, but not in the
% document preamble, where an active character may interfere with
% packages that are loaded.  Remove these short-hands at the end of the
% document before reading the \file{.aux} file, as they may appear in
% labels (for instance, \pkg{l3fp} documents an operation "||").
%    \begin{macrocode}
\AtBeginDocument
  {
    \MakeShortVerb \"
    \MakeShortVerb \|
  }
\AtEndDocument
  {
    \DeleteShortVerb \"
    \DeleteShortVerb \|
  }
%    \end{macrocode}
%
% \begin{macro}{\eTeX, \IniTeX, \Lua, \LuaTeX, \pdfTeX, \XeTeX,
%   \pTeX, \upTeX, \epTeX, \eupTeX}
%   Some commands for logos.
%    \begin{macrocode}
\providecommand*\eTeX{\hologo{eTeX}}
\providecommand*\IniTeX{\hologo{iniTeX}}
\providecommand*\Lua{Lua}
\providecommand*\LuaTeX{\hologo{LuaTeX}}
\providecommand*\pdfTeX{\hologo{pdfTeX}}
\providecommand*\XeTeX{\hologo{XeTeX}}
\providecommand*\pTeX{p\kern-.2em\hologo{TeX}}
\providecommand*\upTeX{up\kern-.2em\hologo{TeX}}
\providecommand*\epTeX{$\varepsilon$-\pTeX}
\providecommand*\eupTeX{$\varepsilon$-\upTeX}
\providecommand*\ConTeXt{\hologo{ConTeXt}}
%    \end{macrocode}
% \end{macro}
%
% \begin{macro}{\cmd, \cs, \tn}
%   They rely on a common auxiliary \cs{@@_cmd:nn} which receives as
%   arguments the options and some tokens whose string representation
%   starts with a backslash (to support cases such as |\cs{pkg_\ldots}|,
%   we do not turn the whole argument into a string).
%    \begin{macrocode}
\DeclareDocumentCommand \cmd { O{} m }
  { \@@_cmd:no {#1} { \token_to_str:N #2 } }
\DeclareDocumentCommand \cs  { O{} m }
  { \@@_cmd:no {#1} { \c_backslash_str #2 } }
\DeclareDocumentCommand \tn  { O{} m }
  {
    \@@_cmd:no
      { module = TeX , replace = false , #1 }
      { \c_backslash_str #2 }
  }
%    \end{macrocode}
% \end{macro}
%
% \begin{macro}{\meta}
%   A document-level command.
%    \begin{macrocode}
\DeclareDocumentCommand \meta { m }
  { \@@_meta:n {#1} }
%    \end{macrocode}
% \end{macro}
%
% \begin{macro}
%   {
%     \@@_pdfstring_cmd:w,
%     \@@_pdfstring_cs:w,
%     \@@_pdfstring_meta:w
%   }
%   To work within a bookmark, these commands must be expandable.
%    \begin{macrocode}
\DeclareExpandableDocumentCommand
  { \@@_pdfstring_cmd:w } { o m } { \token_to_str:N #2 }
\DeclareExpandableDocumentCommand
  { \@@_pdfstring_cs:w }  { o m } { \textbackslash \tl_to_str:n {#2} }
\cs_new:Npn \@@_pdfstring_meta:w #1
  { < \tl_to_str:n {#1} > }
\pdfstringdefDisableCommands
  {
    \cs_set_eq:NN \cmd  \@@_pdfstring_cmd:w
    \cs_set_eq:NN \cs   \@@_pdfstring_cs:w
    \cs_set_eq:NN \tn   \@@_pdfstring_cs:w
    \cs_set_eq:NN \meta \@@_pdfstring_meta:w
  }
%    \end{macrocode}
% \end{macro}
%
% \begin{macro}{\Arg, \marg, \oarg, \parg}
%   |\marg{text}| prints \marg{text}, \enquote{mandatory argument}.\\
%   |\oarg{text}| prints \oarg{text}, \enquote{optional argument}.\\
%   |\parg{te,xt}| prints \parg{te,xt}, \enquote{picture mode argument}.
%   Finally, \cs{Arg} is the same as \cs{marg}.
%    \begin{macrocode}
\newcommand\Arg[1]
  { \texttt{\char`\{} \meta{#1} \texttt{\char`\}} }
\providecommand\marg[1]{ \Arg{#1} }
\providecommand\oarg[1]{ \texttt[ \meta{#1} \texttt] }
\providecommand\parg[1]{ \texttt( \meta{#1} \texttt) }
%    \end{macrocode}
% \end{macro}
%
% \begin{macro}{\file, \env, \pkg, \cls}
%   This list may change\dots this is just my preference for markup.
%    \begin{macrocode}
\DeclareRobustCommand \file {\nolinkurl}
\DeclareRobustCommand \env {\texttt}
\DeclareRobustCommand \pkg {\textsf}
\DeclareRobustCommand \cls {\textsf}
%    \end{macrocode}
% \end{macro}
%
% \begin{macro}{\EnableDocumentation, \EnableImplementation}
% \begin{macro}{\DisableDocumentation, \DisableImplementation}
%   Control whether to typeset the documentation/implementation or not.
%   These simply set two switches.
%    \begin{macrocode}
\NewDocumentCommand \EnableDocumentation { }
  { \bool_gset_true:N \g_@@_typeset_documentation_bool }
\NewDocumentCommand \EnableImplementation { }
  { \bool_gset_true:N \g_@@_typeset_implementation_bool }
\NewDocumentCommand \DisableDocumentation { }
  { \bool_gset_false:N \g_@@_typeset_documentation_bool }
\NewDocumentCommand \DisableImplementation { }
  { \bool_gset_false:N \g_@@_typeset_implementation_bool }
%    \end{macrocode}
% \end{macro}
% \end{macro}
%
% \begin{environment}{documentation}
% \begin{environment}{implementation}
%   If the documentation/implementation should be typeset, then simply
%   set the boolean \cs{l_@@_in_implementation_bool} which indicates
%   whether we are within the implementation section.  Otherwise use
%   \cs{comment} (and a paired \cs{endcomment}).
%    \begin{macrocode}
\NewDocumentEnvironment { documentation } { }
  {
    \bool_if:NTF \g_@@_typeset_documentation_bool
      { \bool_set_false:N \l_@@_in_implementation_bool }
      { \comment }
  }
  { \bool_if:NF \g_@@_typeset_documentation_bool { \endcomment } }
\NewDocumentEnvironment { implementation } { }
  {
    \bool_if:NTF \g_@@_typeset_implementation_bool
      { \bool_set_true:N \l_@@_in_implementation_bool }
      { \comment }
  }
  { \bool_if:NF \g_@@_typeset_implementation_bool { \endcomment } }
%    \end{macrocode}
% \end{environment}
% \end{environment}
%
% \begin{environment}{variable}
%   The \env{variable} environment behaves as a \env{function} or
%   \env{macro} environment depending on the part of the document.
%    \begin{macrocode}
\DeclareDocumentEnvironment { variable } { O{} +v }
  {
    \bool_if:NTF \l_@@_in_implementation_bool
      { \@@_macro:nnw { var , #1 } {#2} }
      { \@@_function:nnw {#1} {#2} }
  }
  {
    \bool_if:NTF \l_@@_in_implementation_bool
      { \@@_macro_end: }
      { \@@_function_end: }
  }
%    \end{macrocode}
% \end{environment}
%
% \begin{environment}{function}
% \begin{environment}{macro}
%   Environment for documenting function(s), and environment for
%   documenting the implementation of a macro.
%    \begin{macrocode}
\DeclareDocumentEnvironment { function } { O{} +v }
  { \@@_function:nnw {#1} {#2} }
  { \@@_function_end: }
\DeclareDocumentEnvironment { macro } { O{} +v }
  { \@@_macro:nnw {#1} {#2} }
  { \@@_macro_end: }
%    \end{macrocode}
% \end{environment}
% \end{environment}
%
% \begin{environment}{syntax}
%   Syntax block placed next to the list of functions to illustrate
%   their use.  TODO: test that the \env{syntax} environment is only
%   used inside the \env{function} environment, and that it only appears
%   once.
%    \begin{macrocode}
\NewDocumentEnvironment { syntax } { }
  { \@@_syntax:w }
  {
    \@@_syntax_end:
    \ignorespacesafterend
  }
%    \end{macrocode}
% \end{environment}
%
% \begin{environment}{texnote}
%   Used to describe information destined to \TeX{} experts only.
%    \begin{macrocode}
\NewDocumentEnvironment { texnote } { }
  {
    \endgraf
    \vspace{3mm}
    \small\textbf{\TeX~hackers~note:}
  }
  {
    \vspace{3mm}
  }
%    \end{macrocode}
% \end{environment}
%
% \begin{environment}{arguments}
%   This environment is designed to be used within a \env{macro}
%   environment to describe the arguments of the macro/function.
%    \begin{macrocode}
\NewDocumentEnvironment { arguments } { }
  {
    \enumerate [
      nolistsep ,
      label = \texttt{\#\arabic*} ~ : ,
      labelsep = * ,
    ]
  }
  {
    \endenumerate
  }
%    \end{macrocode}
% \end{environment}
%
% \begin{macro}{\CodedocExplain, \CodedocExplainEXP, \CodedocExplainREXP, \CodedocExplainTF}
%   Explanation of stars and |TF| notations, for use in third-party
%   packages.
%    \begin{macrocode}
\NewDocumentCommand { \CodedocExplain } { }
  { \CodedocExplainEXP \ \CodedocExplainREXP \ \CodedocExplainTF }
\NewDocumentCommand { \CodedocExplainEXP } { }
  {
    \raisebox{\baselineskip}[0pt][0pt]{\hypertarget{expstar}{}}%
    \write \@auxout { \def \string \Codedoc@expstar { } }
    \@@_typeset_exp:\ indicates~fully~expandable~functions,~which~
    can~be~used~within~an~\texttt{x}-type~argument~(in~plain~
    \TeX{}~terms,~inside~an~\cs{edef}),~as~well~as~within~an~
    \texttt{f}-type~argument.
  }
\NewDocumentCommand { \CodedocExplainREXP } { }
  {
    \raisebox{\baselineskip}[0pt][0pt]{\hypertarget{rexpstar}{}}%
    \write \@auxout { \def \string \Codedoc@rexpstar { } }
    \@@_typeset_rexp:\ indicates~
    restricted~expandable~functions,~which~can~be~used~within~an~
    \texttt{x}-type~argument~but~cannot~be~fully~expanded~within~an~
    \texttt{f}-type~argument.
  }
\NewDocumentCommand { \CodedocExplainTF } { }
  {
    \raisebox{\baselineskip}[0pt][0pt]{\hypertarget{explTF}{}}%
    \write \@auxout { \def \string \Codedoc@explTF { } }
    \@@_typeset_TF:\ indicates~conditional~(\texttt{if})~functions~
    whose~variants~with~\texttt{T},~\texttt{F}~and~\texttt{TF}~
    argument~specifiers~expect~different~
    \enquote{true}/\enquote{false}~branches.
  }
%    \end{macrocode}
% \end{macro}
%
% \subsection{Implementing text markup}
%
% Keys for \cs{cmd}, \cs{cs} and \cs{tn}.
%    \begin{macrocode}
\keys_define:nn { l3doc/cmd }
  {
    index     .tl_set:N     = \l_@@_cmd_index_tl        ,
    module    .tl_set:N     = \l_@@_cmd_module_tl       ,
    no-index  .bool_set:N   = \l_@@_cmd_noindex_bool    ,
    replace   .bool_set:N   = \l_@@_cmd_replace_bool    ,
  }
%    \end{macrocode}
%
% \begin{macro}[do-not-index={\\,\_,\1,\c,\2}]{\@@_cmd:nn, \@@_cmd:no}
%   Apply the key--value \meta{options}~|#1| after setting some
%   default values.  Then (unless |replace=false|) replace |@@| in~|#2|,
%   which is a bit tricky: the |_| must be given the catcode expected by
%   \cs{@@_replace_at_at:N}, but should be reverted to their original
%   catcode (normally active, needed for line-breaking) without
%   rescanning the whole argument.  Then typeset the command in
%   \tn{verbatim@font}, after turning it to harmless characters if
%   needed (and keeping the underscore breakable); in any case, spaces
%   must be turned into \tn{@xobeysp} and we must use \tn{@} to avoid
%   longer spaces after a control sequence that ends for instance with a
%   colon (empty signature).  Finally, produce an index entry.
%   Indexing is suppressed when \cs{l_@@_cmd_noindex_bool} is true.
%    \begin{macrocode}
\cs_new_protected:Npn \@@_cmd:nn #1#2
  {
    \bool_set_false:N \l_@@_cmd_noindex_bool
    \bool_set_true:N \l_@@_cmd_replace_bool
    \tl_set:Nn \l_@@_cmd_index_tl { \q_no_value }
    \tl_set:Nn \l_@@_cmd_module_tl { \q_no_value }
    \keys_set:nn { l3doc/cmd } {#1}
    \tl_set:Nn \l_@@_cmd_tl {#2}
    \bool_if:NT \l_@@_cmd_replace_bool
      {
        \tl_set_rescan:Nnn \l_@@_tmpb_tl { } { _ }
        \tl_replace_all:Non \l_@@_cmd_tl \l_@@_tmpb_tl { _ }
        \@@_replace_at_at:N \l_@@_cmd_tl
        \tl_replace_all:Nno \l_@@_cmd_tl { _ } \l_@@_tmpb_tl
      }
%    \end{macrocode}
% \paragraph{Typesetting}
% Note the replacement for the underscore is to permit linebreaks.
% The \texttt{underscore} package adds the linebreak,
% and the regex results in applying the breakable underscore only to the \emph{last}
% of a run of underscores, and not if the underscore follows a backslash.
%    \begin{macrocode}
    \mode_if_math:T { \mbox }
      {
        \verbatim@font
        \@@_if_almost_str:VT \l_@@_cmd_tl
          {
            \tl_set:Nx \l_@@_cmd_tl { \tl_to_str:N \l_@@_cmd_tl }
            \bool_if:NT \g_@@_cs_break_bool
              {
                \regex_replace_all:nnN
                  {([^\\])_([^\_])}
                  {\1\c{BreakableUnderscore}\2}
                  \l_@@_cmd_tl
              }
          }
        \tl_replace_all:Nnn \l_@@_cmd_tl { ~ } { \@xobeysp }
        \l_@@_cmd_tl
        \@
      }
%    \end{macrocode}
% \paragraph{Indexing}
%    \begin{macrocode}
    \bool_if:NF \l_@@_cmd_noindex_bool
      {
        \quark_if_no_value:NF \l_@@_cmd_index_tl
          {
            \tl_set:Nx \l_@@_cmd_tl
              { \c_backslash_str \exp_not:o { \l_@@_cmd_index_tl } }
          }

        \exp_args:No \@@_key_get:n { \l_@@_cmd_tl }
        \quark_if_no_value:NF \l_@@_cmd_module_tl
          {
            \tl_set:Nx \l_@@_index_module_tl
              { \tl_to_str:N \l_@@_cmd_module_tl }
          }
        \@@_special_index_module:ooonN
          { \l_@@_index_key_tl }
          { \l_@@_index_macro_tl }
          { \l_@@_index_module_tl }
          { usage }
          \l_@@_index_internal_bool
      }
  }
\cs_generate_variant:Nn \@@_cmd:nn { no }
%    \end{macrocode}
% \end{macro}
%
% \begin{macro}
%   {
%     \@@_meta:n,
%     \@@_ensuremath_sb:n,
%     \@@_meta_original:n
%   }
%   Store |#1| in \cs{l_@@_tmpa_tl} and replaces every underscore,
%   regardless of its category (\enquote{math toggle},
%   \enquote{alignment}, \enquote{superscript}, \enquote{subscript},
%   \enquote{letter}, \enquote{other}, or \enquote{active}) by
%   \cs{@@_ensuremath_sb:n} (which creates math subscripts), then runs
%   the code used for \tn{meta} in \pkg{doc.sty}.
%    \begin{macrocode}
\cs_new_protected:Npn \@@_meta:n #1
  {
    \tl_set:Nn \l_@@_tmpa_tl {#1}
    \tl_map_inline:nn
      { { 3 } { 4 } { 7 } { 8 } { 11 } { 12 } { 13 } }
      {
        \tl_set_rescan:Nnn \l_@@_tmpb_tl
          { \char_set_catcode:nn { `_ } {##1} } { _ }
        \tl_replace_all:Non \l_@@_tmpa_tl \l_@@_tmpb_tl
          { \@@_ensuremath_sb:n }
      }
    \exp_args:NV \@@_meta_original:n \l_@@_tmpa_tl
  }
\cs_new_protected:Npn \@@_ensuremath_sb:n #1
  { \ensuremath { \sb {#1} } }
\cs_new_protected:Npn \@@_meta_original:n #1
  {
    \ensuremath \langle
    \mode_if_math:T { \nfss@text }
    {
      \meta@font@select
      \edef \meta@hyphen@restore
        { \hyphenchar \the \font \the \hyphenchar \font }
      \hyphenchar \font \m@ne
      \language \l@nohyphenation
      #1 \/
      \meta@hyphen@restore
    }
    \ensuremath \rangle
  }
%    \end{macrocode}
% \end{macro}
%
% \subsubsection{Common between \env{macro} and \env{function}}
%
% \begin{macro}
%   {
%     \@@_typeset_exp:, \@@_typeset_rexp:,
%     \@@_typeset_TF:, \@@_typeset_aux:n
%   }
%   Used by \cs{@@_macro_single:nNN} and in the \env{function} environment
%   to typeset conditionals and auxiliary functions.
%    \begin{macrocode}
\cs_new_protected:Npn \@@_typeset_exp:
  {
    \cs_if_exist:NTF \Codedoc@expstar
      { \hyperlink { expstar } }
      { \mbox }
    {$\star$}
  }
\cs_new_protected:Npn \@@_typeset_rexp:
  {
    \cs_if_exist:NTF \Codedoc@rexpstar
      { \hyperlink { rexpstar } }
      { \mbox }
    { \ding { 73 } } % hollow star
  }
\cs_new_protected:Npn \@@_typeset_TF:
  {
    \cs_if_exist:NTF \Codedoc@explTF
      { \hyperlink { explTF } }
      { \mbox }
      {
        \color{black}
        \itshape TF
        \makebox[0pt][r]
          {
            \cs_if_exist:NT \Codedoc@explTF { \color{red} }
            \underline { \phantom{\itshape TF} \kern-0.1em }
          }
      }
  }
\cs_new_protected:Npn \@@_typeset_aux:n #1
  {
    { \color[gray]{0.5} #1 }
  }
%    \end{macrocode}
% \end{macro}
%
% \begin{macro}
%   {\@@_get_hyper_target:nN, \@@_get_hyper_target:oN, \@@_get_hyper_target:xN}
%   Create a \pkg{hyperref} anchor from a macro name~|#1| and stores it
%   in the token list variable~|#2|.  For instance, |\prg_replicate:nn|
%   gives |doc/function//prg/replicate:nn|.
%    \begin{macrocode}
\cs_new_protected:Npn \@@_get_hyper_target:nN #1#2
  {
    \tl_set:Nx #2 { \tl_to_str:n {#1} }
    \tl_replace_all:Nxn #2 { \c_underscore_str } { / }
    \tl_remove_all:Nx   #2 { \c_backslash_str }
    \tl_put_left:Nn #2 { doc/function// }
  }
\cs_generate_variant:Nn \@@_get_hyper_target:nN { o , x }
%    \end{macrocode}
% \end{macro}
%
% \begin{macro}{\@@_names_get_seq:nN}
%   The argument~|#1| (argument of a |function| or |macro| environment)
%   has catcodes $10$ (space), $12$ (other) and $13$ (active).  Sanitize
%   catcodes.  If the |verb| option was used, output a one-item
%   sequence.  Otherwise, remove any \enquote{\%} character at the
%   beginning of a line.  Remove tabs and newlines.  Finally, convert
%   |_@@| and |@@| to |__|\meta{module name} (if it is non-empty).  At
%   this point, \cs{l_@@_tmpa_tl} contains a comma-delimited list of
%   names, where |@| and~|_| have category code letter.  Turn it to a
%   string, parse it as a comma-delimited list (in particular this
%   removes spaces), and output a sequence of function/macro names.
%    \begin{macrocode}
\cs_new_protected:Npn \@@_names_get_seq:nN #1#2
  {
    \tl_set:Nx \l_@@_tmpa_tl { \tl_to_str:n {#1} }
    \bool_if:NTF \l_@@_names_verb_bool
      {
        \seq_clear:N #2
        \seq_put_right:NV #2 \l_@@_tmpa_tl
      }
      {
        \tl_remove_all:Nx \l_@@_tmpa_tl
          { \iow_char:N \^^M \c_percent_str }
        \tl_remove_all:Nx \l_@@_tmpa_tl { \tl_to_str:n { ^ ^ A } }
        \tl_remove_all:Nx \l_@@_tmpa_tl { \iow_char:N \^^I }
        \tl_remove_all:Nx \l_@@_tmpa_tl { \iow_char:N \^^M }
        \@@_detect_internals:N \l_@@_tmpa_tl
        \@@_replace_at_at:N \l_@@_tmpa_tl
        \exp_args:NNx \seq_set_from_clist:Nn #2
          { \tl_to_str:N \l_@@_tmpa_tl }
      }
  }
%    \end{macrocode}
% \end{macro}
%
% \begin{macro}{\@@_names_parse:, \@@_names_parse_one:n}
%   The goal is to group variants together.  We populate
%   \cs{l_@@_names_block_tl} with local sequence variable named with
%   \cs{@@_lseq_name:n} after the base forms.  When encountering a new
%   base form, set the corresponding local sequence to hold the
%   \meta{base name} (stripped of the signature) and add the local
%   sequence to the list \cs{l_@@_names_block_tl}.  In all cases append
%   the signature to the local sequence, which thus takes the form
%   \meta{base name}, \meta{signature_1}, \meta{signature_2} and so on.
%   If the original function had no signature (no colon) then use
%   \cs{scan_stop:} as the signature (there can be no variant).  We
%   special case commands |#1| starting with |\::|, namely weird
%   functions named |\::N| and the like.
%    \begin{macrocode}
\cs_new_protected:Npn \@@_names_parse:
  {
    \tl_clear:N \l_@@_names_block_tl
    \seq_map_function:NN
      \l_@@_names_seq
      \@@_names_parse_one:n
  }
\cs_new_protected:Npn \@@_names_parse_one:n #1
  {
    \@@_split_function_do:nn {#1}
      { \@@_names_parse_one_aux:nnNn }
    {#1}
  }
\cs_new_protected:Npn \@@_names_parse_one_aux:nnNn #1#2#3#4
  {
    \bool_if:NTF #3
      {
        \tl_if_head_eq_charcode:nNTF {#2} :
          { \@@_names_parse_aux:nnn {#4} {#4} { \scan_stop: } }
          {
            \exp_args:Nx \@@_names_parse_aux:nnn
              { \@@_base_form_aux:nnN {#1} {#2} #3 }
              {#1} {#2}
          }
      }
      {
        \bool_if:NT \l_@@_macro_TF_bool
          { \msg_error:nnx { l3doc } { no-signature-TF } {#4} }
        \@@_names_parse_aux:nnn {#4} {#4} { \scan_stop: }
      }
  }
\cs_new_protected:Npn \@@_names_parse_aux:nnn #1
  { \exp_args:Nc \@@_names_parse_aux:Nnn { \@@_lseq_name:n {#1} } }
\cs_new_protected:Npn \@@_names_parse_aux:Nnn #1#2#3
  {
    \tl_if_in:NnF \l_@@_names_block_tl {#1}
      {
        \tl_put_right:Nn \l_@@_names_block_tl {#1}
        \seq_clear_new:N #1
        \seq_put_right:Nn #1 {#2}
      }
    \seq_put_right:Nn #1 {#3}
  }
%    \end{macrocode}
% \end{macro}
%
% \begin{macro}{\@@_names_typeset:}
% \begin{macro}{\@@_names_typeset_auxi:n}
%   This code is in particular used when typesetting function names in a
%   \env{function} environment.  The mapping to \cs{l_@@_names_block_tl}
%   cannot use \cs{tl_map_inline:Nn} because the code following |\\|
%   would not be expandable, thus breaking \tn{bottomrule}.
%
%   Call \cs{@@_names_typeset_auxi:n} on each local sequence (which
%   holds a set of variants).  The first step is to pop the base form
%   and change spaces to category other so
%   that they get displayed eventually.  Then store the variants in
%   \cs{g_@@_variants_seq}, remove the first, which will be displayed
%   more prominently, and reconstruct the actual name, passing it to
%   \cs{@@_names_typeset_auxii:n}.
%    \begin{macrocode}
\cs_new_protected:Npn \@@_names_typeset:
  {
    \tl_map_function:NN \l_@@_names_block_tl
      \@@_names_typeset_auxi:n
  }
\cs_new_protected:Npn \@@_names_typeset_auxi:n #1
  {
    \seq_pop:NN #1 \l_@@_tmpa_tl
    \tl_gset_eq:NN \g_@@_base_name_tl \l_@@_tmpa_tl
    \tl_greplace_all:Nno \g_@@_base_name_tl
      { ~ } { \c_catcode_other_space_tl }
    \seq_get:NN #1 \l_@@_tmpa_tl
    \str_if_eq:VnTF \l_@@_tmpa_tl { \scan_stop: }
      {
        \seq_gclear:N \g_@@_variants_seq
        \@@_names_typeset_auxii:x { \g_@@_base_name_tl }
      }
      {
        \seq_gset_eq:NN \g_@@_variants_seq #1
        \seq_gpop:NN \g_@@_variants_seq \l_@@_tmpb_tl
        \@@_names_typeset_auxii:x
          { \g_@@_base_name_tl : \l_@@_tmpb_tl }
      }
  }
%    \end{macrocode}
% \end{macro}
% \end{macro}
%
% \begin{macro}
%   {\@@_names_typeset_auxii:n, \@@_names_typeset_auxii:x}
%   In case the option |pTF| was given, typeset predicates before the
%   |TF| functions.  In case the option |noTF| was given, typeset the
%   non-|TF| function as well.  Pass the relevant boolean in both cases
%   to control whether to append |TF|.
%    \begin{macrocode}
\cs_new_protected:Npn \@@_names_typeset_auxii:n #1
  {
    \bool_if:NT \l_@@_macro_pTF_bool
      {
        \@@_names_typeset_block:xN
          { \@@_predicate_from_base:n {#1} }
          \c_false_bool
      }
    \bool_if:NT \l_@@_macro_noTF_bool
      { \@@_names_typeset_block:nN {#1} \c_false_bool }
    \@@_names_typeset_block:nN {#1} \l_@@_macro_TF_bool
  }
\cs_generate_variant:Nn \@@_names_typeset_auxii:n { x }
%    \end{macrocode}
% \end{macro}
%
% \begin{macro}{\@@_names_typeset_block:nN, \@@_names_typeset_block:xN}
%   Names in \env{function} and \env{macro} environments are typeset
%   differently.  To distinguish the two note that
%   \cs{l_@@_nested_macro_int} is at least one when in an \env{macro}
%   environment (we assume \env{function} is not nested inside it).  A
%   block is a function with all its variants.
%    \begin{macrocode}
\cs_new_protected:Npn \@@_names_typeset_block:nN
  {
    \int_compare:nNnTF \l_@@_nested_macro_int = 0
      { \@@_typeset_function_block:nN }
      { \@@_macro_typeset_block:nN }
  }
\cs_generate_variant:Nn \@@_names_typeset_block:nN { x }
%    \end{macrocode}
% \end{macro}
%
% \begin{macro}[pTF]{\@@_if_macro_internal:n}
% \begin{macro}[EXP]{\@@_if_macro_internal_aux:w}
%   Determines whether the given macro should be considered internal or
%   public.  If an option such as |int| was given then the answer is
%   \cs{l_@@_macro_internal_bool}, otherwise check for whether the macro
%   name contains~|__|.
%    \begin{macrocode}
\prg_new_conditional:Npnn \@@_if_macro_internal:n #1 { p , T , F , TF }
  {
    \bool_if:NTF \l_@@_macro_internal_set_bool
      {
        \bool_if:NTF \l_@@_macro_internal_bool
          { \prg_return_true: } { \prg_return_false: }
      }
      {
        \tl_if_empty:fTF
          {
            \exp_after:wN \@@_if_macro_internal_aux:w
            \tl_to_str:n { #1 ~ __ }
          }
          { \prg_return_false: } { \prg_return_true: }
      }
  }
\exp_last_unbraced:NNNNo
  \cs_new:Npn \@@_if_macro_internal_aux:w #1 { \tl_to_str:n { __ } } { }
%    \end{macrocode}
% \end{macro}
% \end{macro}
%
% \begin{macro}{\@@_names_block_base_map:N}
%   The \cs{l_@@_names_block_tl} contains sequence variables
%   corresponding to different base functions and their variants.  For
%   each such sequence, put the first and second items in
%   \cs{l_@@_tmpa_tl} and \cs{l_@@_tmpb_tl} and build the base
%   function's name.
%    \begin{macrocode}
\cs_new_protected:Npn \@@_names_block_base_map:N #1
  {
    \tl_map_inline:Nn \l_@@_names_block_tl
      {
        \group_begin:
          \seq_set_eq:NN \l_@@_tmpa_seq ##1
          \seq_pop:NN \l_@@_tmpa_seq \l_@@_tmpa_tl
          \seq_get:NN \l_@@_tmpa_seq \l_@@_tmpb_tl
          \exp_args:NNx
        \group_end:
        #1
          {
            \l_@@_tmpa_tl
            \str_if_eq:VnF \l_@@_tmpb_tl { \scan_stop: }
              { : \l_@@_tmpb_tl }
            \bool_if:NT \l_@@_macro_TF_bool { TF }
          }
      }
  }
%    \end{macrocode}
% \end{macro}
%
% \subsubsection{The \env{function} environment}
%
%    \begin{macrocode}
\keys_define:nn { l3doc/function }
  {
    TF .value_forbidden:n = true ,
    TF .code:n =
      {
        \bool_set_true:N \l_@@_macro_TF_bool
      } ,
    EXP .value_forbidden:n = true ,
    EXP .code:n =
      {
        \bool_set_true:N \l_@@_macro_EXP_bool
        \bool_set_false:N \l_@@_macro_rEXP_bool
      } ,
    rEXP .value_forbidden:n = true ,
    rEXP .code:n =
      {
        \bool_set_false:N \l_@@_macro_EXP_bool
        \bool_set_true:N \l_@@_macro_rEXP_bool
      } ,
    pTF .value_forbidden:n = true ,
    pTF .code:n =
      {
        \bool_set_true:N \l_@@_macro_pTF_bool
        \bool_set_true:N \l_@@_macro_TF_bool
        \bool_set_true:N \l_@@_macro_EXP_bool
        \bool_set_false:N \l_@@_macro_rEXP_bool
      } ,
    noTF .value_forbidden:n = true ,
    noTF .code:n =
      {
        \bool_set_true:N \l_@@_macro_noTF_bool
        \bool_set_true:N \l_@@_macro_TF_bool
      } ,
    added .code:n = { \@@_date_set_past:Nn \l_@@_date_added_tl {#1} },
    updated .code:n = { \@@_date_set_past:Nn \l_@@_date_updated_tl {#1} } ,
    deprecated .code:n = { \@@_deprecated_on:n {#1} } ,
    tested .code:n = { } ,
    label .code:n =
      {
        \clist_set:Nn \l_@@_function_label_clist {#1}
        \bool_set_true:N \l_@@_no_label_bool
      } ,
    verb .value_forbidden:n = true ,
    verb .bool_set:N = \l_@@_names_verb_bool ,
    module .tl_set:N = \l_@@_override_module_tl ,
  }
%    \end{macrocode}
%
% \begin{macro}[do-not-index={\A,\Z,\d,\1,\2,\3}]
%     {\@@_date_set:Nn,\@@_date_set_past:Nn}
%   Normalize the date into the format \texttt{YYYY-MM-DD}; more
%   precisely month and day are allowed to be single digits.  The
%   \cs{@@_date_set_past:Nn} function only allows dates in the past (or
%   same day).
%    \begin{macrocode}
\cs_new_protected:Npn \@@_date_set:Nn #1#2
  {
    \tl_set:Nn #1 {#2}
    \regex_replace_once:nnNF
      { \A(\d\d\d\d)[-/](\d\d?)[-/](\d\d?)\Z } { \1-\2-\3 } #1
      {
        \msg_error:nnn { l3doc } { date-format } {#2}
        \tl_set:Nn #1 { 1970-01-01 }
      }
  }
\cs_new_protected:Npn \@@_date_set_past:Nn #1#2
  {
    \@@_date_set:Nn #1 {#2}
    \exp_args:No \@@_date_compare:nNnT
      {#1} > { \c_sys_year_int - \c_sys_month_int - \c_sys_day_int }
      {
        \msg_error:nnxx { l3doc } { future-date }
          { \tl_to_str:N \l_@@_macro_argument_tl }
          {#1}
      }
  }
%    \end{macrocode}
% \end{macro}
%
% \begin{macro}{\@@_deprecated_on:n}
%   The date comparison function expects two dates in the YYYY-MM-DD
%   format (|-|~is not subtraction here).
%   Complain if a deprecated function should have been removed earlier.
%   In any case, mark it as internal to suppress the text
%   \enquote{documented on page \ldots{}}.
%    \begin{macrocode}
\cs_new_protected:Npn \@@_deprecated_on:n #1
  {
    \@@_date_set:Nn \l_@@_tmpa_tl {#1}
    \exp_args:No \@@_date_compare:nNnT
      { \l_@@_tmpa_tl } < { \c_sys_year_int - \c_sys_month_int - \c_sys_day_int }
      {
        \msg_error:nnxx { l3doc } { deprecated-function }
          { \tl_to_str:N \l_@@_macro_argument_tl }
          { \l_@@_tmpa_tl }
      }
    \bool_set_true:N \l_@@_macro_internal_bool
    \bool_set_true:N \l_@@_macro_internal_set_bool
  }
%    \end{macrocode}
% \end{macro}
%
% \begin{macro}{\@@_function:nnw}
%   \begin{arguments}
%     \item Key--value list.
%     \item Comma-separated list of functions; input has already been
%       sanitised by catcode changes before reading the argument.
%   \end{arguments}
% \begin{macro}{\@@_function_end:}
%   Make sure any paragraph is finished, and similar safe practices at
%   the beginning of an environment which will typeset material.
%   Initialize some variables.  Parse the key--value list.  Clean up the
%   list of functions, then go through them to extract some data.  After
%   this, typeset the function names in the coffin
%   \cs{l_@@_functions_coffin} and measure it to know if it fits in the
%   margin.  Finally, start a vertical coffin for the main part of the
%   environment.  This coffin stops when the environment ends, then all
%   the pieces are assembled into a single coffin, which is typeset.
%    \begin{macrocode}
\cs_new_protected:Npn \@@_function:nnw #1#2
  {
    \@@_function_typeset_start:
    \@@_function_init:
    \tl_set:Nn \l_@@_macro_argument_tl {#2}
    \keys_set:nn { l3doc/function } {#1}
    \@@_names_get_seq:nN {#2} \l_@@_names_seq
    \@@_names_parse:
    \@@_function_typeset:
    \@@_function_reset:
    \@@_function_descr_start:w
  }
\cs_new_protected:Npn \@@_function_end:
  {
    \@@_function_descr_stop:
    \@@_function_assemble:
    \@@_function_typeset_stop:
  }
%    \end{macrocode}
% \end{macro}
% \end{macro}
%
% \begin{macro}
%   {\@@_function_typeset_start:, \@@_function_typeset_stop:}
%   At the start of the \env{function} environment, before performing
%   any assignment, close the last paragraph, and set up the typesetting
%   scene.  Further code typesets a coffin, so we end the paragraph and
%   allow a page break.
%    \begin{macrocode}
\cs_new_protected:Npn \@@_function_typeset_start:
  {
    \par \bigskip \noindent
  }
\cs_new_protected:Npn \@@_function_typeset_stop:
  {
    \par
    \dim_set:Nn \prevdepth { \box_dp:N \l_@@_descr_coffin }
    \allowbreak
  }
%    \end{macrocode}
% \end{macro}
%
% \begin{macro}{\@@_function_init:}
%   Complain if \texttt{function} environments are nested.  Clear
%   various variables.
%    \begin{macrocode}
\cs_new_protected:Npn \@@_function_init:
  {
    \box_if_empty:NF \g_@@_syntax_box
      { \msg_error:nn { l3doc } { syntax-nested-function } }
    \coffin_clear:N \l_@@_descr_coffin
    \box_gclear:N \g_@@_syntax_box
    \coffin_clear:N \l_@@_syntax_coffin
    \coffin_clear:N \l_@@_functions_coffin
    \bool_set_false:N \l_@@_macro_TF_bool
    \bool_set_false:N \l_@@_macro_pTF_bool
    \bool_set_false:N \l_@@_macro_noTF_bool
    \bool_set_false:N \l_@@_macro_EXP_bool
    \bool_set_false:N \l_@@_macro_rEXP_bool
    \bool_set_false:N \l_@@_no_label_bool
    \bool_set_false:N \l_@@_names_verb_bool
    \bool_set_true:N \l_@@_in_function_bool
    \clist_clear:N \l_@@_function_label_clist
    \tl_set:Nn \l_@@_override_module_tl { \q_no_value }
    \char_set_active_eq:NN \< \@@_shorthand_meta:
    \char_set_catcode_active:N \<
  }
%    \end{macrocode}
% \end{macro}
%
% \begin{macro}{\@@_shorthand_meta:, \@@_shorthand_meta:w}
%   Allow |<...>| to be used as markup for |\meta{...}|.
%    \begin{macrocode}
\cs_new_protected:Npn \@@_shorthand_meta:
  { \mode_if_math:TF { < } { \@@_shorthand_meta:w } }
\cs_new_protected_nopar:Npn \@@_shorthand_meta:w #1 > { \meta {#1} }
%    \end{macrocode}
% \end{macro}
%
% \begin{macro}{\@@_function_reset:}
%   Clear some variables.
%    \begin{macrocode}
\cs_new_protected:Npn \@@_function_reset:
  {
    \tl_set:Nn \l_@@_override_module_tl { \q_no_value }
  }
%    \end{macrocode}
% \end{macro}
%
% \begin{macro}{\@@_function_typeset:}
%   Typeset in the coffin \cs{l_@@_functions_coffin} the functions listed in
%   \cs{l_@@_names_block_tl} and the relevant dates, then set
%   \cs{l_@@_long_name_bool} to be \texttt{true} if this coffin is
%   larger than the available width in the margin.  The function
%   \cs{@@_typeset_functions:} is quite involved hence given later.
%    \begin{macrocode}
\cs_new_protected:Npn \@@_function_typeset:
  {
    \dim_zero:N \l_@@_trial_width_dim
    \hcoffin_set:Nn \l_@@_functions_coffin { \@@_typeset_functions: }
    \dim_set:Nn \l_@@_trial_width_dim
      { \box_wd:N \l_@@_functions_coffin }
    \bool_set:Nn \l_@@_long_name_bool
      { \dim_compare_p:nNn \l_@@_trial_width_dim > \marginparwidth }
  }
%    \end{macrocode}
% \end{macro}
%
% \begin{macro}
%   {\@@_function_descr_start:w, \@@_function_descr_stop:}
%   The last step in \cs{@@_function:nnw} (the beginning of a
%   \env{function} environment) is to open a coffin which will capture
%   the description of the function, namely the body of the
%   \env{function} environment.  This is closed by \cs{@@_function_end:}
%   (the end of a \env{function} environment).
%    \begin{macrocode}
\cs_new_protected:Npn \@@_function_descr_start:w
  {
    \vcoffin_set:Nnw \l_@@_descr_coffin { \textwidth }
      \noindent \ignorespaces
  }
\cs_new_protected:Npn \@@_function_descr_stop:
  { \vcoffin_set_end: }
%    \end{macrocode}
% \end{macro}
%
% \begin{macro}{\@@_function_assemble:}
%   The box \cs{g_@@_syntax_box} contains the contents of the syntax
%   environment if it was used.  Now that we have all the pieces, join
%   together the syntax coffin, the names coffin, and the description
%   coffin.  The relative positions depend on whether the names coffin
%   fits in the margin.  Then typeset the combination.
%    \begin{macrocode}
\cs_new_protected:Npn \@@_function_assemble:
  {
    \hcoffin_set:Nn  \l_@@_syntax_coffin
      { \box_use_drop:N \g_@@_syntax_box }
    \bool_if:NTF \l_@@_long_name_bool
      {
        \coffin_join:NnnNnnnn
          \l_@@_output_coffin {hc} {vc}
          \l_@@_syntax_coffin {l} {T}
          {0pt} {0pt}
        \coffin_join:NnnNnnnn
          \l_@@_output_coffin {l} {t}
          \l_@@_functions_coffin  {r} {t}
          {-\marginparsep} {0pt}
        \coffin_join:NnnNnnnn
          \l_@@_output_coffin {l} {b}
          \l_@@_descr_coffin  {l} {t}
          {0.75\marginparwidth + \marginparsep} {-\medskipamount}
        \coffin_typeset:Nnnnn \l_@@_output_coffin
          {\l_@@_descr_coffin-l} {\l_@@_descr_coffin-t}
          {0pt} {0pt}
      }
      {
        \coffin_join:NnnNnnnn
          \l_@@_output_coffin {hc} {vc}
          \l_@@_syntax_coffin {l} {t}
          {0pt} {0pt}
        \coffin_join:NnnNnnnn
          \l_@@_output_coffin {l} {b}
          \l_@@_descr_coffin  {l} {t}
          {0pt} {-\medskipamount}
        \coffin_join:NnnNnnnn
          \l_@@_output_coffin {l} {t}
          \l_@@_functions_coffin  {r} {t}
          {-\marginparsep} {0pt}
        \coffin_typeset:Nnnnn \l_@@_output_coffin
          {\l_@@_syntax_coffin-l} {\l_@@_syntax_coffin-T}
          {0pt} {0pt}
      }
  }
%    \end{macrocode}
% \end{macro}
%
% \begin{macro}{\@@_typeset_functions:}
%   This function builds the \cs{l_@@_functions_coffin} by typesetting the
%   function names (with variants) and the relevant dates in a
%   \env{tabular} environment.  The use of rules \tn{toprule},
%   \tn{midrule} and \tn{bottomrule} requires whatever lies between the
%   last |\\| and the rule to be expandable, making our lives a bit
%   complicated.
%    \begin{macrocode}
\cs_new_protected:Npn \@@_typeset_functions:
  {
    \small\ttfamily
    \HD@savedestfalse
    \HD@target
    % \Hy@MakeCurrentHref { HD. \int_use:N \c@HD@hypercount }
    \begin{tabular} [t] { @{} l @{} >{\hspace{\tabcolsep}} r @{} }
      \toprule
      \@@_function_extra_labels:
      \@@_names_typeset:
      \@@_typeset_dates:
      \bottomrule
    \end{tabular}
    \normalfont\normalsize
  }
%    \end{macrocode}
% \end{macro}
%
% ^^A TODO: collect all index targets from a given function environment in a box and stick it at the top.
% \begin{macro}
%   {\@@_typeset_function_block:nN, \@@_typeset_function_block:xN}
% \begin{macro}{\@@_function_index:n, \@@_function_index:x}
%   |#1| is a csname, |#2| a boolean indicating whether to add |TF| or not.
%    \begin{macrocode}
\cs_new_protected:Npn \@@_typeset_function_block:nN #1#2
  {
    \@@_function_index:x
      { #1 \bool_if:NT #2 { \tl_to_str:n {TF} } }
    \@@_function_label:xN {#1} #2
    #1
    \bool_if:NT #2 { \@@_typeset_TF: }
    \@@_typeset_expandability:
    \seq_if_empty:NF \g_@@_variants_seq
      { \@@_typeset_variant_list:nN {#1} #2 }
    \\
  }
\cs_generate_variant:Nn \@@_typeset_function_block:nN { x }
\cs_new_protected:Npn \@@_function_index:n #1
  {
    \seq_gput_right:Nn \g_doc_functions_seq {#1}
    \@@_special_index:nn {#1} { usage }
  }
\cs_generate_variant:Nn \@@_function_index:n { x }
%    \end{macrocode}
%
%    \begin{macrocode}
\cs_new_protected:Npn \@@_typeset_expandability:
  {
    &
    \bool_if:NT \l_@@_macro_EXP_bool  { \@@_typeset_exp: }
    \bool_if:NT \l_@@_macro_rEXP_bool { \@@_typeset_rexp: }
  }
%    \end{macrocode}
%
% |#1| is the function, |#2| whether to add |TF|.
%    \begin{macrocode}
\cs_new_protected:Npn \@@_typeset_variant_list:nN #1#2
  {
    \\
    \@@_typeset_aux:n { \@@_get_function_name:n {#1} }
    :
    \int_compare:nTF { \seq_count:N \g_@@_variants_seq == 1 }
      { \seq_use:Nn \g_@@_variants_seq { } }
      {
        \textrm(
          \seq_use:Nn \g_@@_variants_seq { \textrm| }
        \textrm)
      }
    \bool_if:NT #2 { \@@_typeset_TF: }
    \@@_typeset_expandability:
  }
%    \end{macrocode}
%
% |#1| is the function name, |#2| whether to add |TF|.
%    \begin{macrocode}
\cs_new_protected:Npn \@@_function_extra_labels:
  {
    \bool_if:NT \l_@@_no_label_bool
      {
        \clist_map_inline:Nn \l_@@_function_label_clist
          {
            \@@_get_hyper_target:oN { \token_to_str:N ##1 }
              \l_@@_tmpa_tl
            \exp_args:No \label { \l_@@_tmpa_tl }
          }
      }
  }
\cs_new_protected:Npn \@@_function_label:nN #1#2
  {
    \bool_if:NF \l_@@_no_label_bool
      {
        \@@_get_hyper_target:xN
          {
            \exp_not:n {#1}
            \bool_if:NT #2 { \tl_to_str:n {TF} }
          }
          \l_@@_tmpa_tl
        \exp_args:No \label { \l_@@_tmpa_tl }
      }
  }
\cs_generate_variant:Nn \@@_function_label:nN { x }
%    \end{macrocode}
% \end{macro}
% \end{macro}
%
% \begin{macro}{\@@_typeset_dates:}
%   To display metadata for when functions are added/modified.
%   This function must be expandable since it produces rules for use in
%   alignments.
%    \begin{macrocode}
\cs_new:Npn \@@_typeset_dates:
  {
    \bool_lazy_and:nnF
      { \tl_if_empty_p:N \l_@@_date_added_tl }
      { \tl_if_empty_p:N \l_@@_date_updated_tl }
      { \midrule }
    \tl_if_empty:NF \l_@@_date_added_tl
      {
        \multicolumn { 2 } { @{} r @{} }
          { \scriptsize New: \, \l_@@_date_added_tl } \\
      }

    \tl_if_empty:NF \l_@@_date_updated_tl
      {
        \multicolumn { 2 } { @{} r @{} }
          { \scriptsize Updated: \, \l_@@_date_updated_tl } \\
      }
  }
%    \end{macrocode}
% \end{macro}
%
% \begin{macro}{\@@_syntax:w, \@@_syntax_end:}
%   Implement the \env{syntax} environment.
%    \begin{macrocode}
\dim_new:N \l_@@_syntax_dim
\cs_new_protected:Npn \@@_syntax:w
  {
    \box_if_empty:NF \g_@@_syntax_box
      { \msg_error:nn { l3doc } { multiple-syntax } }
    \dim_set:Nn \l_@@_syntax_dim
      {
        \textwidth
        \bool_if:NT \l_@@_long_name_bool
          { + 0.75 \marginparwidth - \l_@@_trial_width_dim }
      }
    \hbox_gset:Nw \g_@@_syntax_box
      \small \ttfamily
      \arrayrulecolor{white}
      \begin{tabular} { @{} l @{} }
        \toprule
        \begin{minipage}[t]{\l_@@_syntax_dim}
          \raggedright
          \obeyspaces
          \obeylines
  }
\cs_new_protected:Npn \@@_syntax_end:
  {
        \end{minipage}
      \end{tabular}
      \arrayrulecolor{black}
    \hbox_gset_end:
    \bool_if:NF \l_@@_in_function_bool
      {
        \begin{quote}
          \mode_leave_vertical:
          \box_use_drop:N \g_@@_syntax_box
        \end{quote}
      }
  }
%    \end{macrocode}
% \end{macro}
%
% \subsubsection{The \env{macro} environment}
%
% Keyval for the \env{macro} environment.
% TODO: provide document command for documenting keys.
%    \begin{macrocode}
\keys_define:nn { l3doc/macro }
  {
    aux .value_forbidden:n = true ,
    aux .code:n =
      {
        \msg_warning:nnnn { l3doc } { deprecated-option }
          { aux } { function/macro }
      } ,
    internal .value_forbidden:n = true ,
    internal .code:n =
      {
        \bool_set_true:N \l_@@_macro_internal_bool
        \bool_set_true:N \l_@@_macro_internal_set_bool
      } ,
    int .value_forbidden:n = true ,
    int .code:n =
      {
        \bool_set_true:N \l_@@_macro_internal_bool
        \bool_set_true:N \l_@@_macro_internal_set_bool
      } ,
    var .value_forbidden:n = true ,
    var .code:n =
      { \bool_set_true:N \l_@@_macro_var_bool } ,
    TF .value_forbidden:n = true ,
    TF .code:n =
      { \bool_set_true:N \l_@@_macro_TF_bool } ,
    pTF .value_forbidden:n = true ,
    pTF .code:n =
      {
        \bool_set_true:N \l_@@_macro_TF_bool
        \bool_set_true:N \l_@@_macro_pTF_bool
        \bool_set_true:N \l_@@_macro_EXP_bool
        \bool_set_false:N \l_@@_macro_rEXP_bool
      } ,
    noTF .value_forbidden:n = true ,
    noTF .code:n =
      {
        \bool_set_true:N \l_@@_macro_TF_bool
        \bool_set_true:N \l_@@_macro_noTF_bool
      } ,
    EXP .value_forbidden:n = true ,
    EXP .code:n =
      {
        \bool_set_true:N \l_@@_macro_EXP_bool
        \bool_set_false:N \l_@@_macro_rEXP_bool
      } ,
    rEXP .value_forbidden:n = true ,
    rEXP .code:n =
      {
        \bool_set_false:N \l_@@_macro_EXP_bool
        \bool_set_true:N \l_@@_macro_rEXP_bool
      } ,
    tested .code:n =
      {
        \bool_set_true:N \l_@@_macro_tested_bool
      } ,
    added .code:n = {} , % TODO
    updated .code:n = {} , % TODO
    deprecated .code:n = { \@@_deprecated_on:n {#1} } ,
    verb .bool_set:N = \l_@@_names_verb_bool ,
    module .tl_set:N = \l_@@_override_module_tl ,
    documented-as .tl_set:N = \l_@@_macro_documented_tl ,
    do-not-index .value_required:n = true ,
    do-not-index .tl_set:N = \l_@@_macro_do_not_index_tl ,
    % do-not-index .default:n = \q_no_value ,
  }
%    \end{macrocode}
%
% \begin{macro}{\@@_macro:nnw}
%   The arguments are a key--value list of \meta{options} and a
%   comma-list of \meta{names}, read verbatim by \pkg{xparse}.  First
%   initialize some variables before applying the \meta{options}, then
%   parse the \meta{names} to get a sequence of macro names, then apply
%   \cs{@@_macro_single:nNN} to each (this step is more subtle than
%   \cs{seq_map_function:NN} because of |TF|/|pTF|/|noTF|).  Finally typeset
%   the macro names in the margin.
%    \begin{macrocode}
\cs_new_protected:Npn \@@_macro:nnw #1#2
  {
    \@@_macro_init:
    \tl_set:Nn \l_@@_macro_argument_tl {#2}
    \keys_set:nn { l3doc/macro } {#1}
    \@@_names_get_seq:nN {#2} \l_@@_names_seq
    \@@_names_parse:
    \@@_macro_exclude_index:
    \@@_macro_save_names:
    \@@_names_typeset:
    \@@_macro_dump:
    \@@_macro_reset:
  }
%    \end{macrocode}
% \end{macro}
%
% \begin{macro}{\@@_macro_init:}
%   The booleans hold various key--value options,
%   \cs{l_@@_nested_macro_int} counts the number of \env{macro}
%   environments around the current point (is $0$ outside).
%    \begin{macrocode}
\cs_new_protected:Npn \@@_macro_init:
  {
    \int_incr:N \l_@@_nested_macro_int
    \bool_set_false:N \l_@@_macro_internal_bool
    \bool_set_false:N \l_@@_macro_internal_set_bool
    \bool_set_false:N \l_@@_macro_TF_bool
    \bool_set_false:N \l_@@_macro_pTF_bool
    \bool_set_false:N \l_@@_macro_noTF_bool
    \bool_set_false:N \l_@@_macro_EXP_bool
    \bool_set_false:N \l_@@_macro_rEXP_bool
    \bool_set_false:N \l_@@_macro_var_bool
    \bool_set_false:N \l_@@_macro_tested_bool
    \bool_set_false:N \l_@@_names_verb_bool
    \tl_set:Nn \l_@@_override_module_tl { \q_no_value }
    \tl_clear:N \l_@@_macro_documented_tl
    \cs_set_eq:NN \testfile \@@_print_testfile:n
    \box_clear:N \l_@@_macro_index_box
    \vbox_set:Nn \l_@@_macro_box
      {
        \hbox:n
          {
            \strut
            \int_compare:nNnT \l_@@_macro_int = 0
              { \HD@target }
          }
        \vskip \int_eval:n { \l_@@_macro_int - 1 } \baselineskip
      }
  }
%    \end{macrocode}
% \end{macro}
%
% \begin{macro}{\@@_macro_reset:}
%   We ensure that \cs{cs} commands nested inside a macro whose module
%   is imposed are not affected.
%    \begin{macrocode}
\cs_new_protected:Npn \@@_macro_reset:
  {
    \tl_set:Nn \l_@@_override_module_tl { \q_no_value }
  }
%    \end{macrocode}
% \end{macro}
%
% \begin{macro}{\@@_macro_save_names:}
%   The list of names defined in a set of \env{macro} environments is
%   eventually used to display on which page they are documented.  If
%   the |documented-as| key is given, use that, otherwise find names in
%   \cs{l_@@_names_block_tl}.
%    \begin{macrocode}
\cs_new_protected:Npn \@@_macro_save_names:
  {
    \tl_if_empty:NTF \l_@@_macro_documented_tl
      { \@@_names_block_base_map:N \@@_macro_save_names_aux:n }
      {
        \seq_gput_right:Nf \g_@@_nested_names_seq
          { \exp_after:wN \token_to_str:N \l_@@_macro_documented_tl }
      }
  }
\cs_new_protected:Npn \@@_macro_save_names_aux:n #1
  { \seq_gput_right:Nn \g_@@_nested_names_seq {#1} }
%    \end{macrocode}
% \end{macro}
%
% \begin{macro}{\@@_macro_exclude_index:}
%   Some control sequences in a \env{macrocode} environment shouldn't
%   be indexed, for different reasons. This macro parses the argument
%   of the |do-not-index| option and locally removes the given macros
%   from the index.
%
%   The optional argument to \env{macro} is not scanned with verbatim
%   catcodes, so we use \cs{tl_set_rescan:NnV} to rescan the commands
%   with the same catcodes as \cs{DoNotIndex}. The scanned token list
%   contains spaces after control sequences, which are not there when
%   \cs{DoNotIndex} is used. Since \cs{seq_set_from_clist:Nn} removes
%   spaces around the items, we can abuse that and \cs{seq_use:Nn} to
%   normalise each item. After that \cs{DoNotIndex} can do its thing.
%    \begin{macrocode}
\cs_new_protected:Npn \@@_macro_exclude_index:
  {
    \tl_if_empty:NF \l_@@_macro_do_not_index_tl
      {
        \tl_set_rescan:NnV \l_@@_macro_do_not_index_tl
          { \MakePrivateLetters \catcode`\\12 }
          \l_@@_macro_do_not_index_tl
        \exp_args:NNV \seq_set_from_clist:Nn
          \l_@@_tmpa_seq \l_@@_macro_do_not_index_tl
        \tl_set:Nx \l_@@_macro_do_not_index_tl
          { \seq_use:Nn \l_@@_tmpa_seq { , } }
        \exp_args:NV \DoNotIndex \l_@@_macro_do_not_index_tl
      }
  }
%    \end{macrocode}
% \end{macro}
%
% \begin{macro}{\@@_macro_dump:}
%   This calls |\makelabel{}|
%    \begin{macrocode}
\cs_new_protected:Npn \@@_macro_dump:
  {

    \topsep\MacroTopsep
    \trivlist
    \cs_set:Npn \makelabel ##1
      {
        \llap
          {
            \hbox_unpack_drop:N \l_@@_macro_index_box
            \vtop to \baselineskip
              {
                \vbox_unpack_drop:N \l_@@_macro_box
                \vss
              }
          }
      }
    \item [ ]
  }
%    \end{macrocode}
% \end{macro}
%
% \begin{macro}{\@@_macro_typeset_block:nN}
%   Used to typeset a macro and its variants.  |#1| is the macro name,
%   |#2| is a boolean controlling whether to add |TF|.
%    \begin{macrocode}
\cs_new_protected:Npn \@@_macro_typeset_block:nN #1#2
  {
    \@@_macro_single:nNN {#1} \c_true_bool #2
    \seq_if_empty:NF \g_@@_variants_seq
      {
        \@@_macro_typeset_variant_list:xN
          { \@@_get_function_name:n {#1} } #2
      }
  }
\cs_generate_variant:Nn \@@_macro_typeset_block:nN { x }
\cs_new_protected:Npn \@@_macro_typeset_variant_list:nN #1#2
  {
    \seq_map_inline:Nn \g_@@_variants_seq
      { \@@_macro_single:nNN { #1 : ##1 } \c_false_bool #2 }
  }
\cs_generate_variant:Nn \@@_macro_typeset_variant_list:nN { x }
%    \end{macrocode}
% \end{macro}
%
% \begin{macro}{\@@_macro_single:nNN}
%   The arguments are |#1| a macro name (without |TF|), |#2| a boolean
%   determining whether or not to index, and |#3| whether or not to add |TF|.
%   Let's start to mess around with \cls{doc}'s \env{macro} environment.
%   See \file{doc.dtx} for a full explanation of the original
%   environment.  It's rather \emph{enthusiastically} commented.
%   \begin{arguments}
%     \item Macro/function/whatever name; input has already been
%       sanitised.
%   \end{arguments}
%   The assignments to \cs{saved@macroname} and \cs{saved@indexname}
%   are used by \pkg{doc}'s \cs{changes} mechanism.
%    \begin{macrocode}
\cs_new_protected:Npn \@@_macro_single:nNN #1#2#3
  {
    \tl_set:Nn \saved@macroname {#1}
    \@@_macro_typeset_one:nN {#1} #3
    \bool_if:NT #3 { \DoNotIndex {#1} }
    \exp_args:Nx \@@_macro_index:nN
      { #1 \bool_if:NT #3 { \tl_to_str:n { TF } } }
      #2
  }
\cs_new_protected:Npn \@@_macro_index:nN #1#2
  {
    \DoNotIndex {#1}
    \bool_if:NT #2
      {
        \@@_if_macro_internal:nF {#1}
          { \seq_gput_right:Nn \g_doc_macros_seq {#1} }
        \hbox_set:Nw \l_@@_macro_index_box
          \hbox_unpack_drop:N \l_@@_macro_index_box
          \int_gincr:N \c@CodelineNo
          \@@_special_index:nn {#1} { main }
          \int_gdecr:N \c@CodelineNo
        \exp_args:NNNo \hbox_set_end:
          \tl_set:Nn \saved@indexname { \l_@@_index_key_tl }
      }
  }
%    \end{macrocode}
% \end{macro}
%
% \begin{macro}{\@@_macro_typeset_one:nN}
%   For a long time, \cls{l3doc} collected the macro names as labels in
%   the first items of nested \tn{trivlist}, but these were not closed
%   properly with \tn{endtrivlist}.  Also, it interacted in surprising
%   ways with \pkg{hyperref} targets.  Now, we collect typeset macro
%   names by hand in the box \cs{l_@@_macro_box}.  Note the space |\ |.
%   |#1| is the macro name, |#2| whether to add |TF|.
%    \begin{macrocode}
\cs_new_protected:Npn \@@_macro_typeset_one:nN #1#2
  {
    \vbox_set:Nn \l_@@_macro_box
      {
        \vbox_unpack_drop:N \l_@@_macro_box
        \hbox { \llap { \@@_print_macroname:nN {#1} #2 \ } }
      }
    \int_incr:N \l_@@_macro_int
  }
%    \end{macrocode}
% \end{macro}
%
% \begin{macro}{\@@_print_macroname:nN}
%   In the name, spaces are replaced by other spaces to ensure they get
%   displayed in case there are any.
%    \begin{macrocode}
\cs_new_protected:Npn \@@_print_macroname:nN #1#2
  {
    \strut
    \@@_get_hyper_target:xN
      {
        \exp_not:n {#1}
        \bool_if:NT #2 { \tl_to_str:n {TF} }
      }
      \l_@@_tmpa_tl
    \cs_if_exist:cTF { r@ \l_@@_tmpa_tl }
      { \exp_last_unbraced:NNo \hyperref [ \l_@@_tmpa_tl ] }
      { \use:n }
      {
        \int_compare:nTF { \str_count:n {#1} <= 28 }
          { \MacroFont } { \MacroLongFont }
        \tl_set:Nn \l_@@_tmpa_tl {#1}
        \tl_replace_all:Nno \l_@@_tmpa_tl
          { ~ } { \c_catcode_other_space_tl }
        \@@_macroname_prefix:o \l_@@_tmpa_tl
        \@@_macroname_suffix:N #2
      }
  }
\cs_new_protected:Npn \@@_macroname_prefix:n #1
  {
    \@@_if_macro_internal:nTF {#1}
      { \@@_typeset_aux:n {#1} } {#1}
  }
\cs_generate_variant:Nn \@@_macroname_prefix:n { o }
\cs_new_protected:Npn \@@_macroname_suffix:N #1
  { \bool_if:NTF #1 { \@@_typeset_TF: } { } }
%    \end{macrocode}
% \end{macro}
%
% \begin{macro}{\MacroLongFont}
%    \begin{macrocode}
\providecommand \MacroLongFont
  {
    \fontfamily{lmtt}\fontseries{lc}\small
  }
%    \end{macrocode}
% \end{macro}
%
% \begin{macro}{\@@_print_testfile:n, \@@_print_testfile_aux:n}
%   Used to show that a macro has a test, somewhere.
%    \begin{macrocode}
\cs_new_protected:Npn \@@_print_testfile:n #1
  {
    \bool_set_true:N \l_@@_macro_tested_bool
    \tl_if_eq:nnF {#1} {*}
      {
        \seq_if_in:NnF \g_@@_testfiles_seq {#1}
          {
            \seq_gput_right:Nn \g_@@_testfiles_seq {#1}
            \par
            \@@_print_testfile_aux:n {#1}
          }
      }
  }
\cs_new_protected:Npn \@@_print_testfile_aux:n #1
  {
    \footnotesize
    (
    \textit
      {
        The~ test~ suite~ for~ this~ command,~
        and~ others~ in~ this~ file,~ is~ \textsf{#1}
      }.
    )\par
  }
%    \end{macrocode}
% \end{macro}
%
% \begin{macro}{\TestFiles}
%    \begin{macrocode}
\DeclareDocumentCommand \TestFiles {m}
  {
    \par
    \textit
      {
        The~ following~ test~ files~ are~
        used~ for~ this~ code:~ \textsf{#1}.
      }
    \par \ignorespaces
  }
%    \end{macrocode}
% \end{macro}
%
% \begin{macro}{\UnitTested}
%    \begin{macrocode}
\DeclareDocumentCommand \UnitTested { } { \testfile* }
%    \end{macrocode}
% \end{macro}
%
% \begin{macro}{\TestMissing}
%    \begin{macrocode}
\DeclareDocumentCommand \TestMissing { m }
  { \@@_test_missing:n {#1} }
%    \end{macrocode}
% \end{macro}
%
% \begin{macro}{\@@_test_missing:n}
%   Keys in \cs{g_@@_missing_tests_prop} are lists of macros given as
%   arguments of one \env{macro} environment.  Values are pairs of a
%   file name and a comment about the missing tests.
%    \begin{macrocode}
\cs_new_protected:Npn \@@_test_missing:n #1
  {
    \@@_test_missing_aux:Nxn
      \g_@@_missing_tests_prop
      { \seq_use:Nn \l_@@_names_seq { , } }
      { { \g_file_curr_name_str \c_space_tl (#1) } }
  }
\cs_new_protected:Npn \@@_test_missing_aux:Nnn #1#2#3
  {
    \prop_get:NnNTF #1 {#2} \l_@@_tmpa_tl
      { \tl_put_right:Nn \l_@@_tmpa_tl { , #3 } }
      { \tl_set:Nn \l_@@_tmpa_tl {#3} }
    \prop_put:Nno #1 {#2} \l_@@_tmpa_tl
  }
\cs_generate_variant:Nn \@@_test_missing_aux:Nnn { Nx }
%    \end{macrocode}
% \end{macro}
%
% \begin{macro}{\@@_macro_end:}
%   It is too late for anyone to declare a test file for this macro, so
%   we can check now whether the macro is tested.  If the \env{macro}
%   environment which is being ended is the outermost one, then wrap
%   each macro in \tn{texttt} (with the addition of |TF| if relevant)
%   and typeset two informations: that this ends the definition of some
%   macros, and that they are documented on some page.
%    \begin{macrocode}
\cs_new_protected:Npn \@@_macro_end:
  {
    \endtrivlist
    \@@_macro_end_check_tested:
    \int_compare:nNnT \l_@@_nested_macro_int = 1
      { \@@_macro_end_style:n { \@@_print_end_definition: } }
  }
%    \end{macrocode}
% \end{macro}
%
% \begin{macro}{\@@_macro_end_check_tested:}
%   If the |checktest| option was issued and the macro is not an
%   auxiliary nor a variable (and it does not have a test), then add it
%   to the sequence of non-tested macros.
%    \begin{macrocode}
\cs_new_protected:Npn \@@_macro_end_check_tested:
  {
    \bool_lazy_all:nT
     {
       { \g_@@_checktest_bool }
       { ! \l_@@_macro_var_bool }
       { ! \l_@@_macro_tested_bool }
     }
     {
       \seq_set_filter:NNn \l_@@_tmpa_seq \l_@@_names_seq
         { ! \@@_if_macro_internal_p:n {##1} }
       \seq_gput_right:Nx \g_@@_not_tested_seq
         {
           \seq_use:Nn \l_@@_tmpa_seq { , }
           \bool_if:NTF \l_@@_macro_pTF_bool {~(pTF)}
             { \bool_if:NT \l_@@_macro_TF_bool {~(TF)} }
         }
     }
  }
%    \end{macrocode}
% \end{macro}
%
% \begin{macro}{\@@_macro_end_style:n}
%   Style for the extra information at the end of a top-level
%   \env{macro} environment.
%    \begin{macrocode}
\cs_new_protected:Npn \@@_macro_end_style:n #1
  {
    \nobreak \noindent
    { \footnotesize ( \emph{#1} ) \par }
  }
%    \end{macrocode}
% \end{macro}
%
% \begin{macro}
%   {
%     \@@_print_end_definition:,
%     \@@_macro_end_wrap_item:n,
%     \@@_print_documented:
%   }
%   Surround each item by \tn{texttt}, replacing |_|
%   by \tn{_} as well.  Then list the
%   macro names through \cs{seq_use:Nnnn}, unless there are too many.
%   Finally, if the macro is neither auxiliary nor internal, add a link
%   to where it is documented.
%    \begin{macrocode}
\cs_new_protected:Npn \@@_macro_end_wrap_item:n #1
  {
    \tl_set:Nn \l_@@_tmpa_tl {#1}
    \tl_replace_all:Non \l_@@_tmpa_tl
      { \token_to_str:N _ } { \_ }
    \texttt { \l_@@_tmpa_tl }
  }
\cs_new_protected:Npn \@@_print_end_definition:
  {
    \seq_set_map:NNn \l_@@_tmpa_seq
      \g_@@_nested_names_seq
      { \exp_not:n { \@@_macro_end_wrap_item:n {##1} } }
    End~ definition~ for~
    \int_compare:nTF { \seq_count:N \l_@@_tmpa_seq <= 3 }
      {
        \seq_use:Nnnn \l_@@_tmpa_seq
          { \,~and~ } { \,,~ } { \,,~and~ }
      }
      { \seq_item:Nn \l_@@_tmpa_seq {1}\,~and~others }
    \@.
    \@@_print_documented:
  }
\cs_new_protected:Npn \@@_print_documented:
  {
    \seq_gset_filter:NNn \g_@@_nested_names_seq
      \g_@@_nested_names_seq
      { ! \@@_if_macro_internal_p:n {##1} }
    \seq_if_empty:NF \g_@@_nested_names_seq
      {
        \int_set:Nn \l_@@_tmpa_int
          { \seq_count:N \g_@@_nested_names_seq }
        \int_compare:nNnTF \l_@@_tmpa_int = 1 {~This~} {~These~}
        \bool_if:NTF \l_@@_macro_var_bool {variable} {function}
        \int_compare:nNnTF \l_@@_tmpa_int = 1 {~is~} {s~are~}
        documented~on~page~
        \@@_get_hyper_target:xN
          { \seq_item:Nn \g_@@_nested_names_seq { 1 } }
          \l_@@_tmpa_tl
        \exp_args:Nx \pageref { \l_@@_tmpa_tl } .
      }
    \seq_gclear:N \g_@@_nested_names_seq
  }
%    \end{macrocode}
% \end{macro}
%
% \subsubsection{Misc}
%
% \begin{macro}{\DescribeOption}
%   For describing package options.  Due to Joseph Wright.  Name/usage
%   might change soon.
%    \begin{macrocode}
\newcommand*{\DescribeOption}
  {
    \leavevmode
    \@bsphack
    \begingroup
      \MakePrivateLetters
      \Describe@Option
  }
%    \end{macrocode}
%
%    \begin{macrocode}
\newcommand*{\Describe@Option}[1]
  {
    \endgroup
    \marginpar{
      \raggedleft
      \PrintDescribeEnv{#1}
    }
    \SpecialOptionIndex{#1}
    \@esphack
    \ignorespaces
  }
%    \end{macrocode}
%
%    \begin{macrocode}
\newcommand*{\SpecialOptionIndex}[1]
  {
    \@bsphack
    \begingroup
      \HD@target
      \let\HDorg@encapchar\encapchar
      \edef\encapchar usage
        {
          \HDorg@encapchar hdclindex{\the\c@HD@hypercount}{usage}
        }
      \index
        {
          #1\actualchar{\protect\ttfamily#1}~(option)
          \encapchar usage
        }
      \index
        {
          options:\levelchar#1\actualchar{\protect\ttfamily#1}
          \encapchar usage
        }
    \endgroup
    \@esphack
  }
%    \end{macrocode}
% \end{macro}
%
% Here are some definitions for additional markup that helps to
% structure your documentation.
%
% \begin{environment}{danger}
% \begin{environment}{ddanger}
%   \begin{syntax}
%     |\begin{[d]danger}|\\
%       dangerous code\\
%     |\end{[d]danger}|
%   \end{syntax}
%   \begin{danger}
%     Provides a danger bend, as known from the \TeX{}book.
%   \end{danger}
%   The actual character from the font |manfnt|:
%    \begin{macrocode}
\font \manual = manfnt \scan_stop:
\cs_gset:Npn \dbend { {\manual\char127} }
%    \end{macrocode}
%
% Defines the single danger bend. Use it whenever there is a feature in
% your package that might be tricky to use.  FIXME: Has to be fixed when
% in combination with a macro-definition.
%    \begin{macrocode}
\newenvironment {danger}
  {
    \begin{trivlist}\item[]\noindent
    \begingroup\hangindent=2pc\hangafter=-2
    \cs_set:Npn \par{\endgraf\endgroup}
    \hbox to0pt{\hskip-\hangindent\dbend\hfill}\ignorespaces
  }
  {
    \par\end{trivlist}
  }
%    \end{macrocode}
%
% \begin{ddanger}
%   Use the double danger bend if there is something which could cause
%   serious problems when used in a wrong way. Better the normal user
%   does not know about such things.
% \end{ddanger}
%    \begin{macrocode}
\newenvironment {ddanger}
  {
    \begin{trivlist}\item[]\noindent
    \begingroup\hangindent=3.5pc\hangafter=-2
    \cs_set:Npn \par{\endgraf\endgroup}
    \hbox to0pt{\hskip-\hangindent\dbend\kern2pt\dbend\hfill}\ignorespaces
  }{
      \par\end{trivlist}
  }
%    \end{macrocode}
% \end{environment}
% \end{environment}
%
% \subsubsection{NB and NOTE}
%
% These macros are intended for additional notes added to the source that are not typeset.
%
% \begin{macro}{\NB}
% \NB{wspr}{this is what I think about this!}
% \begin{verbatim}
%   \NB{wspr}{this is what I think about this!}
% \end{verbatim}
%    \begin{macrocode}
\bool_if:NTF \g_@@_show_notes_bool
  {
    \NewDocumentCommand\NB{mm}
      {
        (\emph{Note}\footnote{\ttfamily [#1]:~\detokenize{#2}})
      }
  }
  {
    \NewDocumentCommand\NB{mm}{}
  }
%    \end{macrocode}
% \end{macro}
%
% \begin{environment}{NOTE}
% \begin{NOTE}{wspr}
%   this is what I #$%& think about this!
% \end{NOTE}
% \begin{verbatim}
%   \begin{NOTE}{wspr}
%     this is what I #$%& think about this!
%   \end{NOTE}
% \end{verbatim}
%    \begin{macrocode}
\bool_if:NTF \g_@@_show_notes_bool
  {
    \NewDocumentEnvironment{NOTE}{m}
      {
        \par\noindent (\emph{Note}~[\texttt{#1}]:\par
        \verbatim
      }
      {
        \endverbatim
        \par\noindent \emph{Note~end})\par
      }
  }
  {
    \NewDocumentEnvironment{NOTE}{m}{\comment}{\endcomment}
  }
%    \end{macrocode}
% \end{environment}
%
% \subsection{Documenting templates}
%
%    \begin{macrocode}
\newenvironment{TemplateInterfaceDescription}[1]
  {
    \subsection{The~object~type~`#1'}
    \begingroup
    \@beginparpenalty\@M
    \description
    \def\TemplateArgument##1##2{\item[Arg:~##1]##2\par}
    \def\TemplateSemantics
      {
        \enddescription\endgroup
        \subsubsection*{Semantics:}
      }
  }
  {
    \par\bigskip
  }
%    \end{macrocode}
%
%    \begin{macrocode}
\newenvironment{TemplateDescription}[2]
  {
    \subsection{The~template~`#2'~(object~type~#1)}
    \subsubsection*{Attributes:}
    \begingroup
    \@beginparpenalty\@M
    \description
    \def\TemplateKey##1##2##3##4
      {
        \item[##1~(##2)]##3%
        \ifx\TemplateKey##4\TemplateKey\else
%         \hskip0ptplus3em\penalty-500\hskip 0pt plus 1filll Default:~##4%
          \hfill\penalty500\hbox{}\hfill Default:~##4%
          \nobreak\hskip-\parfillskip\hskip0pt\relax
        \fi
        \par
      }
    \def\TemplateSemantics
      {
        \enddescription\endgroup
        \subsubsection*{Semantics~\&~Comments:}
      }
  }
  { \par \bigskip }
%    \end{macrocode}
%
%    \begin{macrocode}
\newenvironment{InstanceDescription}[4][xxxxxxxxxxxxxxx]
  {
    \subsubsection{The~instance~`#3'~(template~#2/#4)}
    \subsubsection*{Attribute~values:}
    \begingroup
    \@beginparpenalty\@M
    \def\InstanceKey##1##2{\>\textbf{##1}\>##2\\}
    \def\InstanceSemantics{\endtabbing\endgroup
      \vskip-30pt\vskip0pt
      \subsubsection*{Layout~description~\&~Comments:}}
    \tabbing
    xxxx\=#1\=\kill
  }
  { \par \bigskip }
%    \end{macrocode}
%
% \subsection{Inheriting doc}
%
% Code here is taken from \pkg{doc}, stripped of comments and translated
% into \pkg{expl3} syntax. New features are added in various places.
%
% \begin{macro}
%   {\StopEventually, \Finale, \AlsoImplementation, \OnlyDescription}
% \begin{variable}{\g_@@_finale_tl}
%   TODO: remove these four commands altogether, document that it is
%   better to use the \env{documentation} and \env{implementation}
%   environments.
%    \begin{macrocode}
\DeclareDocumentCommand \OnlyDescription { }
  { \bool_gset_false:N \g_@@_typeset_implementation_bool }
\DeclareDocumentCommand \AlsoImplementation { }
  { \bool_gset_true:N \g_@@_typeset_implementation_bool }
\DeclareDocumentCommand \StopEventually { m }
  {
    \bool_if:NTF \g_@@_typeset_implementation_bool
      {
        \@bsphack
        \tl_gset:Nn \g_@@_finale_tl { #1 \check@checksum }
        \init@checksum
        \@esphack
      }
      { #1 \endinput }
  }
\DeclareDocumentCommand \Finale { }
  { \tl_use:N \g_@@_finale_tl }
\tl_new:N \g_@@_finale_tl
%    \end{macrocode}
% \end{variable}
% \end{macro}
%
% \begin{macro}{\@@_input:n}
%   Inputting a file, with some setup: the module name should be empty
%   before the first |<@@=|\meta{module}|>| line in the file.
%    \begin{macrocode}
\cs_new_protected:Npn \@@_input:n #1
  {
    \tl_gclear:N \g_@@_module_name_tl
    \MakePercentIgnore
    \input{#1}
    \MakePercentComment
  }
%    \end{macrocode}
% \end{macro}
%
% \begin{macro}{\DocInput}
%   Modified from \pkg{doc} to accept comma-list input (who has commas
%   in filenames?).
%    \begin{macrocode}
\DeclareDocumentCommand \DocInput { m }
  {
    \clist_map_inline:nn {#1}
      {
        \clist_put_right:Nn \g_docinput_clist {##1}
        \@@_input:n {##1}
      }
  }
%    \end{macrocode}
% \end{macro}
%
% \begin{macro}{\DocInputAgain}
%   Uses \cs{g_docinput_clist} to re-input whatever's already been
%   \tn{DocInput}-ed until now.  May be used multiple times.
%    \begin{macrocode}
\DeclareDocumentCommand \DocInputAgain { }
  { \clist_map_function:NN \g_docinput_clist \@@_input:n }
%    \end{macrocode}
% \end{macro}
%
% \begin{macro}{\DocInclude}
%   More or less exactly the same as \tn{include}, but uses
%   \tn{DocInput} on a \file{.dtx} file, not \tn{input} on a \file{.tex}
%   file.
%
%    \begin{macrocode}
\DeclareDocumentCommand \DocInclude { m }
  {
    \relax\clearpage
    \docincludeaux
    \IfFileExists{#1.fdd}
      { \cs_set:Npn \currentfile{#1.fdd} }
      { \cs_set:Npn \currentfile{#1.dtx} }
    \int_compare:nNnTF \@auxout = \@partaux
      { \@latexerr{\string\include\space cannot~be~nested}\@eha }
      { \@docinclude #1 }
  }
%    \end{macrocode}
%
%    \begin{macrocode}
\cs_gset:Npn \@docinclude #1
  {
    \clearpage
    \immediate\write\@mainaux{\string\@input{#1.aux}}
    \@tempswatrue
    \if@partsw
      \@tempswafalse
      \clist_if_in:NnT \@partlist {#1}
        { \@tempswatrue }
    \fi
    \if@tempswa
      \cs_set_eq:NN \@auxout                 \@partaux
      \immediate\openout\@partaux #1.aux
      \immediate\write\@partaux{\relax}
      \cs_set_eq:NN \@ltxdoc@PrintIndex      \PrintIndex
      \cs_set_eq:NN \PrintIndex              \relax
      \cs_set_eq:NN \@ltxdoc@PrintChanges    \PrintChanges
      \cs_set_eq:NN \PrintChanges            \relax
      \cs_set_eq:NN \@ltxdoc@theglossary     \theglossary
      \cs_set_eq:NN \@ltxdoc@endtheglossary  \endtheglossary
      \part{\currentfile}
      {
        \cs_set_eq:NN \ttfamily\relax
        \cs_gset:Npx \filekey
          { \filekey, \thepart = { \ttfamily \currentfile } }
      }
      \DocInput{\currentfile}
      \cs_set_eq:NN \PrintIndex              \@ltxdoc@PrintIndex
      \cs_set_eq:NN \PrintChanges            \@ltxdoc@PrintChanges
      \cs_set_eq:NN \theglossary             \@ltxdoc@theglossary
      \cs_set_eq:NN \endtheglossary          \@ltxdoc@endtheglossary
      \clearpage
      \@writeckpt{#1}
      \immediate \closeout \@partaux
    \else
      \@nameuse{cp@#1}
    \fi
    \cs_set_eq:NN \@auxout \@mainaux
  }
%    \end{macrocode}
%
%    \begin{macrocode}
\cs_gset:Npn \codeline@wrindex #1
  {
    \immediate\write\@indexfile
      {
        \string\indexentry{#1}
          { \filesep \int_use:N \c@CodelineNo }
      }
  }
\tl_gclear:N \filesep
%    \end{macrocode}
% \end{macro}
%
% \begin{macro}{\docincludeaux}
%    \begin{macrocode}
\cs_gset:Npn \docincludeaux
  {
    \tl_set:Nn \thepart { \alphalph { part } }
    \tl_set:Nn \filesep { \thepart - }
    \cs_set_eq:NN \filekey \use_none:n
    \tl_gput_right:Nn \index@prologue
      {
        \cs_gset:Npn \@oddfoot
          {
            \parbox { \textwidth }
              {
                \strut \footnotesize
                \raggedright { \bfseries File~Key: } ~ \filekey
              }
          }
        \cs_set_eq:NN \@evenfoot \@oddfoot
      }
    \cs_gset_eq:NN \docincludeaux \relax
    \cs_gset:Npn \@oddfoot
      {
        \cs_if_exist:cTF { ver @ \currentfile }
          { File~\thepart :~{\ttfamily\currentfile}~ }
          {
            \GetFileInfo{\currentfile}
            File~\thepart :~{\ttfamily\filename}~
            Date:~\ExplFileDate\ % space
            Version~\ExplFileVersion
          }
        \hfill \thepage
      }
    \cs_set_eq:NN \@evenfoot \@oddfoot
  }
%    \end{macrocode}
% \end{macro}
%
% \subsubsection{The \env{macrocode} environment}
%
% \begin{macro}{\xmacro@code, \@@_xmacro_code:n, \@@_xmacro_code:w}
%   Hook into the \texttt{macrocode} environment in a dirty way:
%   \tn{xmacro@code} is responsible for grabbing (and tokenizing) the
%   body of the environment.  Redefine it to pass what it grabs to
%   \cs{@@_xmacro_code:n}.  This new macro replaces all |@@| by the
%   appropriate module name.  One exceptional case is the
%   |<@@=|\meta{module}|>| lines themselves, where |@@| should not be
%   modified.  Actually, we search for such lines, to set the module
%   name automatically.  We need to be careful: no |<@@=| should appear
%   as such in the code below since \pkg{l3doc} is also typeset using
%   this code.
%   At each |<@@=| found, replace the \meta{module} in the code behind
%   it, update the \meta{module}, and loop to check for further
%   occurrences of |<@@=|.
%    \begin{macrocode}
\group_begin:
  \char_set_catcode_other:N \^^A
  \char_set_catcode_active:N \^^S
  \char_set_catcode_active:N \^^B
  \char_set_catcode_other:N \^^L
  \char_set_catcode_other:N \^^R
  \char_set_lccode:nn { `\^^A } { `\% }
  \char_set_lccode:nn { `\^^S } { `\  }
  \char_set_lccode:nn { `\^^B } { `\\ }
  \char_set_lccode:nn { `\^^L } { `\{ }
  \char_set_lccode:nn { `\^^R } { `\} }
  \tex_lowercase:D
    {
      \group_end:
      \cs_set_protected:Npn \xmacro@code
          #1 ^^A ^^S^^S^^S^^S ^^Bend ^^Lmacrocode^^R
        { \@@_xmacro_code:n {#1} \end{macrocode} }
    }
\group_begin:
  \char_set_catcode_active:N \<
  \char_set_catcode_active:N \>
  \char_set_catcode_other:N \%
  \char_set_catcode_comment:N \/
  \cs_new_protected:Npn \@@_xmacro_code:n #1
    {
      \tl_clear:N \l_@@_tmpa_tl
      \tl_if_in:nnTF {#1} { % < @ @ = }
        { \@@_xmacro_code:w #1 % < @ @ = \q_recursion_tail > \q_recursion_stop }
        {
          \tl_set:Nn \l_@@_tmpa_tl {#1}
          \@@_detect_internals:N \l_@@_tmpa_tl
          \@@_replace_at_at:N \l_@@_tmpa_tl
          \tl_use:N \l_@@_tmpa_tl
        }
    }
  \cs_new_protected:Npn \@@_xmacro_code:w #1 % < @ @ = #2 >
    {
      // Add code before <@@@@=...>
      \tl_set:Nn \l_@@_tmpb_tl {#1}
      \@@_detect_internals:N \l_@@_tmpb_tl
      \@@_replace_at_at:N \l_@@_tmpb_tl
      \tl_put_right:NV \l_@@_tmpa_tl \l_@@_tmpb_tl
      // Check for \q_recursion_tail
      \quark_if_recursion_tail_stop_do:nn {#2}
        { \tl_use:N \l_@@_tmpa_tl }
      // Change module name and add <@@@@=#2> to typeset output
      \tl_gset:Nn \g_@@_module_name_tl {#2}
      \tl_put_right:Nn \l_@@_tmpa_tl { < \text { \verbatim@font @ @ = #2 } > }
      // Loop
      \@@_xmacro_code:w
    }
\group_end:
%    \end{macrocode}
% \end{macro}
%
% \subsection{At end document}
%
% Print all defined and documented macros/functions.
%
%    \begin{macrocode}
\iow_new:N \g_@@_func_iow
%    \end{macrocode}
%
%    \begin{macrocode}
\tl_new:N \l_@@_doc_def_tl
\tl_new:N \l_@@_doc_undef_tl
\tl_new:N \l_@@_undoc_def_tl
%    \end{macrocode}
%
%    \begin{macrocode}
\cs_new_protected:Npn \@@_show_functions_defined:
  {
    \bool_lazy_and:nnT
      { \g_@@_typeset_implementation_bool } { \g_@@_checkfunc_bool }
      {
        \iow_term:x { \c_@@_iow_separator_tl \iow_newline: }
        \iow_open:Nn \g_@@_func_iow { \c_sys_jobname_str .cmds }

        \tl_clear:N \l_@@_doc_def_tl
        \tl_clear:N \l_@@_doc_undef_tl
        \tl_clear:N \l_@@_undoc_def_tl
        \seq_map_inline:Nn \g_doc_functions_seq
          {
            \seq_if_in:NnTF \g_doc_macros_seq {##1}
              {
                \tl_put_right:Nx \l_@@_doc_def_tl
                  { ##1 \iow_newline: }
                \iow_now:Nn \g_@@_func_iow { > ~ ##1 }
              }
              {
                \tl_put_right:Nx \l_@@_doc_undef_tl
                  { ##1 \iow_newline: }
                \iow_now:Nn \g_@@_func_iow { ! ~ ##1 }
              }
          }
        \seq_map_inline:Nn \g_doc_macros_seq
          {
            \seq_if_in:NnF \g_doc_functions_seq {##1}
              {
                \tl_put_right:Nx \l_@@_undoc_def_tl
                  { ##1 \iow_newline: }
                \iow_now:Nn \g_@@_func_iow { ? ~ ##1 }
              }
          }
        \@@_functions_typeout:nN
          {
            Functions~both~documented~and~defined: \iow_newline:
            (In~order~of~being~documented)
          }
          \l_@@_doc_def_tl
        \@@_functions_typeout:nN
          { Functions~documented~but~not~defined: }
          \l_@@_doc_undef_tl
        \@@_functions_typeout:nN
          { Functions~defined~but~not~documented: }
          \l_@@_undoc_def_tl

        \iow_close:N \g_@@_func_iow
        \iow_term:x { \c_@@_iow_separator_tl }
      }
  }
\AtEndDocument { \@@_show_functions_defined: }
%    \end{macrocode}
%
% TODO: use \cs{iow_term:x}.
%    \begin{macrocode}
\cs_new_protected:Npn \@@_functions_typeout:nN #1#2
  {
    \tl_if_empty:NF #2
      {
        \typeout
          {
            \c_@@_iow_midrule_tl \iow_newline:
            #1 \iow_newline:
            \c_@@_iow_midrule_tl \iow_newline:
            #2
          }
        \tl_clear:N #2
      }
  }
%    \end{macrocode}
%
%    \begin{macrocode}
\cs_new_protected:Npn \@@_show_not_tested:
  {
    \bool_if:NT \g_@@_checktest_bool
      {
        \tl_clear:N \l_@@_tmpa_tl
        \prop_if_empty:NF \g_@@_missing_tests_prop
          {
            \cs_set:Npn \@@_tmpa:w ##1##2
              {
                \iow_newline:
                \space\space\space\space \exp_not:n {##1}
                \clist_map_function:nN {##2} \@@_tmpb:w
              }
            \cs_set:Npn \@@_tmpb:w ##1
              {
                \iow_newline:
                \space\space\space\space\space\space * ~ ##1
              }
            \tl_put_right:Nx \l_@@_tmpa_tl
              {
                \iow_newline: \iow_newline:
                The~ following~ macro(s)~ have~ incomplete~ tests:
                \iow_newline:
                \prop_map_function:NN
                  \g_@@_missing_tests_prop \@@_tmpa:w
              }
          }
        \seq_if_empty:NF \g_@@_not_tested_seq
          {
            \cs_set:Npn \@@_tmpa:w ##1
              { \clist_map_function:nN {##1} \@@_tmpb:w }
            \cs_set:Npn \@@_tmpb:w ##1
              {
                \iow_newline:
                \space\space\space\space ##1
              }
            \tl_put_right:Nx \l_@@_tmpa_tl
              {
                \iow_newline:
                \iow_newline:
                The~ following~ macro(s)~ do~ not~ have~ any~ tests:
                \iow_newline:
                \seq_map_function:NN
                  \g_@@_not_tested_seq \@@_tmpa:w
              }
          }
        \tl_if_empty:NF \l_@@_tmpa_tl
          {
            \int_set:Nn \l_@@_tmpa_int { \tex_interactionmode:D }
            \errorstopmode
            \ClassError { l3doc } { \l_@@_tmpa_tl } { }
            \int_set:Nn \tex_interactionmode:D { \l_@@_tmpa_int }
          }
      }
  }
\AtEndDocument { \@@_show_not_tested: }
%    \end{macrocode}
%
% \subsection{Indexing}
%
% \subsubsection{Userspace commands}
%
% Fix index (for now):
%    \begin{macrocode}
\g@addto@macro \theindex { \MakePrivateLetters }
\cs_gset:Npn \verbatimchar {&}
%    \end{macrocode}
%
%    \begin{macrocode}
\setcounter { IndexColumns } { 2 }
%    \end{macrocode}
%
% Set up the Index to use \tn{part}
%    \begin{macrocode}
\IndexPrologue
  {
    \part*{Index}
    \markboth{Index}{Index}
    \addcontentsline{toc}{part}{Index}
    The~italic~numbers~denote~the~pages~where~the~
    corresponding~entry~is~described,~
    numbers~underlined~point~to~the~definition,~
    all~others~indicate~the~places~where~it~is~used.
  }
%    \end{macrocode}
%
% \begin{macro}{\SpecialIndex}
%   An attempt at affecting how commands which appear within the
%   \env{macrocode} environment are treated in the index.
%    \begin{macrocode}
\cs_gset_protected:Npn \SpecialIndex #1
  {
    \@bsphack
    \@@_special_index:nn {#1} { }
    \@esphack
  }
%    \end{macrocode}
% \end{macro}
%
%    \begin{macrocode}
\msg_new:nnn { l3doc } { print-index-howto }
  {
    Generate~the~index~by~executing\\
    \iow_indent:n
      { makeindex~-s~gind.ist~-o~\c_sys_jobname_str.ind~\c_sys_jobname_str.idx }
  }
\tl_gput_right:Nn \PrintIndex
  { \AtEndDocument { \msg_info:nn { l3doc } { print-index-howto } } }
%    \end{macrocode}
%
% \subsubsection{Internal index commands}
%
% \begin{macro}{\it@is@a}
%   The index of one-character commands within the \env{macrocode}
%   environment is produced using \tn{it@is@a} \meta{char}.  Alter that
%   command.
%    \begin{macrocode}
\cs_gset_protected:Npn \it@is@a #1
  {
    \use:x
      {
        \@@_special_index_module:nnnnN
          {#1}
          { \bslash #1 }
          { }
          { }
          \c_false_bool
      }
  }
%    \end{macrocode}
% \end{macro}
%
% \begin{macro}{\@@_special_index:nn}
% ^^A TODO this override is somewhat a hack
%    \begin{macrocode}
\cs_new_protected:Npn \@@_special_index:nn #1#2
  {
    \@@_key_get:n {#1}
    \quark_if_no_value:NF \l_@@_override_module_tl
      { \tl_set_eq:NN \l_@@_index_module_tl \l_@@_override_module_tl }
    \@@_special_index_module:ooonN
      { \l_@@_index_key_tl }
      { \l_@@_index_macro_tl }
      { \l_@@_index_module_tl }
      {#2}
      \l_@@_index_internal_bool
  }
\cs_generate_variant:Nn \@@_special_index:nn { o }
%    \end{macrocode}
% \end{macro}
%
% \begin{macro}
%   {
%     \@@_special_index_module:nnnnN,
%     \@@_special_index_module:ooonN,
%     \@@_special_index_aux:nnnnnn,
%     \@@_special_index_set:Nn,
%   }
%   Remotely based on Heiko's replacement to play nicely with
%   \pkg{hypdoc}.  We use \tn{verb} or a \tn{verbatim@font} construction
%   depending on whether the number of tokens in |#2| is equal to its
%   number of characters: if it is not then that suggests that there is
%   a construct such as |\meta{...}|.
%    \begin{macrocode}
\tl_new:N \l_@@_index_escaped_macro_tl
\tl_new:N \l_@@_index_escaped_key_tl
%    \end{macrocode}
%
%    \begin{macrocode}
\cs_new_protected:Npn \@@_special_index_module:nnnnN #1#2#3#4#5
%    \end{macrocode}
% \begin{arguments}
% \item key
% \item macro
% \item module
% \item index `type' (\texttt{main}/\texttt{usage}/\emph{etc.})
% \item boolean whether internal command
% \end{arguments}
%    \begin{macrocode}
  {
    \use:x
      {
        \exp_not:n { \@@_special_index_aux:nnnnnn {#1} {#2} }
          \tl_if_empty:nTF {#3}
            { { } { } { } }
            {
              \str_if_eq:nnTF {#3} { TeX }
                {
                  { TeX~and~LaTeX2e }
                  { \string\TeX{}~and~\string\LaTeXe{} }
                }
                {
                  {#3}
                  { \string\pkg{#3} }
                }
              { \bool_if:NT #5 { ~internal } ~commands: }
            }
      }
          {#4}
  }
%    \end{macrocode}
%
%    \begin{macrocode}
\cs_generate_variant:Nn \@@_special_index_module:nnnnN { ooo }
%    \end{macrocode}
%
%    \begin{macrocode}
\cs_new_protected:Npn \@@_special_index_aux:nnnnnn #1#2#3#4#5#6
%    \end{macrocode}
% \begin{arguments}
% \item key
% \item macro
% \item index subheading string
% \item index subheading text
% \item index subheading suffix (appended to both arg 3 and 4)
% \item index `type' (\texttt{main}/\texttt{usage}/\emph{etc.})
% \end{arguments}
%    \begin{macrocode}
  {
    \tl_set:Nn \l_@@_index_escaped_key_tl {#1}
    \@@_quote_special_char:N \l_@@_index_escaped_key_tl
    \@@_special_index_set:Nn \l_@@_index_escaped_macro_tl {#2}
    \str_if_eq:onTF { \@currenvir } { macrocode }
      { \codeline@wrindex }
      {
        \str_case:nnF {#6}
          {
            { main }  { \codeline@wrindex }
            { usage } { \index }
          }
          { \HD@target \index }
      }
      {
        \tl_if_empty:nF { #3 #4 #5 }
          { #3 #5 \actualchar #4 #5 \levelchar }
        \l_@@_index_escaped_key_tl
        \actualchar
        {
          \token_to_str:N \verbatim@font \c_space_tl
          \l_@@_index_escaped_macro_tl
        }
        \encapchar
        hdclindex{\the\c@HD@hypercount}{#6}
      }
  }
%    \end{macrocode}
%
%    \begin{macrocode}
\cs_new_protected:Npn \@@_special_index_set:Nn #1#2
  {
    \tl_set:Nx #1 { \tl_to_str:n {#2} }
    \@@_if_almost_str:nTF {#2}
      {
        \tl_replace_all:Non #1 { \tl_to_str:n { __ } }
          {
            \verbatimchar
            \token_to_str:N \_ \token_to_str:N \_
            \token_to_str:N \verb * \verbatimchar
          }
        \exp_args:Nx \tl_map_inline:nn
          { \tl_to_str:N \verbatimchar \token_to_str:N _ }
          {
            \tl_replace_all:Nnn #1 {##1}
              {
                \verbatimchar \c_backslash_str ##1
                \token_to_str:N \verb * \verbatimchar
              }
          }
        \tl_set:Nx #1
          {
            \token_to_str:N \verb * \verbatimchar
            #1 \verbatimchar
          }
      }
      {
        \tl_set:Nn #1 {#2}
        \tl_replace_all:Non #1
          { \c_backslash_str }
          { \token_to_str:N \bslash \c_space_tl }
      }
    \@@_quote_special_char:N #1
  }
%    \end{macrocode}
% \end{macro}
%
% \begin{macro}{\@@_quote_special_char:N}
% Quote some special characters.
%    \begin{macrocode}
\cs_new_protected:Npn \@@_quote_special_char:N #1
  {
    \tl_map_inline:nn { \quotechar \actualchar \encapchar \levelchar \bslash }
      {
        \tl_replace_all:Nxn #1
          { \tl_to_str:N ##1 } { \quotechar \tl_to_str:N ##1 }
      }
  }
%    \end{macrocode}
% \end{macro}
%
% \subsubsection{Finding sort-key and module}
%
% \begin{macro}{\@@_key_get:n}
%   Sets \cs{l_@@_index_macro_tl}, \cs{l_@@_index_key_tl}, and
%   \cs{l_@@_index_module_tl} from |#1|.  The base function is stored by
%   \cs{@@_key_get_base:nN} in \cs{l_@@_index_macro_tl}, falling back to
%   |#1| if it contains markup or has no signature.
%
%   The starting point for the \meta{key} is \cs{l_@@_index_key_tl} as a
%   string.  If it the first character is a backslash, remove
%   it.  Then recognize |expl| functions and variables by the presence
%   of |:| or~|_| and \TeX{}/\LaTeXe{} commands by the presence of~|@|.
%   For |expl| names, we call \cs{@@_key_func:} or \cs{@@_key_var:},
%   which are responsible for removing some characters and finding the
%   module name, while for \TeX{}/\LaTeXe{} commands the module name is
%   |TeX|, and others have an empty module name.
%    \begin{macrocode}
\cs_new_protected:Npn \@@_key_get:n #1
  {
    \@@_key_get_base:nN {#1} \l_@@_index_macro_tl
    \tl_set:Nx \l_@@_index_key_tl
      { \tl_to_str:N \l_@@_index_macro_tl }
    \tl_clear:N \l_@@_index_module_tl
    \tl_if_in:NoTF \l_@@_index_key_tl { \tl_to_str:n { __ } }
      { \bool_set_true:N \l_@@_index_internal_bool }
      { \bool_set_false:N \l_@@_index_internal_bool }
    \exp_last_unbraced:NNo
    \tl_if_head_eq_charcode:oNT
      { \l_@@_index_key_tl } \c_backslash_str
      { \@@_key_pop: }
    \tl_if_in:NoTF \l_@@_index_key_tl { \token_to_str:N : }
      { \@@_key_func: }
      {
        \tl_if_in:NoTF \l_@@_index_key_tl { \token_to_str:N _ }
          { \@@_key_var: }
          {
            \tl_if_in:NoT \l_@@_index_key_tl { \token_to_str:N @ }
              { \tl_set:Nn \l_@@_index_module_tl { TeX } }
          }
      }
  }
\cs_new_protected:Npn \@@_key_pop:
  {
    \tl_set:Nx \l_@@_index_key_tl
      { \tl_tail:N \l_@@_index_key_tl }
  }
%    \end{macrocode}
% \end{macro}
%
% \begin{macro}{\@@_key_trim_module:n, \@@_key_drop_underscores:}
%   Helper that removes from \cs{l_@@_index_module_tl} everything after
%   the first occurence of |#1|.  Helper that removes any leading
%   underscore from \cs{l_@@_index_key_tl}.
%    \begin{macrocode}
\cs_new_protected:Npn \@@_key_trim_module:n #1
  {
    \cs_set:Npn \@@_tmpa:w ##1 #1 ##2 \q_stop
      { \exp_not:n {##1} }
    \tl_set:Nx \l_@@_index_module_tl
      { \exp_after:wN \@@_tmpa:w \l_@@_index_module_tl #1 \q_stop }
  }
\cs_new_protected:Npn \@@_key_drop_underscores:
  {
    \tl_if_head_eq_charcode:oNT { \l_@@_index_key_tl } _
      { \@@_key_pop: \@@_key_drop_underscores: }
  }
%    \end{macrocode}
% \end{macro}
%
% \begin{macro}{\@@_key_func:}
%   The function \cs{@@_key_func:} is used if there is a colon, so
%   either for usual \pkg{expl3} functions or for keys from
%   \pkg{l3keys}.  After removing from the key a leading dot (for the
%   latter case), and any leading underscore, the module name is the
%   part before any colon or underscore.
%    \begin{macrocode}
\cs_new_protected:Npn \@@_key_func:
  {
    \tl_if_head_eq_charcode:oNT { \l_@@_index_key_tl } .
      { \@@_key_pop: }
    \@@_key_drop_underscores:
    \tl_set_eq:NN \l_@@_index_module_tl \l_@@_index_key_tl
    \exp_args:No \@@_key_trim_module:n { \token_to_str:N : }
    \exp_args:No \@@_key_trim_module:n { \token_to_str:N _ }
  }
%    \end{macrocode}
% \end{macro}
%
% \begin{macro}{\@@_key_var:, \@@_key_get_module:}
%   The function \cs{@@_key_var:} covers cases with no~|:| but with~|_|,
%   typically variables but occasionally non-\pkg{expl3} functions such
%   as \Lua{} function with underscores.  First test the second
%   character: if that is~|_| then assume we have a proper variable,
%   otherwise use the part before any underscore as the module name.
%   For variables, distinguish quarks and scan marks (starting with |q|
%   and~|s|), then drop the first letter (local/global/constant marker)
%   and underscores.  If there is no underscore left we had something
%   like \cs{c_zero} which we assume is an integer constant.  If there
%   is one underscore we assume it is a variable like \cs{c_empty_tl}
%   whose module name is the last part.  Otherwise the module name is
%   the part before any underscore.
%    \begin{macrocode}
\cs_new_protected:Npn \@@_key_var:
  {
    \exp_args:Nx \tl_if_head_eq_charcode:nNTF
      { \exp_args:No \str_tail:n \l_@@_index_key_tl } _
      {
        \str_case:fn { \str_head:N \l_@@_index_key_tl }
          {
            { q } { \tl_set:Nn \l_@@_index_module_tl { quark } }
            { s } { \tl_set:Nn \l_@@_index_module_tl { quark } }
          }
        \@@_key_pop:
        \@@_key_pop:
        \@@_key_drop_underscores:
        \tl_if_empty:NT \l_@@_index_module_tl
          {
            \seq_set_split:NoV \l_@@_tmpa_seq
              { \token_to_str:N _ } \l_@@_index_key_tl
            \tl_set:Nx \l_@@_index_module_tl
              {
                \int_case:nnF { \seq_count:N \l_@@_tmpa_seq }
                  {
                    { 0 } { }
                    { 1 } { int }
                    { 2 } { \seq_item:Nn \l_@@_tmpa_seq { 2 } }
                  }
                  { \seq_item:Nn \l_@@_tmpa_seq { 1 } }
              }
          }
      }
      {
        \tl_set_eq:NN \l_@@_index_module_tl \l_@@_index_key_tl
        \exp_args:No \@@_key_trim_module:n { \token_to_str:N _ }
      }
  }
%    \end{macrocode}
% \end{macro}
%
% \subsection{Change history}
%
% Set the change history to use \tn{part}.
% Allow control names to be hyphenated in here\dots
%    \begin{macrocode}
\GlossaryPrologue
  {
    \part*{Change~History}
    {\GlossaryParms\ttfamily\hyphenchar\font=`\-}
    \markboth{Change~History}{Change~History}
    \addcontentsline{toc}{part}{Change~History}
  }
%    \end{macrocode}
%
%    \begin{macrocode}
\msg_new:nnn { l3doc } { print-changes-howto }
  {
    Generate~the~change~list~by~executing\\
    \iow_indent:n
      { makeindex~-s~gglo.ist~-o~\c_sys_jobname_str.gls~\c_sys_jobname_str.glo }
  }
\tl_gput_right:Nn \PrintChanges
  { \AtEndDocument { \msg_info:nn { l3doc } { print-changes-howto } } }
%    \end{macrocode}
%
%^^A The standard \changes command modified slightly to better cope with
%^^A this multiple file document.
%^^A\def\changes@#1#2#3{%
%^^A  \let\protect\@unexpandable@protect
%^^A  \edef\@tempa{\noexpand\glossary{#2\space\currentfile\space#1\levelchar
%^^A                                 \ifx\saved@macroname\@empty
%^^A                                   \space
%^^A                                   \actualchar
%^^A                                   \generalname
%^^A                                 \else
%^^A                                   \expandafter\@gobble
%^^A                                   \saved@macroname
%^^A                                   \actualchar
%^^A                                   \string\verb\quotechar*%
%^^A                                   \verbatimchar\saved@macroname
%^^A                                   \verbatimchar
%^^A                                 \fi
%^^A                                 :\levelchar #3}}%
%^^A  \@tempa\endgroup\@esphack}
%
% \subsection{Default configuration}
%
%    \begin{macrocode}
\bool_if:NTF \g_@@_typeset_implementation_bool
  {
    \RecordChanges
    \CodelineIndex
    \EnableCrossrefs
    \AlsoImplementation
  }
  {
    \CodelineNumbered
    \DisableCrossrefs
    \OnlyDescription
  }
%    \end{macrocode}
%
%    \begin{macrocode}
%</package>
%    \end{macrocode}
%
% \subsection{Makeindex configuration}
%
%    \begin{macrocode}
%<*docist>
%    \end{macrocode}
%
% The makeindex style \file{l3doc.ist} is used in place of the usual
% \file{gind.ist} to ensure that |I| is used in the sequence |I J K| not
% |I II II|, which would be the default makeindex behaviour.
%
% Will: Do we need this?
%
% Frank: at the moment we do not distribute or generate this file.
%        \file{gind.ist} is used instead.
%
% \begin{macro}[do-not-index={\\,\n}]{}
%    \begin{macrocode}
actual '='
quote '!'
level '>'
preamble
"\n \\begin{theindex} \n \\makeatletter\\scan@allowedfalse\n"
postamble
"\n\n \\end{theindex}\n"
item_x1   "\\efill \n \\subitem "
item_x2   "\\efill \n \\subsubitem "
delim_0   "\\pfill "
delim_1   "\\pfill "
delim_2   "\\pfill "
% The next lines will produce some warnings when
% running Makeindex as they try to cover two different
% versions of the program:
lethead_prefix   "{\\bfseries\\hfil "
lethead_suffix   "\\hfil}\\nopagebreak\n"
lethead_flag       1
heading_prefix   "{\\bfseries\\hfil "
heading_suffix   "\\hfil}\\nopagebreak\n"
headings_flag       1

% and just for source3:
% Remove R so I is treated in sequence I J K not I II III
page_precedence "rnaA"
%    \end{macrocode}
% \end{macro}
%
%    \begin{macrocode}
%</docist>
%    \end{macrocode}
%
% \end{implementation}
%
% \PrintIndex
