% \iffalse meta-comment
%
%% File: l3fp-aux.dtx
%
% Copyright (C) 2011-2019 The LaTeX3 Project
%
% It may be distributed and/or modified under the conditions of the
% LaTeX Project Public License (LPPL), either version 1.3c of this
% license or (at your option) any later version.  The latest version
% of this license is in the file
%
%    https://www.latex-project.org/lppl.txt
%
% This file is part of the "l3kernel bundle" (The Work in LPPL)
% and all files in that bundle must be distributed together.
%
% -----------------------------------------------------------------------
%
% The development version of the bundle can be found at
%
%    https://github.com/latex3/latex3
%
% for those people who are interested.
%
%<*driver>
\documentclass[full,kernel]{l3doc}
\begin{document}
  \DocInput{\jobname.dtx}
\end{document}
%</driver>
% \fi
%
% \title{^^A
%   The \textsf{l3fp-aux} package\\ Support for floating points^^A
% }
%
% \author{^^A
%  The \LaTeX3 Project\thanks
%    {^^A
%      E-mail:
%        \href{mailto:latex-team@latex-project.org}
%          {latex-team@latex-project.org}^^A
%    }^^A
% }
%
% \date{Released 2019-10-02}
%
% \maketitle
%
% \begin{documentation}
%
% \end{documentation}
%
% \begin{implementation}
%
% \section{\pkg{l3fp-aux} implementation}
%
%    \begin{macrocode}
%<*initex|package>
%    \end{macrocode}
%
%    \begin{macrocode}
%<@@=fp>
%    \end{macrocode}
%
% ^^A todo: make sanitize and pack more homogeneous between modules.
%
% \subsection{Access to primitives}
%
% \begin{macro}{\@@_int_eval:w, \@@_int_eval_end:, \@@_int_to_roman:w}
%   Largely for performance reasons, we need to directly access primitives
%   rather than use \cs{int_eval:n}. This happens \emph{a lot}, so we
%   use private names. The same is true for \tn{romannumeral}, although it
%   is used much less widely.
%    \begin{macrocode}
\cs_new_eq:NN \@@_int_eval:w \tex_numexpr:D
\cs_new_eq:NN \@@_int_eval_end: \scan_stop:
\cs_new_eq:NN \@@_int_to_roman:w \tex_romannumeral:D
%    \end{macrocode}
% \end{macro}
%
% \subsection{Internal representation}
%
% Internally, a floating point number \meta{X} is a
% token list containing
% \begin{quote}
%   \cs{s_@@} \cs{@@_chk:w} \meta{case} \meta{sign} \meta{body} |;|
% \end{quote}
% Let us explain each piece separately.
%
% Internal floating point numbers are used in expressions,
% and in this context are subject to \texttt{f}-expansion. They must
% leave a recognizable mark after \texttt{f}-expansion, to prevent the
% floating point number from being re-parsed. Thus, \cs{s_@@}
% is simply another name for \tn{relax}.
%
% When used directly without an accessor function, floating points
% should produce an error: this is the role of \cs{@@_chk:w}.  We could
% make floating point variables be protected to prevent them from
% expanding under \texttt{x}-expansion, but it seems more convenient to
% treat them as a subcase of token list variables.
%
% The (decimal part of the) IEEE-754-2008 standard requires the format
% to be able to represent special floating point numbers besides the
% usual positive and negative cases. We distinguish the various
% possibilities by their \meta{case}, which is a single digit:
% \begin{itemize}
% \item[0] zeros: |+0| and |-0|,
% \item[1] \enquote{normal} numbers (positive and negative),
% \item[2] infinities: |+inf| and |-inf|,
% \item[3] quiet and signalling \texttt{nan}.
% \end{itemize}
% The \meta{sign} is |0| (positive) or |2| (negative),
% except in the case of \texttt{nan}, which have $\meta{sign} = 1$.
% This ensures that changing the \meta{sign} digit to $2-\meta{sign}$
% is exactly equivalent to changing the sign of the number.
%
% Special floating point numbers have the form
% \begin{quote}
%   \cs{s_@@} \cs{@@_chk:w} \meta{case} \meta{sign} \cs[no-index]{s_@@_\ldots} |;|
% \end{quote}
% where \cs[no-index]{s_@@_\ldots} is a scan mark carrying information about how the
% number was formed (useful for debugging).
%
% Normal floating point numbers ($\meta{case} = 1$) have the form
% \begin{quote}
%   \cs{s_@@} \cs{@@_chk:w} 1 \meta{sign} \Arg{exponent}
%   \Arg{X_1} \Arg{X_2} \Arg{X_3} \Arg{X_4} |;|
% \end{quote}
% Here, the \meta{exponent} is an integer, between
% $-\ExplSyntaxOn\int_use:N\c__fp_minus_min_exponent_int$ and
% $\ExplSyntaxOn\int_use:N\c__fp_max_exponent_int$.  The body consists
% in four blocks of exactly $4$ digits,
% $0000 \leq \meta{X_i} \leq 9999$, and the floating point is
% \[
% (-1)^{\meta{sign}/2} \meta{X_1}\meta{X_2}\meta{X_3}\meta{X_4}\cdot 10^{\meta{exponent}-16}
% \]
% where we have concatenated the $16$ digits.  Currently, floating point numbers are normalized such that
% the \meta{exponent} is minimal, in other words, $1000 \leq \meta{X_1} \leq 9999$.
%
% \begin{table}\centering
%   \caption{Internal representation of floating point numbers.}
%   \label{tab:fp-convert-special}
%   \begin{tabular}{ll}
%     \toprule
%     \multicolumn{1}{c}{Representation} & Meaning \\
%     \midrule
%     0 0 \cs[no-index]{s_@@_\ldots}  \texttt{;} & Positive zero.      \\
%     0 2 \cs[no-index]{s_@@_\ldots}  \texttt{;} & Negative zero.      \\
%     1 0 \Arg{exponent} \Arg{X_1} \Arg{X_2} \Arg{X_3} \Arg{X_4} \texttt{;}
%                                   & Positive floating point. \\
%     1 2 \Arg{exponent} \Arg{X_1} \Arg{X_2} \Arg{X_3} \Arg{X_4} \texttt{;}
%                                   & Negative floating point. \\
%     2 0 \cs[no-index]{s_@@_\ldots}  \texttt{;} & Positive infinity.  \\
%     2 2 \cs[no-index]{s_@@_\ldots}  \texttt{;} & Negative infinity.  \\
%     3 1 \cs[no-index]{s_@@_\ldots}  \texttt{;} & Quiet \texttt{nan}.        \\
%     3 1 \cs[no-index]{s_@@_\ldots}  \texttt{;} & Signalling \texttt{nan}.   \\
%     \bottomrule
%   \end{tabular}
% \end{table}
%
% Calculations are done in base $10000$, \emph{i.e.} one myriad.
%
% \subsection{Using arguments and semicolons}
%
% \begin{macro}[EXP]{\@@_use_none_stop_f:n}
%   This function removes an argument (typically a digit) and replaces
%   it by \cs{exp_stop_f:}, a marker which stops \texttt{f}-type
%   expansion.
%    \begin{macrocode}
\cs_new:Npn \@@_use_none_stop_f:n #1 { \exp_stop_f: }
%    \end{macrocode}
% \end{macro}
%
% \begin{macro}[EXP]{\@@_use_s:n, \@@_use_s:nn}
%   Those functions place a semicolon after one or two arguments
%   (typically digits).
%    \begin{macrocode}
\cs_new:Npn \@@_use_s:n #1 { #1; }
\cs_new:Npn \@@_use_s:nn #1#2 { #1#2; }
%    \end{macrocode}
% \end{macro}
%
% \begin{macro}[EXP]
%   {\@@_use_none_until_s:w, \@@_use_i_until_s:nw, \@@_use_ii_until_s:nnw}
%   Those functions select specific arguments among a set of arguments
%   delimited by a semicolon.
%    \begin{macrocode}
\cs_new:Npn \@@_use_none_until_s:w #1; { }
\cs_new:Npn \@@_use_i_until_s:nw #1#2; {#1}
\cs_new:Npn \@@_use_ii_until_s:nnw #1#2#3; {#2}
%    \end{macrocode}
% \end{macro}
%
% \begin{macro}[EXP]{\@@_reverse_args:Nww}
%   Many internal functions take arguments delimited by semicolons, and
%   it is occasionally useful to swap two such arguments.
%    \begin{macrocode}
\cs_new:Npn \@@_reverse_args:Nww #1 #2; #3; { #1 #3; #2; }
%    \end{macrocode}
% \end{macro}
%
% \begin{macro}[EXP]{\@@_rrot:www}
%   Rotate three arguments delimited by semicolons.  This is the inverse
%   (or the square) of the Forth primitive |ROT|, hence the name.
%    \begin{macrocode}
\cs_new:Npn \@@_rrot:www #1; #2; #3; { #2; #3; #1; }
%    \end{macrocode}
% \end{macro}
%
% \begin{macro}[EXP]{\@@_use_i:ww, \@@_use_i:www}
%   Many internal functions take arguments delimited by semicolons, and
%   it is occasionally useful to remove one or two such arguments.
%    \begin{macrocode}
\cs_new:Npn \@@_use_i:ww #1; #2; { #1; }
\cs_new:Npn \@@_use_i:www #1; #2; #3; { #1; }
%    \end{macrocode}
% \end{macro}
%
% \subsection{Constants, and structure of floating points}
%
% \begin{macro}{\@@_misused:n}
%   This receives a floating point object (floating point number or
%   tuple) and generates an error stating that it was misused.  This is
%   called when for instance an |fp| variable is left in the input
%   stream and its contents reach \TeX{}'s stomach.
%    \begin{macrocode}
\cs_new_protected:Npn \@@_misused:n #1
  { \__kernel_msg_error:nnx { kernel } { misused-fp } { \fp_to_tl:n {#1} } }
%    \end{macrocode}
% \end{macro}
%
% \begin{macro}{\s_@@, \@@_chk:w}
%   Floating points numbers all start with \cs{s_@@} \cs{@@_chk:w},
%   where \cs{s_@@} is equal to the \TeX{} primitive \tn{relax}, and
%   \cs{@@_chk:w} is protected.  The rest of the floating point number
%   is made of characters (or \tn{relax}).  This ensures that nothing
%   expands under \texttt{f}-expansion, nor under \texttt{x}-expansion.
%   However, when typeset, \cs{s_@@} does nothing, and \cs{@@_chk:w} is
%   expanded.  We define \cs{@@_chk:w} to produce an error.
%    \begin{macrocode}
\scan_new:N \s_@@
\cs_new_protected:Npn \@@_chk:w #1 ;
  { \@@_misused:n { \s_@@ \@@_chk:w #1 ; } }
%    \end{macrocode}
% \end{macro}
%
% \begin{macro}{\s_@@_mark, \s_@@_stop}
%   Aliases of \cs{tex_relax:D}, used to terminate expressions.
%    \begin{macrocode}
\scan_new:N \s_@@_mark
\scan_new:N \s_@@_stop
%    \end{macrocode}
% \end{macro}
%
% \begin{macro}
%   {
%     \s_@@_invalid, \s_@@_underflow, \s_@@_overflow,
%     \s_@@_division, \s_@@_exact
%   }
%   A couple of scan marks used to indicate where special floating point
%   numbers come from.
%    \begin{macrocode}
\scan_new:N \s_@@_invalid
\scan_new:N \s_@@_underflow
\scan_new:N \s_@@_overflow
\scan_new:N \s_@@_division
\scan_new:N \s_@@_exact
%    \end{macrocode}
% \end{macro}
%
% \begin{variable}
%   {\c_zero_fp, \c_minus_zero_fp, \c_inf_fp, \c_minus_inf_fp, \c_nan_fp}
%   The special floating points. We define the floating points here as \enquote{exact}.
%    \begin{macrocode}
\tl_const:Nn \c_zero_fp       { \s_@@ \@@_chk:w 0 0 \s_@@_exact ; }
\tl_const:Nn \c_minus_zero_fp { \s_@@ \@@_chk:w 0 2 \s_@@_exact ; }
\tl_const:Nn \c_inf_fp        { \s_@@ \@@_chk:w 2 0 \s_@@_exact ; }
\tl_const:Nn \c_minus_inf_fp  { \s_@@ \@@_chk:w 2 2 \s_@@_exact ; }
\tl_const:Nn \c_nan_fp        { \s_@@ \@@_chk:w 3 1 \s_@@_exact ; }
%    \end{macrocode}
% \end{variable}
%
% \begin{variable}{\c_@@_prec_int, \c_@@_half_prec_int, \c_@@_block_int}
%   The number of digits of floating points.
%    \begin{macrocode}
\int_const:Nn \c_@@_prec_int { 16 }
\int_const:Nn \c_@@_half_prec_int { 8 }
\int_const:Nn \c_@@_block_int { 4 }
%    \end{macrocode}
% \end{variable}
%
% \begin{variable}{\c_@@_myriad_int}
%   Blocks have $4$~digits so this integer is useful.
%    \begin{macrocode}
\int_const:Nn \c_@@_myriad_int { 10000 }
%    \end{macrocode}
% \end{variable}
%
% \begin{variable}{\c_@@_minus_min_exponent_int, \c_@@_max_exponent_int}
%   Normal floating point numbers have an exponent between $-$
%   \texttt{minus_min_exponent} and \texttt{max_exponent} inclusive.
%   Larger numbers are rounded to $\pm\infty$.  Smaller numbers are
%   rounded to $\pm 0$.  It would be more natural to define a
%   \texttt{min_exponent} with the opposite sign but that would waste
%   one \TeX{} count.
%    \begin{macrocode}
\int_const:Nn \c_@@_minus_min_exponent_int { 10000 }
\int_const:Nn \c_@@_max_exponent_int { 10000 }
%    \end{macrocode}
% \end{variable}
%
% \begin{variable}{\c_@@_max_exp_exponent_int}
%   If a number's exponent is larger than that, its exponential
%   overflows/underflows.
%    \begin{macrocode}
\int_const:Nn \c_@@_max_exp_exponent_int { 5 }
%    \end{macrocode}
% \end{variable}
%
% \begin{variable}{\c_@@_overflowing_fp}
%   A floating point number that is bigger than all normal floating
%   point numbers.  This replaces infinities when converting to formats
%   that do not support infinities.
%    \begin{macrocode}
\tl_const:Nx \c_@@_overflowing_fp
  {
    \s_@@ \@@_chk:w 1 0
      { \int_eval:n { \c_@@_max_exponent_int + 1 } }
      {1000} {0000} {0000} {0000} ;
  }
%    \end{macrocode}
% \end{variable}
%
% \begin{macro}[EXP]{\@@_zero_fp:N, \@@_inf_fp:N}
%   In case of overflow or underflow, we have to output
%   a zero or infinity with a given sign.
%    \begin{macrocode}
\cs_new:Npn \@@_zero_fp:N #1
  { \s_@@ \@@_chk:w 0 #1 \s_@@_underflow ; }
\cs_new:Npn \@@_inf_fp:N #1
  { \s_@@ \@@_chk:w 2 #1 \s_@@_overflow ; }
%    \end{macrocode}
% \end{macro}
%
% \begin{macro}[EXP]{\@@_exponent:w}
%   For normal numbers, the function expands to the exponent, otherwise
%   to $0$.  This is used in \pkg{l3str-format}.
%    \begin{macrocode}
\cs_new:Npn \@@_exponent:w \s_@@ \@@_chk:w #1
  {
    \if_meaning:w 1 #1
      \exp_after:wN \@@_use_ii_until_s:nnw
    \else:
      \exp_after:wN \@@_use_i_until_s:nw
      \exp_after:wN 0
    \fi:
  }
%    \end{macrocode}
% \end{macro}
%
% \begin{macro}[EXP]{\@@_neg_sign:N}
%   When appearing in an integer expression or after \cs{int_value:w},
%   this expands to the sign opposite to |#1|, namely $0$ (positive) is
%   turned to $2$ (negative), $1$ (\texttt{nan}) to $1$, and $2$ to $0$.
%    \begin{macrocode}
\cs_new:Npn \@@_neg_sign:N #1
  { \@@_int_eval:w 2 - #1 \@@_int_eval_end: }
%    \end{macrocode}
% \end{macro}
%
% \begin{macro}[EXP]{\@@_kind:w}
%   Expands to $0$ for zeros, $1$ for normal floating point numbers, $2$
%   for infinities, $3$ for \nan{}, $4$ for tuples.
%    \begin{macrocode}
\cs_new:Npn \@@_kind:w #1
  {
    \@@_if_type_fp:NTwFw
      #1 \@@_use_ii_until_s:nnw
      \s_@@ { \@@_use_i_until_s:nw 4 }
      \q_stop
  }
%    \end{macrocode}
% \end{macro}
%
% \subsection{Overflow, underflow, and exact zero}
%
%^^A todo: the sign of exact zeros should depend on the rounding mode.
%
% \begin{macro}[EXP]{\@@_sanitize:Nw, \@@_sanitize:wN}
% \begin{macro}[EXP]{\@@_sanitize_zero:w}
%   Expects the sign and the exponent in some order, then the
%   significand (which we don't touch).  Outputs the corresponding
%   floating point number, possibly underflowed to $\pm 0$ or overflowed
%   to $\pm\infty$.  The functions \cs{@@_underflow:w} and
%   \cs{@@_overflow:w} are defined in \pkg{l3fp-traps}.
%    \begin{macrocode}
\cs_new:Npn \@@_sanitize:Nw #1 #2;
  {
    \if_case:w
        \if_int_compare:w #2 > \c_@@_max_exponent_int 1 ~ \else:
        \if_int_compare:w #2 < - \c_@@_minus_min_exponent_int 2 ~ \else:
        \if_meaning:w 1 #1 3 ~ \fi: \fi: \fi: 0 ~
    \or: \exp_after:wN \@@_overflow:w
    \or: \exp_after:wN \@@_underflow:w
    \or: \exp_after:wN \@@_sanitize_zero:w
    \fi:
    \s_@@ \@@_chk:w 1 #1 {#2}
  }
\cs_new:Npn \@@_sanitize:wN #1; #2 { \@@_sanitize:Nw #2 #1; }
\cs_new:Npn \@@_sanitize_zero:w \s_@@ \@@_chk:w #1 #2 #3;
  { \c_zero_fp }
%    \end{macrocode}
% \end{macro}
% \end{macro}
%
% \subsection{Expanding after a floating point number}
%
% \begin{macro}[EXP]{\@@_exp_after_o:w}
% \begin{macro}[EXP]{\@@_exp_after_f:nw}
%   \begin{syntax}
%     \cs{@@_exp_after_o:w} \meta{floating point}
%     \cs{@@_exp_after_f:nw} \Arg{tokens} \meta{floating point}
%   \end{syntax}
%   Places \meta{tokens} (empty in the case of \cs{@@_exp_after_o:w})
%   between the \meta{floating point} and the following tokens, then
%   hits those tokens with \texttt{o} or \texttt{f}-expansion, and
%   leaves the floating point number unchanged.
%
%   We first distinguish normal floating points, which have a significand,
%   from the much simpler special floating points.
%    \begin{macrocode}
\cs_new:Npn \@@_exp_after_o:w \s_@@ \@@_chk:w #1
  {
    \if_meaning:w 1 #1
      \exp_after:wN \@@_exp_after_normal:nNNw
    \else:
      \exp_after:wN \@@_exp_after_special:nNNw
    \fi:
    { }
    #1
  }
\cs_new:Npn \@@_exp_after_f:nw #1 \s_@@ \@@_chk:w #2
  {
    \if_meaning:w 1 #2
      \exp_after:wN \@@_exp_after_normal:nNNw
    \else:
      \exp_after:wN \@@_exp_after_special:nNNw
    \fi:
    { \exp:w \exp_end_continue_f:w #1 }
    #2
  }
%    \end{macrocode}
% \end{macro}
% \end{macro}
%
% \begin{macro}[EXP]{\@@_exp_after_special:nNNw}
%   \begin{syntax}
%     \cs{@@_exp_after_special:nNNw} \Arg{after} \meta{case} \meta{sign} \meta{scan mark} |;|
%   \end{syntax}
%   Special floating point numbers are easy to jump over since they
%   contain few tokens.
%    \begin{macrocode}
\cs_new:Npn \@@_exp_after_special:nNNw #1#2#3#4;
  {
    \exp_after:wN \s_@@
    \exp_after:wN \@@_chk:w
    \exp_after:wN #2
    \exp_after:wN #3
    \exp_after:wN #4
    \exp_after:wN ;
    #1
  }
%    \end{macrocode}
% \end{macro}
%
% \begin{macro}[EXP]{\@@_exp_after_normal:nNNw}
%   For normal floating point numbers, life is slightly harder, since we
%   have many tokens to jump over.  Here it would be slightly better if
%   the digits were not braced but instead were delimited arguments (for
%   instance delimited by |,|).  That may be changed some day.
%    \begin{macrocode}
\cs_new:Npn \@@_exp_after_normal:nNNw #1 1 #2 #3 #4#5#6#7;
  {
    \exp_after:wN \@@_exp_after_normal:Nwwwww
    \exp_after:wN #2
    \int_value:w #3   \exp_after:wN ;
    \int_value:w 1 #4 \exp_after:wN ;
    \int_value:w 1 #5 \exp_after:wN ;
    \int_value:w 1 #6 \exp_after:wN ;
    \int_value:w 1 #7 \exp_after:wN ; #1
  }
\cs_new:Npn \@@_exp_after_normal:Nwwwww
    #1 #2; 1 #3 ; 1 #4 ; 1 #5 ; 1 #6 ;
  { \s_@@ \@@_chk:w 1 #1 {#2} {#3} {#4} {#5} {#6} ; }
%    \end{macrocode}
% \end{macro}
%
% \subsection{Other floating point types}
%
% \begin{macro}{\s_@@_tuple, \@@_tuple_chk:w}
% \begin{variable}{\c_@@_empty_tuple_fp}
%   Floating point tuples take the form \cs{s_@@_tuple}
%   \cs{@@_tuple_chk:w} |{| \meta{fp 1} \meta{fp 2} \dots |}| |;| where
%   each \meta{fp} is a floating point number or tuple, hence ends with
%   |;| itself.  When a tuple is typeset, \cs{@@_tuple_chk:w} produces
%   an error, just like usual floating point numbers.
%   Tuples may have zero or one element.
%    \begin{macrocode}
\scan_new:N \s_@@_tuple
\cs_new_protected:Npn \@@_tuple_chk:w #1 ;
  { \@@_misused:n { \s_@@_tuple \@@_tuple_chk:w #1 ; } }
\tl_const:Nn \c_@@_empty_tuple_fp
  { \s_@@_tuple \@@_tuple_chk:w { } ; }
%    \end{macrocode}
% \end{variable}
% \end{macro}
%
% \begin{macro}[EXP]{\@@_tuple_count:w, \@@_array_count:n}
% \begin{macro}[EXP]{\@@_tuple_count_loop:Nw}
%   Count the number of items in a tuple of floating points by counting
%   semicolons.  The technique is very similar to \cs{tl_count:n}, but
%   with the loop built-in.  Checking for the end of the loop is done
%   with the |\use_none:n #1| construction.
%    \begin{macrocode}
\cs_new:Npn \@@_array_count:n #1
  { \@@_tuple_count:w \s_@@_tuple \@@_tuple_chk:w {#1} ; }
\cs_new:Npn \@@_tuple_count:w \s_@@_tuple \@@_tuple_chk:w #1 ;
  {
    \int_value:w \@@_int_eval:w 0
      \@@_tuple_count_loop:Nw #1 { ? \prg_break: } ;
      \prg_break_point:
    \@@_int_eval_end:
  }
\cs_new:Npn \@@_tuple_count_loop:Nw #1#2;
  { \use_none:n #1 + 1 \@@_tuple_count_loop:Nw }
%    \end{macrocode}
% \end{macro}
% \end{macro}
%
% \begin{macro}[EXP]{\@@_if_type_fp:NTwFw}
%   Used as \cs{@@_if_type_fp:NTwFw} \meta{marker} \Arg{true code}
%   \cs{s_@@} \Arg{false code} \cs{q_stop}, this test whether the
%   \meta{marker} is \cs{s_@@} or not and runs the appropriate
%   \meta{code}.  The very unusual syntax is for optimization purposes
%   as that function is used for all floating point operations.
%    \begin{macrocode}
\cs_new:Npn \@@_if_type_fp:NTwFw #1 \s_@@ #2 #3 \q_stop {#2}
%    \end{macrocode}
% \end{macro}
%
% \begin{macro}[EXP]{\@@_array_if_all_fp:nTF, \@@_array_if_all_fp_loop:w}
%   True if all items are floating point numbers.  Used for |min|.
%    \begin{macrocode}
\cs_new:Npn \@@_array_if_all_fp:nTF #1
  {
    \@@_array_if_all_fp_loop:w #1 { \s_@@ \prg_break: } ;
    \prg_break_point: \use_i:nn
  }
\cs_new:Npn \@@_array_if_all_fp_loop:w #1#2 ;
  {
    \@@_if_type_fp:NTwFw
      #1 \@@_array_if_all_fp_loop:w
      \s_@@ { \prg_break:n \use_iii:nnn }
      \q_stop
  }
%    \end{macrocode}
% \end{macro}
%
% \begin{macro}[EXP]
%   {\@@_type_from_scan:N, \@@_type_from_scan_other:N, \@@_type_from_scan:w}
%   Used as \cs{@@_type_from_scan:N} \meta{token}.
%   Grabs the pieces of the stringified \meta{token} which lies after
%   the first |s__fp|.  If the \meta{token} does not contain that
%   string, the result is |_?|.
%    \begin{macrocode}
\cs_new:Npn \@@_type_from_scan:N #1
  {
    \@@_if_type_fp:NTwFw
      #1 { }
      \s_@@ { \@@_type_from_scan_other:N #1 }
      \q_stop
  }
\cs_new:Npx \@@_type_from_scan_other:N #1
  {
    \exp_not:N \exp_after:wN \exp_not:N \@@_type_from_scan:w
    \exp_not:N \token_to_str:N #1 \exp_not:N \q_mark
      \tl_to_str:n { s_@@ _? } \exp_not:N \q_mark \exp_not:N \q_stop
  }
\exp_last_unbraced:NNNNo
  \cs_new:Npn \@@_type_from_scan:w #1
    { \tl_to_str:n { s_@@ } } #2 \q_mark #3 \q_stop {#2}
%    \end{macrocode}
% \end{macro}
%
% \begin{macro}[EXP]{\@@_change_func_type:NNN}
% \begin{macro}[EXP]{\@@_change_func_type_aux:w, \@@_change_func_type_chk:NNN}
%   Arguments are \meta{type marker} \meta{function} \meta{recovery}.
%   This gives the function obtained by placing the type after |@@|.  If
%   the function is not defined then \meta{recovery} \meta{function} is
%   used instead; however that test is not run when the \meta{type
%   marker} is \cs{s_@@}.
%    \begin{macrocode}
\cs_new:Npn \@@_change_func_type:NNN #1#2#3
  {
    \@@_if_type_fp:NTwFw
      #1 #2
      \s_@@
        {
          \exp_after:wN \@@_change_func_type_chk:NNN
          \cs:w
            @@ \@@_type_from_scan_other:N #1
            \exp_after:wN \@@_change_func_type_aux:w \token_to_str:N #2
          \cs_end:
          #2 #3
        }
      \q_stop
  }
\exp_last_unbraced:NNNNo
  \cs_new:Npn \@@_change_func_type_aux:w #1 { \tl_to_str:n { @@ } } { }
\cs_new:Npn \@@_change_func_type_chk:NNN #1#2#3
  {
    \if_meaning:w \scan_stop: #1
      \exp_after:wN #3 \exp_after:wN #2
    \else:
      \exp_after:wN #1
    \fi:
  }
%    \end{macrocode}
% \end{macro}
% \end{macro}
%
% \begin{macro}[EXP]{\@@_exp_after_any_f:Nnw, \@@_exp_after_any_f:nw}
% \begin{macro}[EXP]{\@@_exp_after_stop_f:nw}
%   The |Nnw| function simply dispatches to the appropriate
%   \cs[no-index]{@@_exp_after\ldots{}_f:nw} with \enquote{\ldots{}}
%   (either empty or |_|\meta{type}) extracted from |#1|, which should
%   start with |\s__fp|.  If it doesn't start with |\s__fp| the function
%   \cs{@@_exp_after_?_f:nw} defined in \pkg{l3fp-parse} gives an error;
%   another special \meta{type} is |stop|, useful for loops, see below.
%   The |nw| function has an important optimization for floating points
%   numbers; it also fetches its type marker |#2| from the floating
%   point.
%    \begin{macrocode}
\cs_new:Npn \@@_exp_after_any_f:Nnw #1
  { \cs:w @@_exp_after \@@_type_from_scan_other:N #1 _f:nw \cs_end: }
\cs_new:Npn \@@_exp_after_any_f:nw #1#2
  {
    \@@_if_type_fp:NTwFw
      #2 \@@_exp_after_f:nw
      \s_@@ { \@@_exp_after_any_f:Nnw #2 }
      \q_stop
    {#1} #2
  }
\cs_new_eq:NN \@@_exp_after_stop_f:nw \use_none:nn
%    \end{macrocode}
% \end{macro}
% \end{macro}
%
% \begin{macro}[EXP]{\@@_exp_after_tuple_o:w}
% \begin{macro}[EXP]{\@@_exp_after_tuple_f:nw, \@@_exp_after_array_f:w}
%   The loop works by using the |n| argument of
%   \cs{@@_exp_after_any_f:nw} to place the loop macro after the next
%   item in the tuple and expand it.
%   \begin{quote}
%     \cs{@@_exp_after_array_f:w}\\
%       \meta{fp_1} |;|\\
%       \ldots{}\\
%       \meta{fp_n} |;|\\
%       \cs{s_@@_stop}
%   \end{quote}
%    \begin{macrocode}
\cs_new:Npn \@@_exp_after_tuple_o:w
  { \@@_exp_after_tuple_f:nw { \exp_after:wN \exp_stop_f: } }
\cs_new:Npn \@@_exp_after_tuple_f:nw
  #1 \s_@@_tuple \@@_tuple_chk:w #2 ;
  {
    \exp_after:wN \s_@@_tuple
    \exp_after:wN \@@_tuple_chk:w
    \exp_after:wN {
      \exp:w \exp_end_continue_f:w
      \@@_exp_after_array_f:w #2 \s_@@_stop
    \exp_after:wN }
    \exp_after:wN ;
    \exp:w \exp_end_continue_f:w #1
  }
\cs_new:Npn \@@_exp_after_array_f:w
  { \@@_exp_after_any_f:nw { \@@_exp_after_array_f:w } }
%    \end{macrocode}
% \end{macro}
% \end{macro}
%
% \subsection{Packing digits}
%
% When a positive integer |#1| is known to be less than $10^8$, the
% following trick splits it into two blocks of $4$ digits, padding
% with zeros on the left.
% \begin{verbatim}
%   \cs_new:Npn \pack:NNNNNw #1 #2#3#4#5 #6; { {#2#3#4#5} {#6} }
%   \exp_after:wN \pack:NNNNNw
%     \__fp_int_value:w \__fp_int_eval:w 1 0000 0000 + #1 ;
% \end{verbatim}
% The idea is that adding $10^8$ to the number ensures that it has
% exactly $9$ digits, and can then easily find which digits correspond
% to what position in the number. Of course, this can be modified
% for any number of digits less or equal to~$9$ (we are limited by
% \TeX{}'s integers). This method is very heavily relied upon in
% \texttt{l3fp-basics}.
%
% More specifically, the auxiliary inserts |+ #1#2#3#4#5 ; {#6}|, which
% allows us to compute several blocks of $4$ digits in a nested manner,
% performing carries on the fly.  Say we want to compute $1\,2345 \times
% 6677\,8899$.  With simplified names, we would do
% \begin{verbatim}
%   \exp_after:wN \post_processing:w
%   \__fp_int_value:w \__fp_int_eval:w - 5 0000
%     \exp_after:wN \pack:NNNNNw
%     \__fp_int_value:w \__fp_int_eval:w 4 9995 0000
%       + 12345 * 6677
%       \exp_after:wN \pack:NNNNNw
%       \__fp_int_value:w \__fp_int_eval:w 5 0000 0000
%         + 12345 * 8899 ;
% \end{verbatim}
% The \cs{exp_after:wN} triggers \cs{int_value:w} \cs{@@_int_eval:w}, which
% starts a first computation, whose initial value is $- 5\,0000$ (the
% \enquote{leading shift}).  In that computation appears an
% \cs{exp_after:wN}, which triggers the nested computation
% \cs{int_value:w} \cs{@@_int_eval:w} with starting value $4\,9995\,0000$ (the
% \enquote{middle shift}).  That, in turn, expands \cs{exp_after:wN}
% which triggers the third computation.  The third computation's value
% is $5\,0000\,0000 + 12345 \times 8899$, which has $9$ digits. Adding
% $5\cdot 10^{8}$ to the product allowed us to know how many digits to
% expect as long as the numbers to multiply are not too big; it
% also works to some extent with negative results.  The \texttt{pack}
% function puts the last $4$ of those $9$ digits into a brace group,
% moves the semi-colon delimiter, and inserts a |+|, which combines the
% carry with the previous computation.  The shifts nicely combine into
% $5\,0000\,0000 / 10^{4} + 4\,9995\,0000 = 5\,0000\,0000$.  As long as
% the operands are in some range, the result of this second computation
% has $9$ digits.  The corresponding \texttt{pack} function,
% expanded after the result is computed, braces the last $4$ digits, and
% leaves |+| \meta{5 digits} for the initial computation.  The
% \enquote{leading shift} cancels the combination of the other shifts,
% and the |\post_processing:w| takes care of packing the last few
% digits.
%
% Admittedly, this is quite intricate.  It is probably the key in making
% \pkg{l3fp} as fast as other pure \TeX{} floating point units despite
% its increased precision.  In fact, this is used so much that we
% provide different sets of packing functions and shifts, depending on
% ranges of input.
%
% \begin{macro}[EXP]{\@@_pack:NNNNNw}
% \begin{variable}
%   {
%     \c_@@_trailing_shift_int ,
%     \c_@@_middle_shift_int   ,
%     \c_@@_leading_shift_int  ,
%   }
%   This set of shifts allows for computations involving results in the
%   range $[-4\cdot 10^{8}, 5\cdot 10^{8}-1]$.  Shifted values all have
%   exactly $9$ digits.
%    \begin{macrocode}
\int_const:Nn \c_@@_leading_shift_int  { - 5 0000 }
\int_const:Nn \c_@@_middle_shift_int   { 5 0000 *  9999 }
\int_const:Nn \c_@@_trailing_shift_int { 5 0000 * 10000 }
\cs_new:Npn \@@_pack:NNNNNw #1 #2#3#4#5 #6; { + #1#2#3#4#5 ; {#6} }
%    \end{macrocode}
% \end{variable}
% \end{macro}
%
% \begin{macro}[EXP]{\@@_pack_big:NNNNNNw}
% \begin{variable}
%   {
%     \c_@@_big_trailing_shift_int ,
%     \c_@@_big_middle_shift_int   ,
%     \c_@@_big_leading_shift_int  ,
%   }
%   This set of shifts allows for computations involving results in the
%   range $[-5\cdot 10^{8}, 6\cdot 10^{8}-1]$ (actually a bit more).
%   Shifted values all have exactly $10$ digits.  Note that the upper
%   bound is due to \TeX{}'s limit of $2^{31}-1$ on integers.  The
%   shifts are chosen to be roughly the mid-point of $10^{9}$ and
%   $2^{31}$, the two bounds on $10$-digit integers in \TeX{}.
%    \begin{macrocode}
\int_const:Nn \c_@@_big_leading_shift_int  { - 15 2374 }
\int_const:Nn \c_@@_big_middle_shift_int   { 15 2374 *  9999 }
\int_const:Nn \c_@@_big_trailing_shift_int { 15 2374 * 10000 }
\cs_new:Npn \@@_pack_big:NNNNNNw #1#2 #3#4#5#6 #7;
  { + #1#2#3#4#5#6 ; {#7} }
%    \end{macrocode}
% \end{variable}
% \end{macro}
%
% ^^A \@@_pack_Bigg:NNNNNNw = \@@_pack_big:NNNNNNw ?
% \begin{macro}[EXP]{\@@_pack_Bigg:NNNNNNw}
% \begin{variable}
%   {
%     \c_@@_Bigg_trailing_shift_int ,
%     \c_@@_Bigg_middle_shift_int   ,
%     \c_@@_Bigg_leading_shift_int  ,
%   }
%   This set of shifts allows for computations with results in the
%   range $[-1\cdot 10^{9}, 147483647]$; the end-point is $2^{31} - 1 -
%   2\cdot 10^{9} \simeq 1.47\cdot 10^{8}$.  Shifted values all have
%   exactly $10$ digits.
%    \begin{macrocode}
\int_const:Nn \c_@@_Bigg_leading_shift_int  { - 20 0000 }
\int_const:Nn \c_@@_Bigg_middle_shift_int   { 20 0000 *  9999 }
\int_const:Nn \c_@@_Bigg_trailing_shift_int { 20 0000 * 10000 }
\cs_new:Npn \@@_pack_Bigg:NNNNNNw #1#2 #3#4#5#6 #7;
  { + #1#2#3#4#5#6 ; {#7} }
%    \end{macrocode}
% \end{variable}
% \end{macro}
%
% \begin{macro}[EXP]{\@@_pack_twice_four:wNNNNNNNN}
%   \begin{syntax}
%     \cs{@@_pack_twice_four:wNNNNNNNN} \meta{tokens} |;| \meta{$\geq 8$ digits}
%   \end{syntax}
%   Grabs two sets of $4$ digits and places them before the semi-colon
%   delimiter.  Putting several copies of this function before a
%   semicolon packs more digits since each takes the digits
%   packed by the others in its first argument.
%    \begin{macrocode}
\cs_new:Npn \@@_pack_twice_four:wNNNNNNNN #1; #2#3#4#5 #6#7#8#9
  { #1 {#2#3#4#5} {#6#7#8#9} ; }
%    \end{macrocode}
% \end{macro}
%
% \begin{macro}[EXP]{\@@_pack_eight:wNNNNNNNN}
%   \begin{syntax}
%     \cs{@@_pack_eight:wNNNNNNNN} \meta{tokens} |;| \meta{$\geq 8$ digits}
%   \end{syntax}
%   Grabs one set of $8$ digits and places them before the semi-colon
%   delimiter as a single group.  Putting several copies of this
%   function before a semicolon packs more digits since each
%   takes the digits packed by the others in its first argument.
%    \begin{macrocode}
\cs_new:Npn \@@_pack_eight:wNNNNNNNN #1; #2#3#4#5 #6#7#8#9
  { #1 {#2#3#4#5#6#7#8#9} ; }
%    \end{macrocode}
% \end{macro}
%
% \begin{macro}[EXP]
%   {
%     \@@_basics_pack_low:NNNNNw,
%     \@@_basics_pack_high:NNNNNw,
%     \@@_basics_pack_high_carry:w
%   }
%   Addition and multiplication of significands are done in two steps:
%   first compute a (more or less) exact result, then round and pack
%   digits in the final (braced) form.  These functions take care of the
%   packing, with special attention given to the case where rounding has
%   caused a carry.  Since rounding can only shift the final digit by
%   $1$, a carry always produces an exact power of $10$.  Thus,
%   \cs{@@_basics_pack_high_carry:w} is always followed by four times
%   |{0000}|.
%
%   This is used in \pkg{l3fp-basics} and \pkg{l3fp-extended}.
%    \begin{macrocode}
\cs_new:Npn \@@_basics_pack_low:NNNNNw #1 #2#3#4#5 #6;
  { + #1 - 1 ; {#2#3#4#5} {#6} ; }
\cs_new:Npn \@@_basics_pack_high:NNNNNw #1 #2#3#4#5 #6;
  {
    \if_meaning:w 2 #1
      \@@_basics_pack_high_carry:w
    \fi:
    ; {#2#3#4#5} {#6}
  }
\cs_new:Npn \@@_basics_pack_high_carry:w \fi: ; #1
  { \fi: + 1 ; {1000} }
%    \end{macrocode}
% \end{macro}
%
% \begin{macro}[EXP]
%   {
%     \@@_basics_pack_weird_low:NNNNw,
%     \@@_basics_pack_weird_high:NNNNNNNNw
%   }
%   This is used in \pkg{l3fp-basics} for additions and
%   divisions.  Their syntax is confusing, hence the name.
%    \begin{macrocode}
\cs_new:Npn \@@_basics_pack_weird_low:NNNNw #1 #2#3#4 #5;
  {
    \if_meaning:w 2 #1
      + 1
    \fi:
    \@@_int_eval_end:
    #2#3#4; {#5} ;
  }
\cs_new:Npn \@@_basics_pack_weird_high:NNNNNNNNw
   1 #1#2#3#4 #5#6#7#8 #9; { ; {#1#2#3#4} {#5#6#7#8} {#9} }
%    \end{macrocode}
% \end{macro}
%
% \subsection{Decimate (dividing by a power of 10)}
%
% ^^A begin[todo]
% \begin{macro}[EXP]{\@@_decimate:nNnnnn}
%   \begin{syntax}
%     \cs{@@_decimate:nNnnnn} \Arg{shift} \meta{f_1}
%     ~~\Arg{X_1} \Arg{X_2} \Arg{X_3} \Arg{X_4}
%   \end{syntax}
%   Each \meta{X_i} consists in $4$ digits exactly,
%   and $1000\leq\meta{X_1}<9999$. The first argument determines
%   by how much we shift the digits. \meta{f_1} is called as follows:
%   \begin{syntax}
%     \meta{f_1} \meta{rounding} \Arg{X'_1} \Arg{X'_2} \meta{extra-digits} |;|
%   \end{syntax}
%   where $0\leq\meta{X'_i}<10^{8}-1$ are $8$ digit integers,
%   forming the truncation of our number. In other words,
%   \[
%   \left(
%     \sum_{i=1}^{4} \meta{X_i} \cdot 10^{-4i} \cdot 10^{-\meta{shift}}
%   \right)
%   - \bigl( \meta{X'_1} \cdot 10^{-8} + \meta{X'_2} \cdot 10^{-16} \bigr)
%   = 0.\meta{extra-digits} \cdot 10^{-16}
%   \in [0,10^{-16}).
%   \]
%   To round properly later, we need to remember some information
%   about the difference. The \meta{rounding} digit is $0$ if and
%   only if the difference is exactly $0$, and $5$ if and only if
%   the difference is exactly $0.5\cdot 10^{-16}$. Otherwise, it
%   is the (non-$0$, non-$5$) digit closest to $10^{17}$ times the
%   difference.  In particular, if the shift is $17$ or more, all
%   the digits are dropped, \meta{rounding} is $1$ (not $0$), and
%   \meta{X'_1} and \meta{X'_2} are both zero.
%
%   If the shift is $1$, the \meta{rounding} digit is simply the
%   only digit that was pushed out of the brace groups (this is
%   important for subtraction). It would be more natural for the
%   \meta{rounding} digit to be placed after the \meta{X'_i},
%   but the choice we make involves less reshuffling.
%
%   Note that this function treats negative \meta{shift} as $0$.
%    \begin{macrocode}
\cs_new:Npn \@@_decimate:nNnnnn #1
  {
    \cs:w
      @@_decimate_
      \if_int_compare:w \@@_int_eval:w #1 > \c_@@_prec_int
        tiny
      \else:
        \@@_int_to_roman:w \@@_int_eval:w #1
      \fi:
      :Nnnnn
    \cs_end:
  }
%    \end{macrocode}
%   Each of the auxiliaries see the function \meta{f_1},
%   followed by $4$ blocks of $4$ digits.
% \end{macro}
%
% \begin{macro}[EXP]{\@@_decimate_:Nnnnn, \@@_decimate_tiny:Nnnnn}
%   If the \meta{shift} is zero, or too big, life is very easy.
%    \begin{macrocode}
\cs_new:Npn \@@_decimate_:Nnnnn #1 #2#3#4#5
  { #1 0 {#2#3} {#4#5} ; }
\cs_new:Npn \@@_decimate_tiny:Nnnnn #1 #2#3#4#5
  { #1 1 { 0000 0000 } { 0000 0000 } 0 #2#3#4#5 ; }
%    \end{macrocode}
% \end{macro}
%
% \begin{macro}[EXP]
%   {
%     \@@_decimate_auxi:Nnnnn,    \@@_decimate_auxii:Nnnnn,
%     \@@_decimate_auxiii:Nnnnn,  \@@_decimate_auxiv:Nnnnn,
%     \@@_decimate_auxv:Nnnnn,    \@@_decimate_auxvi:Nnnnn,
%     \@@_decimate_auxvii:Nnnnn,  \@@_decimate_auxviii:Nnnnn,
%     \@@_decimate_auxix:Nnnnn,   \@@_decimate_auxx:Nnnnn,
%     \@@_decimate_auxxi:Nnnnn,   \@@_decimate_auxxii:Nnnnn,
%     \@@_decimate_auxxiii:Nnnnn, \@@_decimate_auxxiv:Nnnnn,
%     \@@_decimate_auxxv:Nnnnn,   \@@_decimate_auxxvi:Nnnnn
%   }
%   \begin{syntax}
%     \cs{@@_decimate_auxi:Nnnnn} \meta{f_1} \Arg{X_1} \Arg{X_2} \Arg{X_3} \Arg{X_4}
%   \end{syntax}
%   Shifting happens in two steps: compute the \meta{rounding} digit,
%   and repack digits into two blocks of $8$. The sixteen functions
%   are very similar, and defined through \cs{@@_tmp:w}.
%   The arguments are as follows: |#1| indicates which function is
%   being defined; after one step of expansion, |#2| yields the
%   \enquote{extra digits} which are then converted by
%   \cs{@@_round_digit:Nw} to the \meta{rounding} digit (note the |+|
%   separating blocks of digits to avoid overflowing \TeX{}'s integers).
%   This triggers the \texttt{f}-expansion of
%   \cs{@@_decimate_pack:nnnnnnnnnnw},\footnote{No, the argument
%     spec is not a mistake: the function calls an auxiliary to
%     do half of the job.} responsible for building two blocks of
%   $8$ digits, and removing the rest. For this to work, |#3|
%   alternates between braced and unbraced blocks of $4$ digits,
%   in such a way that the $5$ first and $5$ next token groups
%   yield the correct blocks of $8$ digits.
%    \begin{macrocode}
\cs_new:Npn \@@_tmp:w #1 #2 #3
  {
    \cs_new:cpn { @@_decimate_ #1 :Nnnnn } ##1 ##2##3##4##5
      {
        \exp_after:wN ##1
        \int_value:w
          \exp_after:wN \@@_round_digit:Nw #2 ;
        \@@_decimate_pack:nnnnnnnnnnw #3 ;
      }
  }
\@@_tmp:w {i}   {\use_none:nnn      #50}{    0{#2}#3{#4}#5               }
\@@_tmp:w {ii}  {\use_none:nn       #5 }{    00{#2}#3{#4}#5              }
\@@_tmp:w {iii} {\use_none:n        #5 }{    000{#2}#3{#4}#5             }
\@@_tmp:w {iv}  {                   #5 }{   {0000}#2{#3}#4 #5            }
\@@_tmp:w {v}   {\use_none:nnn    #4#5 }{   0{0000}#2{#3}#4 #5           }
\@@_tmp:w {vi}  {\use_none:nn     #4#5 }{   00{0000}#2{#3}#4 #5          }
\@@_tmp:w {vii} {\use_none:n      #4#5 }{   000{0000}#2{#3}#4 #5         }
\@@_tmp:w {viii}{                 #4#5 }{  {0000}0000{#2}#3 #4 #5        }
\@@_tmp:w {ix}  {\use_none:nnn  #3#4+#5}{  0{0000}0000{#2}#3 #4 #5       }
\@@_tmp:w {x}   {\use_none:nn   #3#4+#5}{  00{0000}0000{#2}#3 #4 #5      }
\@@_tmp:w {xi}  {\use_none:n    #3#4+#5}{  000{0000}0000{#2}#3 #4 #5     }
\@@_tmp:w {xii} {               #3#4+#5}{ {0000}0000{0000}#2 #3 #4 #5    }
\@@_tmp:w {xiii}{\use_none:nnn#2#3+#4#5}{ 0{0000}0000{0000}#2 #3 #4 #5   }
\@@_tmp:w {xiv} {\use_none:nn #2#3+#4#5}{ 00{0000}0000{0000}#2 #3 #4 #5  }
\@@_tmp:w {xv}  {\use_none:n  #2#3+#4#5}{ 000{0000}0000{0000}#2 #3 #4 #5 }
\@@_tmp:w {xvi} {             #2#3+#4#5}{{0000}0000{0000}0000 #2 #3 #4 #5}
%    \end{macrocode}
% \end{macro}
%
% \begin{macro}[EXP]{\@@_decimate_pack:nnnnnnnnnnw}
%   The computation of the \meta{rounding} digit leaves an unfinished
%   \cs{int_value:w}, which expands the following functions. This
%   allows us to repack nicely the digits we keep. Those digits come
%   as an alternation of unbraced and braced blocks of $4$ digits,
%   such that the first $5$ groups of token consist in $4$ single digits,
%   and one brace group (in some order), and the next $5$ have the same
%   structure. This is followed by some digits and a semicolon.
%    \begin{macrocode}
\cs_new:Npn \@@_decimate_pack:nnnnnnnnnnw #1#2#3#4#5
  { \@@_decimate_pack:nnnnnnw { #1#2#3#4#5 } }
\cs_new:Npn \@@_decimate_pack:nnnnnnw #1 #2#3#4#5#6
  { {#1} {#2#3#4#5#6} }
%    \end{macrocode}
% \end{macro}
% ^^A end[todo]
%
% \subsection{Functions for use within primitive conditional branches}
%
% The functions described in this section are not pretty and can easily
% be misused.  When correctly used, each of them removes one \cs{fi:} as
% part of its parameter text, and puts one back as part of its
% replacement text.
%
% Many computation functions in \pkg{l3fp} must perform tests on the
% type of floating points that they receive.  This is often done in an
% \cs{if_case:w} statement or another conditional statement, and only a
% few cases lead to actual computations: most of the special cases are
% treated using a few standard functions which we define now.  A typical
% use context for those functions would be
% \begin{syntax}
%   \cs{if_case:w} \meta{integer} \cs{exp_stop_f:}
%   |     |\cs{@@_case_return_o:Nw} \meta{fp var}
%   \cs{or:} \cs{@@_case_use:nw} \Arg{some computation}
%   \cs{or:} \cs{@@_case_return_same_o:w}
%   \cs{or:} \cs{@@_case_return:nw} \Arg{something}
%   \cs{fi:}
%   \meta{junk}
%   \meta{floating point}
% \end{syntax}
% In this example, the case $0$ returns the floating point
% \meta{fp~var}, expanding once after that floating point.  Case $1$
% does \meta{some computation} using the \meta{floating point}
% (presumably compute the operation requested by the user in that
% non-trivial case).  Case $2$ returns the \meta{floating point}
% without modifying it, removing the \meta{junk} and expanding once
% after.  Case $3$ closes the conditional, removes the \meta{junk}
% and the \meta{floating point}, and expands \meta{something} next.  In
% other cases, the \enquote{\meta{junk}} is expanded, performing some
% other operation on the \meta{floating point}.  We provide similar
% functions with two trailing \meta{floating points}.
%
% \begin{macro}[EXP]{\@@_case_use:nw}
%   This function ends a \TeX{} conditional, removes junk until the next
%   floating point, and places its first argument before that floating
%   point, to perform some operation on the floating point.
%    \begin{macrocode}
\cs_new:Npn \@@_case_use:nw #1#2 \fi: #3 \s_@@ { \fi: #1 \s_@@ }
%    \end{macrocode}
% \end{macro}
%
% \begin{macro}[EXP]{\@@_case_return:nw}
%   This function ends a \TeX{} conditional, removes junk and a floating
%   point, and places its first argument in the input stream.  A quirk
%   is that we don't define this function requiring a floating point to
%   follow, simply anything ending in a semicolon.  This, in turn, means
%   that the \meta{junk} may not contain semicolons.
%    \begin{macrocode}
\cs_new:Npn \@@_case_return:nw #1#2 \fi: #3 ; { \fi: #1 }
%    \end{macrocode}
% \end{macro}
%
% \begin{macro}[EXP]{\@@_case_return_o:Nw}
%   This function ends a \TeX{} conditional, removes junk and a floating
%   point, and returns its first argument (an \meta{fp~var}) then expands
%   once after it.
%    \begin{macrocode}
\cs_new:Npn \@@_case_return_o:Nw #1#2 \fi: #3 \s_@@ #4 ;
  { \fi: \exp_after:wN #1 }
%    \end{macrocode}
% \end{macro}
%
% \begin{macro}[EXP]{\@@_case_return_same_o:w}
%   This function ends a \TeX{} conditional, removes junk, and returns
%   the following floating point, expanding once after it.
%    \begin{macrocode}
\cs_new:Npn \@@_case_return_same_o:w #1 \fi: #2 \s_@@
  { \fi: \@@_exp_after_o:w \s_@@ }
%    \end{macrocode}
% \end{macro}
%
% \begin{macro}[EXP]{\@@_case_return_o:Nww}
%   Same as \cs{@@_case_return_o:Nw} but with two trailing floating
%   points.
%    \begin{macrocode}
\cs_new:Npn \@@_case_return_o:Nww #1#2 \fi: #3 \s_@@ #4 ; #5 ;
  { \fi: \exp_after:wN #1 }
%    \end{macrocode}
% \end{macro}
%
% \begin{macro}[EXP]{\@@_case_return_i_o:ww, \@@_case_return_ii_o:ww}
%   Similar to \cs{@@_case_return_same_o:w}, but this returns the first
%   or second of two trailing floating point numbers, expanding once
%   after the result.
%    \begin{macrocode}
\cs_new:Npn \@@_case_return_i_o:ww #1 \fi: #2 \s_@@ #3 ; \s_@@ #4 ;
  { \fi: \@@_exp_after_o:w \s_@@ #3 ; }
\cs_new:Npn \@@_case_return_ii_o:ww #1 \fi: #2 \s_@@ #3 ;
  { \fi: \@@_exp_after_o:w }
%    \end{macrocode}
% \end{macro}
%
% \subsection{Integer floating points}
%
% \begin{macro}[EXP, pTF]{\@@_int:w}
%   Tests if the floating point argument is an integer.  For normal
%   floating point numbers, this holds if the rounding digit resulting
%   from \cs{@@_decimate:nNnnnn} is~$0$.
%    \begin{macrocode}
\prg_new_conditional:Npnn \@@_int:w \s_@@ \@@_chk:w #1 #2 #3 #4;
  { TF , T , F , p }
  {
    \if_case:w #1 \exp_stop_f:
           \prg_return_true:
    \or:
      \if_charcode:w 0
        \@@_decimate:nNnnnn { \c_@@_prec_int - #3 }
          \@@_use_i_until_s:nw #4
        \prg_return_true:
      \else:
        \prg_return_false:
      \fi:
    \else: \prg_return_false:
    \fi:
  }
%    \end{macrocode}
% \end{macro}
%
% \subsection{Small integer floating points}
%
% \begin{macro}[EXP]{\@@_small_int:wTF}
% \begin{macro}[EXP]
%   {
%     \@@_small_int_true:wTF,
%     \@@_small_int_normal:NnwTF,
%     \@@_small_int_test:NnnwNTF
%   }
%   Tests if the floating point argument is an integer or $\pm\infty$.
%   If so, it is clipped to an integer in the range $[-10^{8},10^{8}]$
%   and fed as a braced argument to the \meta{true code}.
%   Otherwise, the \meta{false code} is performed.
%
%   First filter special cases: zeros and infinities are integers,
%   \texttt{nan} is not.  For normal numbers, decimate.  If the rounding
%   digit is not $0$ run the \meta{false code}.  If it is, then the
%   integer is |#2| |#3|; use |#3| if |#2| vanishes and otherwise
%   $10^{8}$.
%    \begin{macrocode}
\cs_new:Npn \@@_small_int:wTF \s_@@ \@@_chk:w #1#2
  {
    \if_case:w #1 \exp_stop_f:
           \@@_case_return:nw { \@@_small_int_true:wTF 0 ; }
    \or:   \exp_after:wN \@@_small_int_normal:NnwTF
    \or:
      \@@_case_return:nw
        {
          \exp_after:wN \@@_small_int_true:wTF \int_value:w
            \if_meaning:w 2 #2 - \fi: 1 0000 0000 ;
        }
    \else: \@@_case_return:nw \use_ii:nn
    \fi:
    #2
  }
\cs_new:Npn \@@_small_int_true:wTF #1; #2#3 { #2 {#1} }
\cs_new:Npn \@@_small_int_normal:NnwTF #1#2#3;
  {
    \@@_decimate:nNnnnn { \c_@@_prec_int - #2 }
      \@@_small_int_test:NnnwNw
      #3 #1
  }
\cs_new:Npn \@@_small_int_test:NnnwNw #1#2#3#4; #5
  {
    \if_meaning:w 0 #1
      \exp_after:wN \@@_small_int_true:wTF
      \int_value:w \if_meaning:w 2 #5 - \fi:
        \if_int_compare:w #2 > 0 \exp_stop_f:
          1 0000 0000
        \else:
          #3
        \fi:
      \exp_after:wN ;
    \else:
      \exp_after:wN \use_ii:nn
    \fi:
  }
%    \end{macrocode}
% \end{macro}
% \end{macro}
%
% \subsection{Fast string comparison}
%
% \begin{macro}{\@@_str_if_eq:nn}
%   A private version of the low-level string comparison function. As
%   the nature of the arguments is restricted and as speed is of the essence,
%   this version does not seek to deal with |#| tokens.
%   No \pkg{l3sys} or \pkg{l3luatex} just yet so we have to define in terms of
%   primitives.
%    \begin{macrocode}
\sys_if_engine_luatex:TF
  {
    \cs_new:Npn \@@_str_if_eq:nn #1#2
      {
        \tex_directlua:D
          {
            l3kernel.strcmp
              (
                " \tex_luaescapestring:D {#1}",
                " \tex_luaescapestring:D {#2}"
              )
          }
      }
  }
  { \cs_new_eq:NN \@@_str_if_eq:nn \tex_strcmp:D }
%    \end{macrocode}
% \end{macro}
%
% \subsection{Name of a function from its \pkg{l3fp-parse} name}
%
% \begin{macro}[EXP]{\@@_func_to_name:N, \@@_func_to_name_aux:w}
%   The goal is to convert for instance \cs{@@_sin_o:w} to |sin|.
%   This is used in error messages hence does not need to be fast.
%    \begin{macrocode}
\cs_new:Npn \@@_func_to_name:N #1
  {
    \exp_last_unbraced:Nf
      \@@_func_to_name_aux:w { \cs_to_str:N #1 } X
  }
\cs_set_protected:Npn \@@_tmp:w #1 #2
  { \cs_new:Npn \@@_func_to_name_aux:w ##1 #1 ##2 #2 ##3 X {##2} }
\exp_args:Nff \@@_tmp:w { \tl_to_str:n { @@_ } }
  { \tl_to_str:n { _o: } }
%    \end{macrocode}
% \end{macro}
%
% \subsection{Messages}
%
% Using a floating point directly is an error.
%    \begin{macrocode}
\__kernel_msg_new:nnnn { kernel } { misused-fp }
  { A~floating~point~with~value~'#1'~was~misused. }
  {
    To~obtain~the~value~of~a~floating~point~variable,~use~
    '\token_to_str:N \fp_to_decimal:N',~
    '\token_to_str:N \fp_to_tl:N',~or~other~
    conversion~functions.
  }
%    \end{macrocode}
%
%    \begin{macrocode}
%</initex|package>
%    \end{macrocode}
%
% \end{implementation}
%
% \PrintChanges
%
% \PrintIndex
