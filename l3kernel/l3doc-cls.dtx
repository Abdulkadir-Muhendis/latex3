% \iffalse meta-comment
%
%% File: l3doc.dtx
%
% Copyright (C) 1990-2019 The LaTeX3 Project
%
% It may be distributed and/or modified under the conditions of the
% LaTeX Project Public License (LPPL), either version 1.3c of this
% license or (at your option) any later version.  The latest version
% of this license is in the file
%
%    https://www.latex-project.org/lppl.txt
%
% This file is part of the "l3kernel bundle" (The Work in LPPL)
% and all files in that bundle must be distributed together.
%
% -----------------------------------------------------------------------
%
% The development version of the bundle can be found at
%
%    https://github.com/latex3/latex3
%
% for those people who are interested.
%
%
%<*driver|class>
\RequirePackage{expl3,xparse,calc}
%</driver|class>
%
%<*driver>
\ProvidesFile{l3doc.dtx}[2017/03/18 L3 Experimental documentation class]
\documentclass{l3doc}
\usepackage{framed,lipsum}
\begin{document}
  \DocInput{l3doc.dtx}
\end{document}
%</driver>
%
% This isn't included in the typeset documentation because it's a bit
% ugly:
%<*class>
\ProvidesExplClass{l3doc}{2019-08-27}{}
  {L3 Experimental documentation class}
%</class>
% \fi
%
% \title{The \cls{l3doc} class}
% \author{\Team}
% \date{Released 2019-08-25}
% \maketitle
% \tableofcontents
%
% \begin{documentation}
%
% \section{Introduction}
%
% This is an ad-hoc class for documenting the \pkg{expl3} bundle, a
% collection of modules or packages that make up \LaTeX3's programming
% environment.  Eventually it will replace the \cls{ltxdoc} class for
% \LaTeX3, but not before the good ideas in \pkg{hypdoc}, \cls{xdoc2},
% \pkg{docmfp}, and \cls{gmdoc} are incorporated.
%
% \textbf{It is much less stable than the main \pkg{expl3} packages.
%   Use at own risk!}
%
% It is written as a \enquote{self-contained} docstrip file: executing
% |latex l3doc.dtx| generates the \file{l3doc.cls} file and typesets
% this documentation; execute |tex l3doc.dtx| to only generate
% \file{l3doc.cls}.
%
% \section{Features of other packages}
%
% This class builds on the \pkg{ltxdoc} class and the \pkg{doc} package,
% but in the time since they were originally written some improvements
% and replacements have appeared that we would like to use as
% inspiration.
%
% These packages or classes are \pkg{hypdoc}, \pkg{docmfp}, \pkg{gmdoc},
% and \pkg{xdoc}.  I have summarised them below in order to work out
% what sort of features we should aim at a minimum for \pkg{l3doc}.
%
% \subsection{The \pkg{hypdoc} package}
%
% This package provides hyperlink support for the \pkg{doc} package.  I
% have included it in this list to remind me that cross-referencing
% between documentation and implementation of methods is not very
% good. (\emph{E.g.}, it would be nice to be able to automatically
% hyperlink the documentation for a function from its implementation and
% vice-versa.)
%
% \subsection{The \pkg{docmfp} package}
%
% \begin{itemize}
%   \item Provides \cs{DescribeRoutine} and the \env{routine}
%     environment (\emph{etc.}) for MetaFont and MetaPost code.
%   \item Provides \cs{DescribeVariable} and the \env{variable}
%     environment (\emph{etc.})  for more general code.
%   \item Provides \cs{Describe} and the \env{Code} environment
%     (\emph{etc.})  as a generalisation of the above two
%     instantiations.
%   \item Small tweaks to the DocStrip system to aid non-\LaTeX{} use.
% \end{itemize}
%
% \subsection{The \pkg{xdoc2} package}
%
% \begin{itemize}
%   \item Two-sided printing.
%   \item \cs{NewMacroEnvironment}, \cs{NewDescribeEnvironment}; similar
%     idea to \pkg{docmfp} but more comprehensive.
%   \item Tons of small improvements.
% \end{itemize}
%
% \subsection{The \pkg{gmdoc} package}
%
% Radical re-implementation of \pkg{doc} as a package or class.
% \begin{itemize}
%   \item Requires no |\begin{macrocode}| blocks!
%   \item Automatically inserts |\begin{macro}| blocks!
%   \item And a whole bunch of other little things.
% \end{itemize}
%
% \section{Problems \& Todo}
%
% Problems at the moment:
% (1)~not flexible in the types of things that can be documented;
% (2)~no obvious link between the |\begin{function}| environment for
%     documenting things to the |\begin{macro}| function that's used
%     analogously in the implementation.
%
% The \env{macro} should probably be renamed to \env{function} when it
% is used within an implementation section.  But they should have the
% same syntax before that happens!
%
% Furthermore, we need another \enquote{layer} of documentation commands
% to account for \enquote{user-macro} as opposed to
% \enquote{code-functions}; the \pkg{expl3} functions should be
% documented differently, probably, to the \pkg{xparse} user macros (at
% least in terms of indexing).
%
% In no particular order, a list of things to do:
% \begin{itemize}
%   \item Rename \env{function}/\env{macro} environments to better
%     describe their use.
%   \item Generalise \env{function}/\env{macro} for documenting
%     \enquote{other things}, such as environment names, package
%     options, even keyval options.
%   \item New function like \tn{part} but for files (remove awkward
%     \enquote{File} as \tn{partname}).
%   \item Something better to replace \cs{StopEventually}; I'm thinking
%     two environments \env{documentation} and \env{implementation} that
%     can conditionally typeset/ignore their material.  (This has been
%     implemented but needs further consideration.)
%   \item Hyperlink documentation and implementation of macros (see the
%     \textsc{dtx} file of \pkg{svn-multi} v2 as an example).  This is
%     partially done, now, but should be improved.
% \end{itemize}
%
% \section{Documentation}
%
% \subsection{Configuration}
%
% Before class options are processed, \pkg{l3doc} loads a configuration
% file \file{l3doc.cfg} if it exists, allowing you to customise the
% behaviour of the class without having to change the documentation
% source files.
%
% For example, to produce documentation on letter-sized paper instead of
% the default A4 size, create \file{l3doc.cfg} and include the line
% \begin{verbatim}
% \PassOptionsToClass{letterpaper}{l3doc}
% \end{verbatim}
%
% By default, \pkg{l3doc} selects the |T1| font encoding and loads the
% Latin Modern fonts.  To prevent this, use the class option
% |cm-default|.
%
% \subsection{Partitioning documentation and implementation}
%
% \pkg{doc} uses the \cs{OnlyDocumentation}/\cs{AlsoImplementation}
% macros to guide the use of \cs{StopEventually}|{}|, which is intended
% to be placed to partition the documentation and implementation within
% a single \file{.dtx} file.
%
% This isn't very flexible, since it assumes that we \emph{always} want
% to print the documentation.  For the \pkg{expl3} sources, I wanted to
% be be able to input \file{.dtx} files in two modes: only displaying
% the documentation, and only displaying the implementation.  For
% example:
% \begin{verbatim}
% \DisableImplementation
% \DocInput{l3basics,l3prg,...}
% \EnableImplementation
% \DisableDocumentation
% \DocInputAgain
% \end{verbatim}
%
% The idea being that the entire \pkg{expl3} bundle can be documented,
% with the implementation included at the back.  Now, this isn't
% perfect, but it's a start.
%
% Use |\begin{documentation}...\end{documentation}| around the
% documentation, and |\begin{implementation}...\end{implementation}|
% around the implementation.  The
% \cs{EnableDocumentation}/\cs{EnableImplementation} causes them to
% be typeset when the \file{.dtx} file is \cs{DocInput}; use
% \cs{DisableDocumentation}/\cs{DisableImplementation} to omit the
% contents of those environments.
%
% Note that \cs{DocInput} now takes comma-separated arguments, and
% \cs{DocInputAgain} can be used to re-input all \file{.dtx} files
% previously input in this way.
%
% \subsection{General text markup}
%
% Many of the commands in this section come from \pkg{ltxdoc} with some
% improvements.
%
% \begin{function}{\cmd, \cs}
%   \begin{syntax}
%     \cmd{\cmd} \oarg{options} \meta{control sequence}\\
%     \cs{cs} \oarg{options} \marg{csname}
%   \end{syntax}
%   These commands are provided to typeset control sequences.
%   |\cmd\foo| produces \enquote{\cmd\foo} and |\cs{foo}| produces the
%   same.  In general, \cs{cs} is more robust since
%   it doesn't rely on catcodes being \enquote{correct} and is therefore
%   recommended.
%
%   These commands are aware of the |@@| \pkg{l3docstrip} syntax and
%   replace such instances correctly in the typeset documentation.
%   This only happens after a |%<@@=|\meta{module}|>| declaration.
%
%   Additionally, commands can be used in the argument of \cs{cs}.  For
%   instance, |\cs{\meta{name}:\meta{signature}}| produces
%   \cs[no-index]{\meta{name}:\meta{signature}}.
%
%   The \meta{options} are a key--value list which can contain the
%   following keys:
%   \begin{itemize}
%     \item |index=|\meta{name}: the \meta{csname} is indexed as if
%       one had written \cs{cs}\Arg{name}.
%     \item |no-index|: the \meta{csname} is not indexed.
%     \item |module=|\meta{module}: the \meta{csname} is indexed in
%       the list of commands from the \meta{module}; the \meta{module}
%       can in particular be |TeX| for \enquote{\TeX{} and \LaTeXe{}}
%       commands, or empty for commands which should be placed in the
%       main index.  By default, the \meta{module} is deduced
%       automatically from the command name.
%     \item |replace| is a boolean key (\texttt{true} by default) which
%       indicates whether to replace |@@| as \pkg{l3docstrip} does.
%   \end{itemize}
%   These commands allow hyphenation of control sequences after (most) underscores.
%   By default, a hyphen is used to mark the hyphenation, but this can be changed with
%   the \texttt{cs-break-nohyphen} class option.
%   To disable hyphenation of control sequencies entirely, use \texttt{cs-break-off}.
% \end{function}
%
%
% \begin{function}{\tn}
%   \begin{syntax}
%     \cs{tn} \oarg{options} \marg{csname}
%   \end{syntax}
%   Analoguous to \cs{cs} but intended for \enquote{traditional} \TeX{}
%   or \LaTeXe{} commands; they are indexed accordingly.  This is in
%   fact equivalent to \cs{cs} |[module=TeX, replace=false,|
%   \meta{options}|]| \Arg{csname}.
% \end{function}
%
% \begin{function}{\meta}
%   \begin{syntax}
%     \cs{meta} \Arg{name}
%   \end{syntax}
%   \cs{meta} typesets the \meta{name} italicised in \meta{angle
%     brackets}.  Within a \env{function} environment or similar, angle
%   brackets |<...>| are set up to be a shorthand for |\meta{...}|.
%
%   This function has additional functionality over its \pkg{ltxdoc}
%   versions; underscores can be used to subscript material as in math
%   mode.  For example, |\meta{arg_{xy}}| produces
%   \enquote{\meta{arg_{xy}}}.
% \end{function}
%
% \begin{function}{\Arg, \marg, \oarg, \parg}
%   \begin{syntax}
%     |\Arg| \Arg{name}
%   \end{syntax}
%   Typesets the \meta{name} as for \cs{meta} and wraps it in braces.
%
%   The \cs{marg}/\cs{oarg}/\cs{parg} versions follow from \pkg{ltxdoc}
%   in being used for \enquote{mandatory} or \enquote{optional} or
%   \enquote{picture} brackets as per \LaTeXe{} syntax.
% \end{function}
%
% \begin{function}{\file, \env, \pkg, \cls}
%   \begin{syntax}
%     \cs{pkg} \Arg{name}
%   \end{syntax}
%   These all take one argument and are intended to be used as semantic
%   commands for representing files, environments, package names, and
%   class names, respectively.
% \end{function}
%
% \begin{function}{\NB, \NOTE}
%   \begin{syntax}
%     \cs{NB} \marg{tag} \marg{comments}
%     \verb|\begin{NOTE}| \marg{tag}
%     \qquad\meta{comments}
%     \verb|\end{NOTE}|
%   \end{syntax}
%   Make notes in the source that are not typeset by default. When the \verb|show-notes|
%   class option is active, the comments are typeset in a detokenized and verbatim mode, respectively.
% \end{function}
%
% \subsection{Describing functions in the documentation}
%
% \DescribeEnv{function}
% \DescribeEnv{syntax}
% Two heavily-used environments are defined to describe the syntax of
% \pkg{expl3} functions and variables.
% \begin{framed}
%   \vspace{-\baselineskip}
% \begin{verbatim}
% \begin{function}{\function_one:, \function_two:}
%   \begin{syntax}
%     |\foo_bar:| \Arg{meta} \meta{test_1}
%   \end{syntax}
% \meta{description}
% \end{function}
% \end{verbatim}
%   \hrulefill
%   \par
%   \hspace*{0.25\textwidth}
%   \begin{minipage}{0.5\textwidth}
%     \begin{function}{\function_one:, \function_two:}
%       \begin{syntax}
%         |\foo_bar:| \Arg{meta} \meta{test_1}
%       \end{syntax}
%       \meta{description}
%     \end{function}
%   \end{minipage}
% \end{framed}
%
% Function environments take an optional argument to indicate whether
% the function(s) it describes are expandable or restricted-expandable
% or defined in conditional forms. Use |EXP|, |rEXP|, |TF|, |pTF|, or |noTF| for
% this; note that |pTF| implies |EXP| since predicates must always be
% expandable, and that |noTF| means that the function without |TF|
% should be documented in addition to |TF|.  As an example:
% \begin{framed}
%   \vspace{-\baselineskip}
% \begin{verbatim}
% \begin{function}[pTF]{\cs_if_exist:N}
%   \begin{syntax}
%     \cs{cs_if_exist_p:N} \meta{cs}
%   \end{syntax}
% \meta{description}
% \end{function}
% \end{verbatim}
%   \hrulefill
%   \par
%   \hspace*{0.25\textwidth}
%   \begin{minipage}{0.5\textwidth}
%     \begin{function}[pTF]{\cs_if_exist:N}
%       \begin{syntax}
%         \cs{cs_if_exist_p:N} \meta{cs}
%       \end{syntax}
%       \meta{description}
%     \end{function}
%   \end{minipage}
% \end{framed}
%
% \DescribeEnv{variable}
% If you are documenting a variable instead of a function, use the
% \env{variable} environment instead; it behaves identically to the
% \env{function} environment above.
%
% \DescribeEnv{texnote}
% This environment is used to call out sections within \env{function}
% and similar that are only of interest to seasoned \TeX{} developers.
%
% \subsection{Describing functions in the implementation}
%
% \DescribeEnv{macro}
% The well-used environment from \LaTeXe{} for marking up the
% implementation of macros/functions remains the \env{macro}
% environment.  Some changes in \pkg{l3doc}: it now accepts
% comma-separated lists of functions, to avoid a very large number of
% consecutive |\end{macro}| statements.
% Spaces and new lines are ignored (the option |[verb]| prevents this).
% \begin{verbatim}
% % \begin{macro}{\foo:N, \foo:c}
% %   \begin{macrocode}
% ... code for \foo:N and \foo:c ...
% %   \end{macrocode}
% % \end{macro}
% \end{verbatim}
% If you are documenting an auxiliary macro, it's generally not
% necessary to highlight it as much and you also don't need to check it
% for, say, having a test function and having a documentation chunk
% earlier in a \env{function} environment.  \pkg{l3doc} will pick up these
% cases from the presence of |__| in the name, or you may force marking
% as internal by using |\begin{macro}[int]| to mark it as such. The margin
% call-out is then printed in grey for such cases.
%
% For documenting \pkg{expl3}-type conditionals, you may also pass this
% environment a |TF| option (and omit it from the function name) to
% denote that the function is provided with |T|, |F|, and |TF| suffixes.
% A similar |pTF| option prints both |TF| and |_p| predicate forms.
% An option |noTF| prints both the |TF| forms and a form with neither
% |T| nor |F|, to document functions such as \cs[no-index]{prop_get:NN}
% which also have conditional forms (\cs[no-index]{prop_get:NNTF}).
%
%
% \DescribeMacro{\TestFiles}
% \cs{TestFiles}\marg{list of files} is used to indicate which test
% files are used for the current code; they are printed in the
% documentation.
%
% \DescribeMacro{\UnitTested}
% Within a \env{macro} environment, it is a good idea to mark whether a
% unit test has been created for the commands it defines.  This is
% indicated by writing \cs{UnitTested} anywhere within |\begin{macro}|
%   \dots |\end{macro}|.
%
% If the class option |checktest| is enabled, then it is an \emph{error}
% to have a \env{macro} environment without a call to
% \file{Testfiles}.  This is intended for large packages such as
% \pkg{expl3} that should have absolutely comprehensive tests suites and
% whose authors may not always be as sharp at adding new tests with new
% code as they should be.
%
% \DescribeMacro{\TestMissing}
% If a function is missing a test, this may be flagged by writing (as
% many times as needed) \cs{TestMissing} \marg{explanation of test
%   required}.  These missing tests are summarised in the listing
% printed at the end of the compilation run.
%
% \DescribeEnv{variable}
% When documenting variable definitions, use the \env{variable}
% environment instead.  Here it behaves identically to the
% \env{macro} environment, except that if the class option |checktest|
% is enabled, variables are not required to have a test file.
%
% \DescribeEnv{arguments}
% Within a \env{macro} environment, you may use the \env{arguments}
% environment to describe the arguments taken by the function(s).  It
% behaves like a modified enumerate environment.
% \begin{verbatim}
% % \begin{macro}{\foo:nn, \foo:VV}
% % \begin{arguments}
% %   \item Name of froozle to be frazzled
% %   \item Name of muble to be jubled
% % \end{arguments}
% %   \begin{macrocode}
% ... code for \foo:nn and \foo:VV ...
% %   \end{macrocode}
% % \end{macro}
% \end{verbatim}
%
%
% \subsection{Keeping things consistent}
%
% Whenever a function is either documented or defined with
% \env{function} and \env{macro} respectively, its name is stored in a
% sequence for later processing.
%
% At the end of the document (\emph{i.e.}, after the \file{.dtx} file
% has finished processing), the list of names is analysed to check
% whether all defined functions have been documented and vice versa. The
% results are printed in the console output.
%
% If you need to do more serious work with these lists of names, take a
% look at the implementation for the data structures and methods used to
% store and access them directly.
%
% \subsection{Documenting templates}
%
% The following macros are provided for documenting templates; might end
% up being something completely different but who knows.
% \begin{quote}\parskip=0pt\obeylines
%   |\begin{TemplateInterfaceDescription}| \Arg{template type name}
%   |  \TemplateArgument{none}{---}|
%   \textsc{or one or more of these:}
%   |  \TemplateArgument| \Arg{arg no} \Arg{meaning}
%   \textsc{and}
%   |\TemplateSemantics|
%   |  | \meta{text describing the template type semantics}
%   |\end{TemplateInterfaceDescription}|
% \end{quote}
%
% \begin{quote}\parskip=0pt\obeylines
%   |\begin{TemplateDescription}| \Arg{template type name} \Arg{name}
%   \textsc{one or more of these:}
%   |  \TemplateKey| \marg{key name} \marg{type of key}
%   |    |\marg{textual description of meaning}
%   |    |\marg{default value if any}
%   \textsc{and}
%   |\TemplateSemantics|
%   |  | \meta{text describing special additional semantics of the template}
%   |\end{TemplateDescription}|
% \end{quote}
%
% \begin{quote}\parskip=0pt\obeylines
%   |\begin{InstanceDescription}| \oarg{text to specify key column width (optional)}
%   \hfill\marg{template type name}\marg{instance name}\marg{template name}
%   \textsc{one or more of these:}
%   |  \InstanceKey| \marg{key name} \marg{value}
%   \textsc{and}
%   |\InstanceSemantics|
%   |  | \meta{text describing the result of this instance}
%   |\end{InstanceDescription}|
% \end{quote}
%
% \end{documentation}
%
% \begin{implementation}
%
% \section{\pkg{l3doc} implementation}
%
%    \begin{macrocode}
%<@@=codedoc>
%    \end{macrocode}
%
%    \begin{macrocode}
%<*class>
%    \end{macrocode}
%
% \subsection{Variables}
%
% \begin{variable}{\g_@@_lmodern_bool}
%   Boolean option whether to use \pkg{lmodern} instead of the default
%   Computer Modern font.
%    \begin{macrocode}
\bool_new:N \g_@@_lmodern_bool
%    \end{macrocode}
% \end{variable}
%
% \begin{variable}{\l_@@_tmpa_dim}
%   A temporary variables dimension variable.
%    \begin{macrocode}
\dim_new:N \l_@@_tmpa_dim
%    \end{macrocode}
% \end{variable}
%
% \subsection{Class options and configuration}
%
% Make the \opt{a5paper} option raise an error.
%    \begin{macrocode}
\DeclareOption { a5paper } { \@latexerr { Option~not~supported } { } }
%    \end{macrocode}
%
% Option to use the \pkg{lmodern} font instead of the default Computer
% Modern.
%    \begin{macrocode}
\DeclareOption { cm-default }
  { \bool_gset_false:N \g_@@_lmodern_bool }
\DeclareOption { lm-default }
  { \bool_gset_true:N \g_@@_lmodern_bool }
%    \end{macrocode}
%
% Maintain support for old class options which now belong to the
% \pkg{l3doc} package.
%    \begin{macrocode}
\DeclareOption { full }
  { \PassOptionsToPackage { full } { l3doc } }
\DeclareOption { onlydoc }
  { \PassOptionsToPackage { onlydoc } { l3doc } }
\DeclareOption { check }
  { \PassOptionsToPackage { check } { l3doc } }
\DeclareOption { nocheck }
  { \PassOptionsToPackage { nocheck } { l3doc } }
\DeclareOption { checktest }
  { \PassOptionsToPackage { checktest } { l3doc } }
\DeclareOption { nochecktest }
  { \PassOptionsToPackage { nochecktest } { l3doc } }
%    \end{macrocode}
%
% All remaining options are passed to the base \cls{article} class.
%    \begin{macrocode}
\DeclareOption* { \PassOptionsToClass { \CurrentOption } { article } }
\ExecuteOptions { lm-default }
\PassOptionsToClass { a4paper } { article }
%    \end{macrocode}
%
% Input a local configuration file, if it exists, with a message to the
% console that this has happened. Since we distribute a \file{.cfg} file
% with the class, this should usually always be true. Therefore, check
% for \cs{ExplMakeTitle} (defined in \enquote{our} \file{.cfg} file) and
% only output the informational message if it's not found.
%    \begin{macrocode}
\msg_new:nnn { l3doc } { input-cfg }
  { Local~config~file~l3doc.cfg~loaded. }
\file_if_exist:nT { l3doc.cfg }
  {
    \file_input:nT { l3doc.cfg }
      {
        \cs_if_exist:NF \ExplMakeTitle
          { \msg_info:nn { l3doc } { input-cfg } }
      }
  }
%    \end{macrocode}
%
%    \begin{macrocode}
\ProcessOptions
%    \end{macrocode}
%
% Depending on the option, load the package \pkg{lmodern} to set the
% font.  Then replace the italic typewriter font with the oblique shape
% instead; the former makes my skin crawl. (Will, Aug 2011)
% ^^A      Thanks Will, you ruined that font for me (you're right :-).
%    \begin{macrocode}
\bool_if:NT \g_@@_lmodern_bool
  {
    \RequirePackage[T1]{fontenc}
    \RequirePackage{lmodern}
    \group_begin:
      \ttfamily
      \DeclareFontShape{T1}{lmtt}{m}{it}{<->ec-lmtto10}{}
    \group_end:
  }
%    \end{macrocode}
%
% \subsection{Class and package loading}
%
%    \begin{macrocode}
\LoadClass{article}
\RequirePackage{doc}
\RequirePackage
  {
    array,% Okay
    alphalph,% Used in \DocInclude to redefine \thepart
    amsmath,% Okay
    amssymb,% Okay
    booktabs,% Used \...rule, used in syntax, function, and variable environments
    color, % Okay
    colortbl, % \arrayrulecolor, used in syntax environment
    hologo, % Used for logos :-)
    enumitem, % Used for the arguments environment
    pifont, % hollow star for rEXP
    textcomp, % Okay
    trace, % Okay?
    csquotes, % Okay
    fancyvrb, % Used for typesetting verbatim
    underscore, % Breakable underscores (expl3 has some :-)
    verbatim, % Okay
  }
\raggedbottom
%    \end{macrocode}
%
% Must be last, as usual.
%    \begin{macrocode}
\RequirePackage{hypdoc}
%    \end{macrocode}
%
% \subsection{Configuration and tweaks}
%
% Increase the text width slightly so that with the width the standard
% fonts, 72~columns of code may appear in a \env{macrocode} environment.
% Increase the marginpar width slightly, for long command names.  And
% increase the left margin by a similar amount.
%    \begin{macrocode}
\dim_set:Nn \textwidth      { 385 pt }
\dim_add:Nn \marginparwidth {  30 pt }
\dim_add:Nn \oddsidemargin  {  20 pt }
\dim_add:Nn \evensidemargin {  20 pt }
%    \end{macrocode}
% (These were introduced when \cls{article} was the documentclass, but
% I've left them here for now to remind me to do something about them
% later.)
%
% \begin{macro}{\list}
% \begin{macro}{\@@_oldlist:nn}
%   Customise lists.
%    \begin{macrocode}
\cs_new_eq:NN \@@_oldlist:nn \list
\cs_gset:Npn \list #1 #2
  { \@@_oldlist:nn {#1} { #2 \dim_zero:N \listparindent } }
\dim_set:Nn \parindent  { 2 em }
\dim_set:Nn \itemindent { 0 pt }
\dim_set:Nn \parskip    { 0 pt plus 3 pt minus 0 pt }
%    \end{macrocode}
% \end{macro}
% \end{macro}
%
% \begin{macro}{\partname}
%   Use \enquote{File} as a name in Part titles.
%    \begin{macrocode}
\tl_gset:Nn \partname {File}
%    \end{macrocode}
% \end{macro}
%
% \begin{macro}{\l@section, \l@subsection}
%   Customise the table of contents (as we have so many sections).
%   Different design and/or structure is called for).
%    \begin{macrocode}
\@addtoreset { section } { part }
\cs_gset:Npn \l@section #1#2
  {
    \int_compare:nNnT { \c@tocdepth } > { \c_zero_int }
      {
        \addpenalty \@secpenalty
        \addvspace { 1.0 em \@plus \p@ }
        \dim_set:Nn \l_@@_tmpa_dim { 2.5 em }  % was 1.5em
        \group_begin:
          \dim_set_eq:NN \parindent \c_zero_dim
          \dim_set_eq:NN \rightskip \@pnumwidth
          \dim_set:Nn \parfillskip { -\@pnumwidth }
          \mode_leave_vertical:
          \bfseries
          \dim_add:Nn \leftskip { \l_@@_tmpa_dim }
          \skip_horizontal:n { -\leftskip }
          #1 \nobreak \tex_hfil:D \nobreak
          \hbox_to_wd:nn { \@pnumwidth } { \tex_hss:D #2 }
          \par
        \group_end:
      }
  }
\cs_gset:Npn \l@subsection
  { \@dottedtocline { 2 } { 2.5 em } { 2.3 em } }  % #2 = 1.5em
%    \end{macrocode}
% \end{macro}
%
% Now load the \pkg{l3doc} package to do the heavy-lifting.
%    \begin{macrocode}
\RequirePackage [ full , kernel , nocheck , nochecktest ] { l3doc }
%    \end{macrocode}
%
%
% \subsection{Internal macros for \LaTeX3 sources}
%
% These definitions are only used by the \LaTeX3 documentation; they are
% not necessary for third-party users of \cls{l3doc}.  In time this will
% be broken into a separate package that is specifically loaded in the
% various \pkg{expl3} modules, \emph{etc.}
%
%    \begin{macrocode}
%<*cfg>
%    \end{macrocode}
%
% The Guilty Parties.
%    \begin{macrocode}
\tl_const:Nn \Team
  {
    The~\LaTeX3~Project\thanks
      { \url{https://www.latex-project.org/latex3/} }
  }
%    \end{macrocode}
%
%    \begin{macrocode}
\NewDocumentCommand { \ExplMakeTitle } { m m }
  {
    \title
      { The~\pkg{#1}~package \\ #2 }
    \author
      {
        The~\LaTeX3~Project
        \thanks
          {
            E-mail:~
            \href{mailto:latex-team@latex-project.org}
                        {latex-team@latex-project.org}
          }
      }
    \date { Released~\ExplFileDate }
    \maketitle
  }
%    \end{macrocode}
%
% \subsection{Math extras}
%
% For \pkg{l3fp}.
%    \begin{macrocode}
\AtBeginDocument
  {
    \clist_map_inline:nn
      {
        asin, acos, atan, acot,
        asinh, acosh, atanh, acoth,
        round, floor, ceil
      }
      { \exp_args:Nc \DeclareMathOperator {#1} {#1} }
  }
%    \end{macrocode}
%
% \begin{macro}{\nan}
%    \begin{macrocode}
\NewDocumentCommand { \nan } { } { \text { \texttt { nan } } }
%    \end{macrocode}
% \end{macro}
%
%    \begin{macrocode}
%</cfg>
%    \end{macrocode}
%
% \end{implementation}
%
% \PrintIndex
